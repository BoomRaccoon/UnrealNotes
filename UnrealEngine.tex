\documentclass{scrbook}
%\usepackage{showframe}
\usepackage[utf8]{inputenc}
\usepackage[T1]{fontenc}
\usepackage{xcolor}
\usepackage{lmodern}
\usepackage[ngerman, english]{babel}
\usepackage{amsmath}
\usepackage{listings}
\usepackage[colorlinks]{hyperref}
\hypersetup{
    colorlinks = true,
    linkcolor = green,
    urlcolor = blue
}

\usepackage[backend=biber]{biblatex}
\usepackage[xindy, toc, nonumberlist]{glossaries}
\usepackage{float}
\usepackage{enumitem}
\usepackage{tabularray}
\usepackage{makecell}
\usepackage{ulem}
\usepackage{arydshln}
\usepackage{mathabx}
\setlist{nosep}
\usepackage{graphicx}
\graphicspath{{Bilder/}}	 % pfad zum Verzeichnis das die Bilder enthält
\usepackage{tikz}
\usetikzlibrary{positioning}
\usepackage{tcolorbox}
\usepackage{multicol}
\usepackage{csquotes}
\usepackage{lscape}
\usepackage{pdfcomment}
\usepackage[includeheadfoot]{geometry}
\geometry
{
    top=10mm,
    left=10mm,
    right=5mm,
}
\author{}
\newcommand{\code}[1]{\colorbox{mygray}{\lstinline|#1|}}

\usepackage{fancyhdr}
\pagestyle{fancy}


\loadglsentries{/Gdrive/Glossaries/CG_Glossary.tex}

\makeglossaries
\definecolor{mygray}{rgb}{0.8,0.8,0.8}
\lstset{
    backgroundcolor=\color{mygray},
    basicstyle=\ttfamily,
    columns=fullflexible,
    frame=single,
    breaklines=true,
    postbreak=\mbox{\textcolor{red}{$\hookrightarrow$}\space}
}
\title{Learning Unreal}
    
\begin{document}
\newglossaryentry{Controller}{
    name={Controller},
    description={Is a non-physical Actor that can posses a Pawn/Pawn-Derived. It is, in contrast to AIController, used for human player to control pawns.
    In order to controll the character the Posses function is used. },
    plural={}
}


\newglossaryentry{UPROPERTY}{
    name={UPROPERTY},
    description={is placed in front of a Variable declaration in order to make is accessible to the UE},
    plural={}
}

\newglossaryentry{Asset}{
    name={Asset},
    description={
        is just a set of UObjects serialized to disk },
    plural={}
}

\newglossaryentry{Package}{
    name={Package},
    description={
        UPackage is the top-level object inside an asset file and can have many objects inside it
        (somethings like /Game/MyPackage.uobject0 and /Game/MyPackage.uobject1)},
    plural={}
}

\newglossaryentry{Collisio Channel}{
    name={Collision Channel},
    description={
        Defines how the object will react to traces. },
    plural={}
}

\newglossaryentry{Collision Preset}{
    name={Collision Preset},
    description={
        Each Preset contains the Object Type that should be assigned
        and the responses for Trace and Object Channel},
    plural={}
}


\newglossaryentry{Locomotion}{
    name={Locomotion},
    description={Advanced setup for animations with blendspaces},
    plural={}
}


\newglossaryentry{Delegate}{
    name={Delegate},
    description={Binds a function to an event(which is Broadcast).},
    plural={}
}

\newglossaryentry{Collision Response (ECR)}{
    name={ECR},
    description={The Enum ECollisionResponse is made up of Ignore, Overlap and Block and defines how objects collide (interact with each other)},
    plural={}
}

\newglossaryentry{Procedure}{
    name={Procedure},
    description={Function with no return value},
    plural={}
}

\newglossaryentry{Animation Graph}{
    name={Animation Graph},
    description={Contains the logic of the Animation-Blueprint. A State-Machine will determine which animation should be played.},
    plural={}
}

\newglossaryentry{CRTP}{
    name={curiously recurring template pattern},
    description={
        Solves the problem of slicing (of information) by inheriting from a templatet class
        that takes in the defined class as the Template parameter.
        },
    plural={}
}

\newglossaryentry{Animation Modifier}{
    name={Animation Modifier},
    description={},
    plural={}
}

\newglossaryentry{Behavior Trees}{
    name={Behavior Trees},
    description={Stores information about the states you can be in, and actions/tasks to perform depending on those states.},
    plural={}
}

\newglossaryentry{Blackboard}{
    name={Blackboard},
    description={Stores the information for the different branches of the behavior tree to make the decisions},
    plural={}
}

\newglossaryentry{EQS}{
    name={EQS},
    description={Environmental-Query-System },
    plural={}
}

\newglossaryentry{Nav Mesh Volumes}{
    name={Nav Mesh Volumes},
    description={},
    plural={}
}


\newglossaryentry{AI Perception}{
    name={AI Perception},
    description={Sends stiumli(information about what's happening in the surrounding area) to the behavior tree},
    plural={}
}

\newglossaryentry{Cooking}{
    name={Cooking},
    description={
        is the process of converting from an internal asset format to a platform-specific format.},
    plural={}
}

\newglossaryentry{Modal window}{
    name={Modal window},
    description={
        a window that is on top of the main window,
        disables it and must be interacted with first before returning control to the main window},
    plural={}
}

\newglossaryentry{FField}{
    name={FField},
    description={
        Base class for refltion data objects},
    plural={}
}

\newglossaryentry{UMG}{
    name={UMG},
    description={Unreal-Motion-Graphics is a visual UI builder},
    plural={}
}

\newglossaryentry{Slate}{
    name={Slate},
    description={More advanced way to build an interface using C++. instead of the editor and visual scripting},
    plural={}
}

\newglossaryentry{Level Streaming}{
    name={Level Streaming},
    description={Is a Asynchronous level loading mechanism to decrease memory usage by loading and unloading maps dynamically and thus creating seamless worlds},
    plural={}
}

\newglossaryentry{Texture Streaming}{
    name={Texture Streaming},
    description={In essence just LOD for textures. Manages the texture size depending on distance.},
    plural={}
}

\newglossaryentry{Object wild cards}{
    name={Object wild cards},
    description={Are the equivalent to generics in C++. Object Wild Card in Blueprint == Generic in C++},
    plural={}
}

\newglossaryentry{Pool}{
    name={Pool},
    description={Refers to reserved (conceptual) memory and does not relate to acutal taken memory.},
    plural={}
}

\newglossaryentry{SSGI}{
    name={SSGI},
    description={Is a technique to add dynamic indirect lighting. Best use with  lightmass.},
    plural={}
}

\newglossaryentry{Texel}{
    name={Texel},
    description={Is a pixel of the texture map},
    plural={}
}

\newglossaryentry{IES}{
    name={IES},
    description={Illumination Engineering Society }, %TODO:
    plural={}
}

\newglossaryentry{Cached Shadow Map}{
    name={Cached Shadow Map},
    description={}, %TODO:
    plural={}
}

\newglossaryentry{Volumetric Lightmaps}{
    name={Volumetric Lightmaps},
    description={}, %TODO:
    plural={}
}

\newglossaryentry{Bounds}{
    name={Bounds},
    description={Are boxes drawn around the objects.},
    plural={}
}

\newglossaryentry{Gameplay Timer}{
    name={Gameplay Timer},
    description={Schedules actions.},
    plural={}
}

\newglossaryentry{Timer Manager}{
    name={Timer Manager},
    description={Manages all the Gameplay Timer.},
    plural={}
}

\newglossaryentry{streaming distance}{
    name={streaming distance},
    description={},
    plural={}
}

\newglossaryentry{Build rules}{
    name={Build rules},
    description={Script to specify the referenced modules by your custom module},
    plural={}
}

\newglossaryentry{Target rules}{
    name={Target rules},
    description={Script to specify },
    plural={}
}

\newglossaryentry{GameMode}{
    name={GameMode},
    description={Defines the rules of the game (APawn, APlayerController, APlayerState).},
    plural={}
}

\newglossaryentry{Blueprintable}{
    name={Blueprintable},
    description={Is allowed to be used as a base class for blueprints},
    plural={}
}

\newglossaryentry{BlueprintType}{
    name={BlueprintType},
    description={Is allowed to be specified as a variable type in blueprints},
    plural={}
}

\newglossaryentry{RHI}{
    name={\uline{R}endering \uline{H}ardware \uline{I}nterface},
    description={Is allowed to be specified as a variable type in blueprints},
    plural={}
}

%%%%%%%%%%%%%%%%%%%%%%%%%%%%%%%%%%%%%%%%%%%%%%%%%%%%%%%%%%%%%%%%%%%%%%
%   NIAGARA   %
%%%%%%%%%%%%%%%%%%%%%%%%%%%%%%%%%%%%%%%%%%%%%%%%%%%%%%%%%%%%%%%%%%%%%%
\newglossaryentry{Niagara System}{
    name={Niagara System},
    description={Are containers for multiple emitters combined in ONE effect.  },
    plural={}
}

\newglossaryentry{Niagara Emitters}{
    name={Niagara Emitters},
    description={Are containers for Modules. },
    plural={}
}

\newglossaryentry{Niagara Modules}{
    name={Niagara Modules},
    description={Are the base of Niagara VFX. The Modules are built using High-Level-Shading-Language.},
    plural={}
}

\newglossaryentry{Niagara Item}{
    name={Niagara Item},
    description={Is a part of a system or emitter that the user cannot create. Example: system properties, emitter properties and renderers},
    plural={}
}

\newglossaryentry{Niagara Data interface}{
    name={Niagara Data Interface},
    description={Allow access to arbitrary data like mesh data, audio, skeletal mesh data ...},
    plural={}
}

\newglossaryentry{Niagara Stage}{
    name={Niagara Stage},
    description={Is a way to organize the order of execution},
    plural={}
}

\newglossaryentry{Niagara Group}{
    name={Niagara Group},
    description={Is a way to group the data of modules and hence defines the NAMESPACE},
    plural={}
}

\newglossaryentry{Asset Registry}{
    name={Asset Registry},
    description={in memory stored list of unloaded assets. },
    plural={}
}

\newglossaryentry{Character}{
    name={Character},
    description={Pawns that have a mesh, collision, and built-in movement logic},
    plural={}
}


\newglossaryentry{Animation Sequence}{
    name={Animation Sequence},
    description={is an animation for a specific Skeleton},
    plural={}
}


\newglossaryentry{Blend Space}{
    name={Blend Space},
    description={enables you to play different Animation Sequences depending on a given variable (e.g. speed)},
    plural={}
}


\newglossaryentry{Animation Montage}{
    name={Animation Montage},
    description={enables you to add additional logic to Animation Sequences like sound, particle effects, notifcations ...},
    plural={}
}


\newglossaryentry{Component}{
    name={Component},
    description={provides a piece of functionality that can be added to an actor
    and thus do not exist by themselves. Are the only way to render meshes and images
    ,implement collision, play audio ... everything the player sees or interacts with
    in the world. In contrast to sub-objects, components
    instances wihthin an actor will be instanced},
    plural={}
}

\newglossaryentry{bSweep}{
    name={bSweep},
    description={Whether we sweep to the destination location, triggering overlaps along the way and stopping short of the target if blocked by something.},
    plural={}
}


\newglossaryentry{Subsystem}{
    name={Subsystem},
    description={are automatically instanced classes with managed lifetimes.},
    plural={}
}


\newglossaryentry{Blueprint Interface}{
    name={Blueprint Interface},
    description={where you declare functions but leave the implementation to the blueprints that use the interface. So it's just like a interface in programming},
    plural={}
}



\newglossaryentry{Game State}{
    name={Game State},
    description={},
    plural={}
}


\newglossaryentry{Event Dispatcher}{
    name={Event Dispatcher},
    description={Allows a blueprint class to report on its state to the level blueprint},
    plural={}
}


\newglossaryentry{BSP Brush}{
    name={BSP Brush},
    description={is an Actor used to block-out levels. },
    plural={}
}

\newglossaryentry{Primary Asset ID}{
    name={Primary Asset ID},
    description={How to refer to an asset the asset manager is managing.},
    plural={}
}

\newglossaryentry{Data Layer}{
    name={Data Layer},
    description={},
    plural={}
}

\newglossaryentry{Material Layer}{
    name={Material Layer},
    description={is an Asset to create a layer that can be masked or blended with other layers in the material},
    plural={}
}

\newglossaryentry{Material Layer Blend}{
    name={Material Layer Blend},
    description={is an asset to create a mask for blending two layers together},
    plural={}
}

\newglossaryentry{Material Expresiion}{
    name={Material Expresiion},
    description={},
    plural={}
}

\newglossaryentry{World}{
    name={World},
    description={The World is the top level object representing a level or a sandbox in which Actors and Components will exist and be rendered.},
    plural={}
}

\newglossaryentry{Navigation Invocer}{
    name={Navigation Invocer},
    description={},
    plural={}
}

\newglossaryentry{Anim Instance}{
    name={Anim Instance},
    description={Animation, -Sequence, -Montage or -Blueprint derives from the 'UAnimInstance'-Class. Used for character animaations.},
    plural={}
}

\newglossaryentry{Conduit}{
    name={Conduit},
    description={Is a transition rule that can do '1 to many' , 'many to 1' or 'many to many' instead of the simple '1 to 1'},
    plural={}
}

\newglossaryentry{mipmap}{
    name={mipmap},
    description={are pre-calculated, optimized sequences of images, each of which is a progressively lower resolution. MIP(multum in parvo = much in little)},
    plural={}
}

\newglossaryentry{Module}{
    name={Module},
    description={},
    plural={}
}

%%%%%%%%%%%%%%%%%%%%%%%%%%%%%%%%%%%%%%%%%%%%%%%
%%%%%%%%%%%%      GAS    %%%%%%%%%%%%%%%%%%%%%%
%%%%%%%%%%%%%%%%%%%%%%%%%%%%%%%%%%%%%%%%%%%%%%%

\newglossaryentry{GLS-Avatar}{
    name={GLS-Avatar},
    description={
        Is the physical representation of the ASC},
    plural={}
}

\newglossaryentry{GLS-Owner}{
    name={GLS-Owner},
    description={
        Actor that holds the ASC},
    plural={}
}

\newglossaryentry{ASC}{
    name={ASC},
    description={
        AbilitySystemComponent},
    plural={}
}

\newglossaryentry{Latent Task}{
    name={Latent Task},
    description={
        A task that is executed over multiple frames},
    plural={}
}
\maketitle
{
  \hypersetup{linkcolor=black}
  \tableofcontents
}

    \chapter{General}
        \section{Notes}
        \begin{itemize}
            \item CTRL+[{1,2, ... 9}] places a camera at the current viewport position to jump to them with {1,2, ... 9}
            \item TSubclassOf<[CLASSNAME]> [NAME];
            \item EV (Exposure-Value) can be set in: View Mode
            \begin{itemize}
                \item Home/Interior: 5-7
                \item Office: 7/8
                \item Outdoor: 15
            \end{itemize}
            \item \code{OnPossess} is executed before \code{OnBeginPlay}
            \item \code{GetOuterByClass} to find specific parent
            \item don't bind to delegates from constructor
            \item \code{inline} basically tells the compiler/linker that the function can be defined in multiple translation units, in the end it chooses one of them
            \item a looped montage will still stop if it’s Enable auto blend out is checked
        \end{itemize}

        \uline{template for a .GITIGNORE file}
        \begin{lstlisting}
    # Visual Studio 2015 user specific files
    .vs/
    
    # Compiled Object files
    *.slo
    *.lo
    *.o
    *.obj
    
    # Precompiled Headers
    *.gch
    *.pch
    
    # Compiled Dynamic libraries
    *.so
    *.dylib
    *.dll
    
    # Fortran module files
    *.mod
    
    # Compiled Static libraries
    *.lai
    *.la
    *.a
    *.lib
    
    # Executables
    *.exe
    *.out
    *.app
    *.ipa
    
    # These project files can be generated by the engine
    *.xcodeproj
    *.xcworkspace
    *.sln
    *.suo
    *.opensdf
    *.sdf
    *.VC.db
    *.VC.opendb
    
    # Precompiled Assets
    SourceArt/**/*.png
    SourceArt/**/*.tga
    
    # Builds
    Build/*
    
    # Whitelist PakBlacklist-<BuildConfiguration>.txt files
    !Build/*/
    Build/*/**
    !Build/*/PakBlacklist*.txt
    
    # Don't ignore icon files in Build
    !Build/**/*.ico
    
    
    # Configuration files generated by the Editor
    Saved/*
    
    # Compiled source files for the engine to use
    Intermediate/*
    
    #Binaries
    Binaries/*
    .idea/*
    Plugins/*
    
    # Cache files for the editor to use
    DerivedDataCache/*
    
    Platforms/*
        \end{lstlisting}

        \uline{Window-Resize}
        \includegraphics[width=\textwidth]{WindowResize.jpg}




        \section{Game Engine Concepts}
            \begin{itemize}
                \item \href{https://docs.unrealengine.com/5.1/en-US/actor-ticking-in-unreal-engine/https://docs.unrealengine.com/5.1/en-US/actor-ticking-in-unreal-engine/}{Official Docs}
                \item Tick-groups:
                \begin{itemize}
                    \item TG\_PrePhysics :
                            This is the tick group to use if your actor is intended to interact with physics objects, including physics-based attachments. This way, the actor's movement is complete and can be factored into physics simulation.
                            Physics simulation data during this tick will be one frame old - i.e. the data that was rendered to the screen last frame. 
                    \item TG\_DuringPhysics
                            Since this runs at the same time as physics simulation, it is unknown whether physics data during this tick is from the previous frame or the current frame. Physics simulation can finish at any time during this tick group and will not present any information to indicate this fact.
                            Because physics simulation data could be current or one frame out of date, this tick group is recommended only for logic that doesn't care about physics data or that can afford to be one frame off. Common cases might be updating inventory screens or minimap displays, where physics data is either entirely irrelevant, or displayed coarsely enough that the potential to have a single frame of lag does not matter.
                    \item TG\_PostPhysics
                            Results from this frame's physics simulation are complete by the time this tick group runs.
                            A good use of this group might be for weapon or movement traces, so that all physics objects are known to be in their final positions, as they will be when this frame is rendered. This is especially useful for things like laser sights in shooting games, where the laser beam must appear to come from the player's gun at its final position, and even a single frame of lag will be very noticeable. 
                    \item TG\_PostUpdateWork
                            This runs after TG\_PostPhysics. Historically, its primary function has been to feed last-possible-moment information into particle systems.
                \end{itemize}
            \end{itemize}

        \section{Config files}

            \begin{itemize}
                \item example config variable: \code{[Core.Log]LogTemp=warning}
                \item \code{SECTION}
                \begin{itemize}
                    \item alphabetic strings
                    \item sections for configurable objects contained in modules use the syntex \\
                    Modules: \code{[/Script/ModuleName.ClassName]} \\
                    Plugins: \code{[[/Script/PluginName.ClassName]]} \\
                    Blueprints: \code{[/PathToUAsset/UAssetName.UAssetName_C]} \\

                \end{itemize}
                \item \code{KEY}
                \begin{itemize}
                    \item belong to a \code{[SECTION]}
                    \item have a \code{KEY} and a \code{VALUE}
                    \item the \code{KEY} is always a string
                    \item the \code{VALUE} can be a string, number, boolean, array, struct or empty
                \end{itemize}
                \item \code{Comments}: are made using \code{;}
            \end{itemize}
            For more information about the configuration file hierarchy, see the header file \code{ConfigHierarchy.h} located in \code{Engine/Source/Runtime/Core/Public/Misc}

            GConfig is a global object that can be used to read and write configuration files. It is a member of the FConfigCacheIni class \\
            \begin{table}[!htb]
                \begin{tblr}{p{6cm}}
                    \hline
                        Access specifier \\
                    \hline
                        GetBool  \\
                        GetInt  \\
                        GetInt64  \\
                        GetFloat  \\
                        GetDouble  \\
                        GetArray  \\
                        GetString  \\
                        GetText  \\
                    \hline
                \end{tblr}
            \caption{ \code{GConfig->} functions }  
            \end{table}

            \begin{table}[!htb]
                \begin{tblr}{p{6cm}}
                    \hline
                        Category \\
                    \hline
                        GEditorIni \\
                        GEditorKeyBindingsIni \\
                        GEditorSettingsIni \\
                        GEditorIni \\
                        GEditorPerProjectIni \\
                        GCompatIni \\
                        GLightMassIni \\
                        GScalabilityIni \\
                        GHardwareIni \\
                        GInputIni \\
                        GGameIni \\
                        GGameUserSettingsIni \\
                        GRuntimeOptionsIni \\
                        GInstallBundleIni \\
                        GDeviceProfilesIni \\
                        GGameplayTagsIni \\
                        GNearClippingPlaneIni \\
                        GNearClippingPlaneIni\_RenderThread \\
                    \hline
                \end{tblr} 
            \caption{ Ini Categories }  
            \end{table}
            
            

Structure:
\begin{lstlisting}
[SECTION1]
<KEY1>=<VALUE1>
<KEY2>=<VALUE2>
    
[SECTION2]
<KEY3>=<VALUE3>
\end{lstlisting}
Example:
\begin{lstlisting}
// your actual class
    UCLASS(config=Game)
class AMyConfigActor : public UObject
{
    GENERATED_BODY()
    
    UPROPERTY(Config)
    int32 MyConfigVariable;
}
\end{lstlisting}
\begin{lstlisting}
// the config file (whatever category you chose)
[MyCategoryName]
MyConfigVariable=2
\end{lstlisting}

Loading value manually
\begin{lstlisting}
int MyConfigVariable;
GConfig->GetInt(TEXT("MyCategoryName"), TEXT("MyVariable"), MyConfigVariable, GGameIni);
\end{lstlisting}


        \section{Migrate}
            \begin{itemize}
                \item things that probably need to be changed in \code{DefaultEngine.ini}:
                \begin{itemize}
                    \item \code{ActiveGameNameRedirects} MUST be used $\rightarrow$ BPs will automatically fix error
                    \item 
                    \item \uline{optional}
                    \item \code{GameViewportClassName} if using CUI
                    \item \code{GameUserSettingsClassName} if using custom GameSettingsClass
                    \item \code{LocalPlayerClassName} if using custom LocalPlayerClass
                \end{itemize}
            \end{itemize}
        \begin{lstlisting}
    +ActiveGameNameRedirects=(OldGameName="/Script/TakingCharge",NewGameName="/Script/LearnTheBasics")

    GameViewportClientClassName=/Script/CommonUI.CommonGameViewportClient
    GameUserSettingsClassName=/Script/TakingCharge.UserSettings_TC
    LocalPlayerClassName=/Script/TakingCharge.LocalPlayer_TC
        \end{lstlisting}    
        
        \section{Unsorted}
        \begin{itemize}
            \item Is mixed in real time and precomputed
            \item scalability: adjusting the quality in runtime
            \item 
            \item Deferred rendering(unreal): 
            \item Forward rendering: 
            \item GBuffer(used by deferred renderer): stores the different images (reflection, roughness, normal ...) that will be mixed together to make the image that will be displayed on the screen
            \item Pixel Shader: most used shaders making per pixel calculations
            \item Vertex Shader: converts local location to world location; using Vertex-Color-Maps to offset object or tell what should be effected in simulations; only for visual effects
            \item materials are more expensive than triangles
            \item dynamic shadows: polycount
            \item Geometry rendering: Project settings->Rendering->
            \item Merge meshes that are used a lot (windows statistics actors count) and afterwards combine materials or far away actors (VR or mobile)
            \item Export selected to further tweak it
            \item Instanced Rendering must be setup at least 50 objects
            \item LOD steps .5 vertex count
            \item HLOD to group outdoor environments
            \item 
            \item Same static mesh with differnet distance culling (better effect while still mentaining performance Optimisation)
            \item 
        \end{itemize}
\bigskip        
        Drawcalls:
        \begin{itemize}
            \item 1000: Mobile
            \item 2000-3000: Good
            \item 5000: OK
            \item >5000: problematic
            \item >10000: most likely unplayable
        \end{itemize}
\bigskip


        \subsection{Redirectors}
            \begin{itemize}
                \item Changes go in an ini file under Config.
                \item Must be in a section called [CoreRedirects]
                \item Usually Config/DefaultEngine.ini
                \item You use the “fully qualified name” of the class ONLY in NewName and not OldName.
                \item 
                \item BP-related-errors might occur $\rightarrow$ resave the BPs
            \end{itemize}
\uline{Example (assuming you moved a class called MyCppClass from MyGameModule to MyNewModule):}
\begin{lstlisting}
[CoreRedirects]
; ... other redirects
+ClassRedirects=(OldName=\"MyCppClass\",NewName=\"/Script/MyNewModule.MyCppClass\")

    //For UObject derived classes you omit the actor A or object U prefix
    //So if we have AMyActor in MyGameModule and we move it to MyCoreModule then the line is

+ClassRedirects=(OldName=\"MyActor\",NewName=\"/Script/MyCoreModule.MyActor\")

    //For struct and enum types we include the FULL name of the class including any prefixes and they also have a different redirect key:
+StructRedirects=(OldName="FMyStruct",NewName="/Script/MyCoreModule.FMyStruct")
+EnumRedirects=(OldName="EMyEnum",NewName="/Script/MyCoreModule.EMyEnum")
\end{lstlisting}

        \section{Before Rendering}
            Performed in order
            \begin{itemize}
                \item Distance Culling: needs setup; removes any object that is further away then [NUMBER] (cull distance volume)
                \item Frustim Culling: always active; removes any object outside of the view
                \item Precomputed Visibility: needs setup(world settings); (precomputed visibility volume)(needs rebuild light)('show'->'visualize'->'precomputed visibility cells'); 
                \item Occlusion Culling: big things rnder most of the time
            \end{itemize}
            CONCLUSION: sublevels will increase performance because culling calculations don't have to be calculated

            
            \subsection{Rasterization}
                Is the conversion of 3D-Objects into pixels \\
                Seeing only 1 pixel of an object does not reduce the calculations to display on it $\rightarrow$ LOD \\
                \includegraphics[width=\textwidth]{Rasterizing.png} \\
                \includegraphics[width=\textwidth]{Overshading.png} \\
                Shading happens in 2x2 Pixels $\rightarrow$ 
                'Optimisation Viewmodes' $\rightarrow$ 'Quad Overdraw'  (transparent object will go above 4; important for forward rendering)\\
                Custom Depth \\


    \chapter{Engine Concepts}
        \begin{itemize}
            \item Everything built with the UnrealBuildTool
            \item Each target compiled from C++ modules
            \item modules use ohter modules by referencing them in their build rules (build.cs file)
            \item 
            \item the UBT will parse the file, look for macros like UCLASS or UPROPERTY and generate code which will be included (thus the include of [ModuleName].generated.h)
            \item Simple Function: can be declared 'FORCEINLINE' in the header to insert the actual code at the place it's called instead of actually calling it
        \end{itemize}

        \section{UObjects / UClass}
            \begin{itemize}
                \item UClass: 
                \begin{itemize}
                    \item is associated to a default instance of the associated UObject class called the Class Default Object (CDO)
                    \item CDO is allocated first and then constructed, only once, via the class constructor when the engine is initialised
                    \item The CDO acts as a template from which all other instances of the class are copied, and the constructor is never called again
                    \item $\rightarrow$ class constructors cannot contain any runtime logic
                \end{itemize} 
            \end{itemize}

        \section{Actors}
            \subsection{Reflections}
                \begin{itemize}
                    \item reflection capture actors: are static reflection actors (quality can be adjusted in project settings under 'reflection resolution')
                    \item screenspace reflections are post render reflections with a medium resource cost
                    \item planar reflections must be eneabled (support global clip plane), are very expensive but are full dynamic  
                \end{itemize}

            \section{Creation}
            \href{https://docs.unrealengine.com/5.0/en-US/unreal-engine-actor-lifecycle/}{offic. docs}
                \begin{table}[!htb]
                    \begin{tblr}{p{6cm} | p{12cm}}
                        \hline
                            Method & Description \\
                        \hline
                            UObject::PostLoad & for statically placed actors in a level; called in edtiro \& gameplay \\
                            UActorComponent::OnComponentCreated &
                            when actor is spawned; \\
                            AACtor::PreRegisterAllComponents &
                            statically \& spawned actors htat have native root components \\
                            UActorComponent::RegisterComponent &
                            creaets the physical representation of the components \\
                            AActorPostRegisterAllComponents & 
                            last functino that gets called for all actors \\
                            AActor::PostActorCreated &
                            called right before construction \\
                            AActor::PreInitializeComponents &
                            only during gameplay and certain editor preview windows \\
                            UActorComponent::InitializeComponents &
                            if \code{bWantsInizializeComponetSet} is true; only once per gameplay session \\
                            AActor::PostInitializeComponents &
                            after the actor's components ahve beeen initialized; during gameplay \\
                        \hline
                    \end{tblr}
                \end{table}
                \begin{figure}
                    \includegraphics[width=\textwidth, height=\textheight]{ActorLifeCycle.png}
                    \caption{Lifecycle}
                \end{figure}
                

        \section{Project Structure}
        \underline{In File-Explorer:}
        \begin{itemize}
            \item Binaries: compiled game libraries \& debug database
            \item config: default configureation file
            \item Content: contains the assets
            \item intermediate: temporary files (can be safely deleted)
            \item Saved: configuration files created at runtime (also log files)
            \item 
            \item $[$ ProjectName $]$.project: JSON-file containing the information of the project (to display in the launcher)
        \end{itemize}

        \section{Gameplay Framework}
             \begin{figure}[ht]
                \includegraphics[width=\textwidth]{GameFramework.jpg}
                \caption{This flowchart illustrates how these core gameplay classes relate to each other.
                A game is made up of a GameMode and GameState. Human players joining the game are associated
                with PlayerControllers. These PlayerControllers allow players to possess pawns in the game
                so they can have physical representations in the level. PlayerControllers also give players
                input controls, a heads-up display, or HUD, and a PlayerCameraManager for handling camera views. }
             \end{figure}

            \subsection{Pawn}
                \begin{itemize}
                    \item is an Actor that can be an agent wihthin the world
                    \item can be possessed by Controllers
                    \item accept input
                \end{itemize}
            \subsection{Character}
                \begin{itemize}
                    \item humanoid Pawn
                    \item contains
                    \begin{itemize}
                        \item CapsuleComponent
                        \item CharacterMovementComponent
                    \end{itemize}
                    \item can replicate movement smoothly across the network
                \end{itemize}
            \subsection{Controller}
                \begin{itemize}
                    \item Base-Actor-Class that is responsible for directing a Pawn
                \end{itemize}
            \subsection{PlayerController}
                \begin{itemize}
                    \item interface between the Pawn and human player controlling it
                    \item has
                    \begin{itemize}
                        \item \code{PlayerCameraManager}
                        \item \code{PlayerInput}
                    \end{itemize} 
                    \item 
                \end{itemize}
            \subsection{AIController}
                \begin{itemize}
                    \item 
                \end{itemize}
            \subsection{HUD}
                \begin{itemize}
                    \item is used to displayed information to the player 
                    \item 
                \end{itemize}
            \subsection{Camera}
                \begin{itemize}
                    \item 
                \end{itemize}
            \subsection{GameMode}
                \begin{itemize}
                    \item on server
                    \item always 1 present
                    \item contains information/rules like:
                    \begin{itemize}
                        \item number of present\&maximum player and spectators
                        \item how players enter the game e.g. spawn locations and spawn behavior
                        \item how pausing is handled
                        \item level transitions
                    \end{itemize}
                    \item AGameModeBase: Is the new simpler and more efficient class from which every other game mode is dervied
                    \item AGameMode: is a complex class with additional features and a child of AGameModeBase
                \end{itemize}
                
                \underline{functionality}
                \begin{table}[H]
                    \begin{tabular}{|p{6cm}|p{12cm}|}
                        \hline
                            Function/Event & Purpose \\
                        \hline
                            InitGame (E) & Is fired before any other script. Initializes parameters and spawns helper classes \\
                            PreLogin (F) & Is called before the Login and will not call Login if ErrorMessage is set \\
                            PostLogin (F) & Is called after Login and the first place to call replicated functions safely \\
                            Logout & Called when player leaves a game or is destroyed implementation via 'OnLogout'-Event \\
                            HandleStatingNewPlayer & Called after PostLogin or after seamless travel \\
                            RestartPlayer & called when spawning a player \\
                            RestartPlayerAtPlayerStart &  \\
                            RestartPlayerAtTransform &  \\
                            SpawnDefaultPawnAtTransform &  \\
                        \hline
                    \end{tabular}
                \end{table}

            \subsection{GameState}
                \begin{itemize}
                    \item on Server \& Client
                    \item Saved on Server REPLICATED to client
                    \item handles information about the game that everonye should know about (scores, quests, ) 
                    \item not specific to any player $\rightarrow$ game-wide-properties
                \end{itemize}
            \subsection{PlayerState}
                \begin{itemize}
                    \item on Server \& Server
                    \item contains Player-specific-data (name, score, in-match-level)
                \end{itemize}
        \section{Game Flow}
            \begin{figure}[ht]
                \includegraphics[width=\textwidth]{GameFlow.jpg}
                \caption{}
            \end{figure}

            \begin{itemize}
                \item init engine
                \item init \code{GameInstance}
                \item load level
                \item start playing
            \end{itemize}

    \chapter{Core engine classes}

        \section{UEngine}
            

        \section{Customize core engine classes}
            \begin{figure}
                \includegraphics[]{EngineCustomization.png}
                \caption{Customizable engine classes}
                \label{}
            \end{figure}
        
            \begin{itemize}
                \item go into \code{DefaultEngine.ini}
                \item under \code{[/Script/Engine.Engine]} set for example:
                \begin{itemize}
                    \item \code{GameViewportClientClassName=/Script/CommonUI.CommonGameViewportClient}
                    \item \code{GameUserSettingsClassName=/Script/TakingCharge.GameUserSettings_TC}
                    \item \code{LocalPlayerClassName=/Script/TakingCharge.LocalPlayer_TC}
                \end{itemize}
            \end{itemize}

    \chapter{Configuration files}
        \section{Config/DefaultEngine.ini}
            \begin{itemize}
                \item Contains:
                \begin{itemize}
                    \item Default starting map (Game+Editor)
                    \item Default GameMode
                \end{itemize}
            \end{itemize}

        \section{Config/DefaultGame.ini}
            \begin{itemize}
                \item Contains:
                \begin{itemize}
                    \item \code{PrimaryAssets} to scan
                \end{itemize}
            \end{itemize}

    \chapter{Input Processing (Gameplay)}
        UI has it's own input processing and comes before gameplay (\hyperlink{input:UI}{UI input processing})\\
        \section{Basics}
            \begin{itemize}
                \item \code{UInputComponent} := an \code{UActorComponent} enables actor to bind input events to delegate functions
                \item \code{PlayerController} := manages a stack of InputComponents
                \item \code{UPlayerInput} := processess input
                \item \code{IEnhancedInputSubsystemInterface} := useful functions for enhanced input
                \item \code{UEnhancedInputLocalPlayerSubsystem} := 
            \end{itemize}

        \section{PlayerInput}
        \begin{itemize}
            \item \code{PlayerInput} is inside the \code{PlayerController}
            \item processess player input
            \item only on client
            \item 
            \item \code{PlayerInput} contained structures are:
            \begin{itemize}
                \item \code{FInputActionKeyMapping}: binds a key press to an event-driven behavior
                \item \code{FInputAcisKeyMapping}: binds a key, so it is continuously polled
            \end{itemize}
        \end{itemize}
            \subsection{EnhancedPlayerInput}
                \begin{itemize}
                    \item derives from \code{PlayerInput}
                    \item 
                \end{itemize}
        
        \section{InputComponent}
            \begin{itemize}
                \item \code{InputComponent} is commonly inside \code{Pawn} or \code{Controller}
                \item links the \code{Mappings} to \code{Game-Actions}
                \item input will be handle in the following order:
                \begin{itemize}
                    \item 1. Actor with 'Accepts input' from most recently enabled
                    \item 2. Controllers
                    \item 3. Level Script
                    \item 4. Pawns
                \end{itemize}
                \item if one component takes the input, it is not available down the stack
            \end{itemize}

            \begin{figure}[ht]
                \includegraphics[width=\textwidth]{InputFlow.jpg}
                \caption{Order of Input processing}
            \end{figure}

        \chapter{Enhanced Input}

\section{Genneral Input Stack}
    \includegraphics[width=\textwidth]{Bilder/InputStack.jpg}
    \begin{itemize}
        \item ViewportWidget
        \item GameViewportClient
    \end{itemize}

    \begin{itemize}
        \item InputRouter := used to specify a specific order for input
        \item PlayerController::BuildInputStack := 
    \end{itemize}

    \subsection{InputEvent}
     \includegraphics[width=\textwidth]{Bilder/InputEvents.jpg}
      \includegraphics[width=\textwidth]{Bilder/NavigationSource.jpg}

\section{How to enable}
    \begin{itemize}
        \item Enable the 'Enhanced Input'-Plugin
        \item 'Project Settings' $\rightarrow$ 'Input' $\rightarrow$ 'Default Classes' $\rightarrow$ Change to 'EnhancedPlayerInput' \& 'EnhancedInputComponent'
        \item $\uparrow$ can also be done manually by adding following to \code{Config/DefaultInput.ini}
        \begin{lstlisting}
[/Script/Engine.InputSettings]
DefaultPlayerInputClass=/Script/EnhancedInput.EnhancedPlayerInput
DefaultInputComponentClass=/Script/EnhancedInput.EnhancedInputComponent                
        \end{lstlisting}
    \end{itemize}

\section{Main Parts}
    \begin{itemize}
        \item \code{UInputActions}:
        \begin{itemize}
            \item the main communication link between the Enhanced Input system and the project code
            \item can be anything that an interactive character might do like jump, open a door or a state like holding a button
            \item doesn't care where the raw input comes from but \code{has a state}
            \item eg.: pick up item action is either on or off but walking needs direction and speed
            \item property-types are:
            \begin{itemize}
                \item 
            \end{itemize}
        \end{itemize}
        \item \code{UInputMappingContext}
        \begin{itemize}
            \item creates the link between \code{Input} and \code{Action}
            \item 
            \item specify conditions to change the key behaviour for different situations
        \end{itemize}
        \item Modifiers (inside Actions and MappingContexts)
        \begin{itemize}
            \item can adjust the raw input coming from the users device
        \end{itemize}
        \item Triggers (inside Actions and MappingContexts)
        \begin{itemize}
            \item applied after the raw-input was modified by the Modifiers
            \item can use output of other Input-Actions
            \item eg.: One Input-Action must be active to trigger another
        \end{itemize}
    \end{itemize}

\section{Using it}
    \begin{itemize}
        \item 1. create \code{Input-Actions}
        \item 2. create \code{Input-Mapping-Context}
        \item 3. Assign \code{Inputs} to \code{Actions} inside the Input-Mapping-Context
        \item 4. add Input-Mapping-Context to \code{LocalPlayer}'s \code{Enhanced-Input-Local-Player-Subsystem}
    \end{itemize}

    \begin{lstlisting}
// There are ways to bind a UInputAction* to a handler function and multiple types of ETriggerEvent that may be of interest.

// This calls the handler function on the tick when MyInputAction starts, such as when pressing an action button.
if (MyInputAction)
{
PlayerEnhancedInputComponent->BindAction(MyInputAction, ETriggerEvent::Started, this, &AMyPawn::MyInputHandlerFunction);
}

// This calls the handler function (a UFUNCTION) by name on every tick while the input conditions are met, such as when holding a movement key down.
if (MyOtherInputAction)
{
PlayerEnhancedInputComponent->BindAction(MyOtherInputAction, ETriggerEvent::Triggered, this, TEXT("MyOtherInputHandlerFunction"));
}
    \end{lstlisting}

    \uline{Signatures for input handlers}
    \begin{itemize}
        \item \code{void ... ()}
        \item \code{void ... (const FInputActionValue& ActionValue)} := simplest that only tells that input was made
        \item \code{void ... (const FInputActinoInstance& Action Instance)} := provides access to the current value of the input
        \item \code{void ... (FInputActionValue ActionValue, float ElapsedTime, float TriggeredTime)} := when dynamically binding to a UFunction by its name
    \end{itemize}

    \begin{itemize}
        \item unreal creates a UPlayer for each player:
        \begin{itemize}
            \item ULocalPlayer := 
            \item UNetConnection := 
        \end{itemize}
        \item the connection between the UPlayer and the pawns is provided with the playercontroller
    \end{itemize}

    \uline{needed Headers inside PlayerController:}
    \begin{itemize}
        \item \code{#include "InputMappingContext.h"}
        \item \code{#include "InputAction.h"}
        \item \code{#include "InputModifiers.h"}
    \end{itemize}
        
    \uline{needed Headers inside Pawn/Character:}
    \begin{itemize}
        \item \code{#include "MyPlayerController.h"}
        \item \code{#include "EnhancedInputComponent.h"}
        \item \code{#include "EnhancedInputSubsystems.h"}
    \end{itemize}

    \uline{actual coding needed inside PlayerController}
    \begin{itemize}
        \item Ceate Mapping Context(s)
        \item Create Input Actions
        \item Assign ValueTypes to the Actions
        \item Optional: Add modifiers (eg.: invert, clamp, multiply)
        \item Now the Actions can be mapped to a Key of a Mapping Context
    \end{itemize}


    \subsection{Autorun}
        \begin{itemize}
            \item 
        \end{itemize}

        \begin{figure}[H]
            \includegraphics[width=\textwidth]{AutoRunEIS.png}
            \caption{Implementing AutoRun}
            \label{}
        \end{figure}


\section{Saving}

    \subsection{Overview}
        \begin{itemize}
            \item \code{EnhancedInputUserSettings} := is the class that holds the settings for the Enhanced Input system
            \item 
            \item PlayerMappableKeySettings :=
            \begin{itemize}
                \item Holds setting information of an Action Input or a Action Key Mapping for setting screen and save purposes
                \item can be set for Input Actions OR overriden in the IMC (Setting Behavior)
            \end{itemize} 
        \end{itemize}
    
        



    \begin{itemize}
        \item 
    \end{itemize}

\newpage



    \chapter{Save Game}
        \href{https://docs.unrealengine.com/4.27/en-US/InteractiveExperiences/SaveGame/}{Official DOCs}
        \begin{itemize}
            \item there is an existing SaveGame class
            \item global information and game session information can be saved in 2 seperate SaveGame classes
            \item Saving: the information is transfered from the current game world into a SaveGame object
            \item Loading: the information inside the SaveGame object is transfered to the world object like Characters, Player Controller ...
            \item 
        \end{itemize}
        \underline{Basic Steps:}
        \begin{itemize}
            \item create Blueprint Class with SaveGame as parent
            \item add variables for any information you need to save
            \item 
        \end{itemize}
        \underline{Used Nodes:}
        \begin{itemize}
            \item Create Save Game Object
            \item Async Save Game to Slot
        \end{itemize}

        \section{Using JSON}
            \begin{itemize}
                \item 
            \end{itemize}

        \section{Using c++}
            \begin{itemize}
                \item we need a \code{FArchive&} := holds also additional information:
                \begin{itemize}
                    \item \code{UsingCustomVersion}
                    \item \code{IsSaveGame}
                    \item \code{IsLoading}
                    \item \code{IsSaving}
                    \item ...
                \end{itemize}
                \item a \code{FObjectAndNameAsStringProxyArchive} is used to serialize pointers
                \item we set \code{FObjectAndNameAsStringProxyArchive::ArIsSaveGame} = \code{true} $\rightarrow$ only SaveGame uprops get saved
                \item 
                \item Advanced:
                \item we can use a Structured Archive that uses:
                \begin{itemize}
                    \item ArchiveFormatter (FBinaryArchiveFromatter, )
                    \item a \code{FStructuredArchive}
                \end{itemize}
                \item
                \item a structuredArchiveConsists of
                \begin{itemize}
                    \item Records: a container for named slots
                    \item Slots: storage location for values (literals or containers)
                    \item Containers:
                    \begin{itemize}
                        \item Array: contains fixed number of unnamed child slots
                        \item Stream: like an array, but unbound size
                        \item Map: contains a fixed number  of \textbf{named} child slots
                    \end{itemize}
                \end{itemize}
            \end{itemize}


\smallskip

    \chapter{LOD}
        \begin{itemize}
            \item LOD groups can be specified in a INI file \code{BaseEngine.ini} under \code{StaticMeshLODSettings}
            \item is a mesh based setting
            \item 
        \end{itemize}

        \underline{Colorcoding for LODs}
        \begin{itemize}
            \item 0 = White
            \item 1 = Red
            \item 2 = Green
            \item 3/Remaining = Blue
        \end{itemize}

        \underline{change LOD distance}
        \begin{itemize}
            \item screen size will trigger a LOD change
            \item ENABLE SCREEN SIZE: disable auto compute lod distance under 'LOD Settings'
            \item REMOVE LOD: inside of the LOD group the last entry is 'Remove LOD'
            \item 
        \end{itemize}
    
    \chapter{Level Design}
        \section{Basic Workflow}    
            \begin{itemize}
                \item Primitives
                \item replace primitives with blockouts, still simple lights
                \item Post Process, Lights, 
                \item Reflections, Effects, Volumes, Audio
            \end{itemize}

        \section{Levels}
            \subsection{Level Streaming (linear map loading)}
                \begin{itemize}
                    \item Level Streaming Volumes
                    \item Scripted Level Streaming
                    \item 
                \end{itemize}

                Load Level Instance:

            \underline{Useful commands}
            \begin{table}[!htb]
                \begin{tabular}{|l|l|}
                    \hline
                        Command & Description \\
                    \hline
                        stat levels & shows the level states \\
                    \hline
                \end{tabular}
            \caption{ caption }
            \end{table}

            

            \subsection{World Partition (open world map loading)}
                \begin{itemize}
                    \item replaces the old World Composition
                    \item World-Partition holds the Data-Layers
                    \item Data-Layers hold Actors
                    \item Actors can be assigned to any Layer (also multiple); is stored inside the Actor under \code{Layers}-category
                    
                \end{itemize}
                \chapter{Textures}
                \section{Texture Streaming}
                    \begin{itemize}
                        \item also called streamer
                        \item responsible for increasing and decreasing resolution of each texture
                    \end{itemize}
        \smallskip
                    Update Cycle
                    \begin{itemize}
                        \item update world view
                        \item choose optimal resolution for each texture
                        \item determine possible resolution depending on the pool size (RAM)
                        \item choose which textures to update
                        \item generate load/unload request
                    \end{itemize}
        \smallskip
                    Order in which the streamer will dropp the mips textures
                    \begin{itemize}
                        \item keep landscape texture, forced load textures and textures already missing resolutions
                        \item keep mips that are visible on screen
                        \item keep character textures and tesxtures that don't take much memory
                        \item drop mips that are not visible, dropping the last recently seen first
                    \end{itemize}
        \smallskip
                    Which textures should updated first
                    \begin{itemize}
                        \item Load visible mips
                        \item load forced load textures, landscape and character textures
                        \item load textures which are far from their target resolution first
                        \item load most recently seen first (not visible)
                    \end{itemize}
        \bigskip
                    After all that a batch of update requests is generated            
        
                
\chapter{Materials}
\begin{itemize}
    \item Component Mask node
    \item Saturate instead of Clamp
    \item 
\end{itemize}

    \section{Basics}
        \includegraphics[width=\textwidth]{MaterialMaster.png} \\
        \subsection{Math basics}
            \begin{itemize}
                \item Power := the black values grow (small values reach 0 much faster)
                \item Multiply := 
                \item VertexNormal := the direction the vertex is facing (gives a smooth look (Phong-shading))
                \item FaceNormal := the direction the face is facing (gives a facetted look (Flat-shading))
            \end{itemize}


    \section{Material Expressions}
        \begin{itemize}
            \item 
        \end{itemize}

    \section{Virtual Texturing}
        \begin{itemize}
            \item divides textures into tiles of fixed size (typically 128x128)
            \item analyse the pixels and only stream what's needed
            \item loads only visible parts of the texture
            \item closer $\rightarrow$ high-res
            \item example usage = landscape (you only see a small part of the whole landscape/texture at a time)
        \end{itemize}

        \subsection{Runtime-Virtual-Texturing}
            \begin{itemize}
                \item fills in with GPU instead of texture from disk
                \item render to cache and stream from it
                \item size of virtual-texture
            \end{itemize}
            \textbf{\underline{Worflow}}
            \begin{itemize}
                \item enable 'Virtual Texture Streaming' on the texture
                \item create rvt
                \item create rvt-volume
                \item material with 'Rutime-Virtual-Texture-Output' and 'Runtime-Virtual-Texture-Sample'
                \item 
                \item Far-Blend-Distance
            \end{itemize}

        \subsection{Runtime Virtual Texture}
            \uline{Workflow:}
            \begin{itemize}
                \item create Virtual Textures for the heightmap and colormap of the landscape
                \item depending of the usecase, change the 'Virtual Texture Content'
                \item create Runtime 'Virtual Texture Volume'
                \item drag the RVT into the 'Virtual Texture'-slot in the 'Virtual Texture Volume' details
                \item add the 'Runtime Virtual Texture Output'-Node to the landscape material so you can write the info to VTs
                \item add 2 VT slots on the landscape material and drag in the VTs
                \item add the setup to blend with the VT to the desired Materials
            \end{itemize}

            \begin{figure}
                \includegraphics[width=\textwidth]{RVT_Setup.jpg}
                \caption{Node Setup for the RVT inside the Landscape material}
            \end{figure}

            \begin{figure}
                \includegraphics[width=\textwidth]{VT-Material.jpg}
                \caption{Blend Setup}
            \end{figure}

    \section{Material Domain}
        Define the overall usage of a material and the attributes/nodes it has available (UI will also change preview in the material editor):
        \begin{itemize}
            \item Surface
            \item UI
            \item Post Process
        \end{itemize}

    \section{Transparency/Opcaity/Alpha Map}
        Differentiated in vertex-alpha and pixel-alpha. Where vertex alpha is more coarse. \\
        \underline{Methods:}
        \begin{itemize}
            \item Alpha-Blend: Has a wide range of transparency values $\rightarrow$ semi-transparency possible
            \begin{itemize}
                \item Overlapping alpha-blended meshes (paricles/grass) increase fill rate because same screen pixels are drawn again and again
                \item Sorting problems
            \end{itemize}
            \item Alpha-Test: totally-hard-transparency with a threshhold to make mid-values transparent or solid
            \begin{itemize}
                \item no sorting problem
                \item much faster than alpha-blend
                \item no AA $\rightarrow$ pixelly
            \end{itemize}
        \end{itemize}

    \section{Constant Material Instance/Dynamic Material Instance}
        \begin{itemize}
            \item Constant Material Instance: is a material-instance with non-changing parameters
            \item Dynamic Material Instance: is a material-instance with changing parameters. Achieved by BP-Node 'Create Dynamic Material Instance' or code
        \end{itemize}

    \section{World Displacement}
        The object bounds will not be changed. \\


    \section{Vertex Shader}
        \begin{itemize}
            \item in order to execute code on the Vertex-Shader use the \code{VertexInterpolator}-Node 
            \item 
        \end{itemize}

    \section{World Position Offset}


    \section{Render Targets}
        \begin{itemize}
            \item storing buffers for deferred renderer
            \item display complex effects (ripples on water)
        \end{itemize}
        \subsection{Create Render Targets}
            \begin{itemize}
                \item are a asset type
                \item $\curvearrowright$  $\rightarrow$ Render Target
                \item Blueprint:
                \begin{itemize}
                    \item ok
                \end{itemize}
            \end{itemize}

        
    \section{Tessellation}
        \subsection{Flat-Triangles}

        \subsection{PN-Triangles}

    \section{Parallax Occlusion Mapping}
        Is like a Bump-Map but a lot better \\
        Has a dedicated node in the material editor \\


    \section{Vertex-Color-Material}
        \begin{itemize}
            \item Add a 'Vertex Color'-Node to change something (Simplest is to multiply it with the 'Base-Color'-Values)
            \item Then in 'Paint-Mode' make sure Mode = Colors (Not Blend Weights)
            \item Select the channel you want to paint on !!!
        \end{itemize}

    
    \section{Material Layers}
        \uline{Solves what problem?:} 
        \begin{itemize}
            \item per-pixel control over where Materials are placed, then use layered Material functions or Material Layers.
            \item Simplify large and complex setups into custom nodes to make these operations reusable and graphs using them esier to read
        \end{itemize}
        \uline{When Shouldn't you use them?}
        \begin{itemize}
            \item 
            \item can be heavy in terms of performance, if the Materials used in layer functions are complex themselves
            \item all your layers are rendered simultaneously, and then blended $\rightarrow$
            \begin{itemize}
                \item you have four layers in a Material
                \item engine tests for each pixel of object to see which is blended
                \item then reject any not in use
            \end{itemize} 
            \item Layered Material is generally too heavy for this to be used on mobile platforms
            \item more Material Elements increase draw calls, but are generally much more efficient $\rightarrow$ multiple Materials instead of using a Layered Material
        \end{itemize}

        \begin{itemize}
            \item create a 'Material Layer asset' and add 'MakeMaterialAttributes' node == where you define parameters and where they go
            \item create 'Material Layer Blend Asset' == where the actual blending is defined
            \item Create 'Material'
            \item add 'Material Attribute Layers' Node 
            \item create 'Material Instance' of the Base Material and enable the 'Use Material Attributes' property
            \item use 'Layer Parameters'
        \end{itemize}

        \begin{itemize}
            \item Create a new Material Function and edit the node graph to perfection. This function will act as a layer when you call it in your base Material.
            \item Connect your node network to a new Make Material Attributes node, and connect it to the Function output.
            \item Save the Material Function.
            \item Repeat this process for any other Material Function layers you wish to create.
            \item Create a new Material and open it in the Material Editor.
            \item Drag your Material Functions from the Content Browser into the new Material to use as layers.
            \item Blend multiple Material Functions together using the Material Layer Blend functions. 
        \end{itemize}

        \begin{figure}
            \includegraphics[width=\textwidth]{MaterialLayerAssetGraph.png}
            \caption{Node Setup for a 'Material Layer'-Asset}
            %\label{fig:MaterialLayer}
        \end{figure}

        \begin{figure}
            \includegraphics[width=\textwidth]{MaterialLayerBlendAssetGraph.png}
            \caption{Basic 'Material Blend' setup}
            %\label{fig:MaterialBlend}
        \end{figure}


    \section{Post Process Material}
        \begin{itemize}
            \item are used to apply effects to the screen after the scene has been rendered
            \item NOT for effects like color correction or adjustments, bloom, depth of field, and various other effects, you should use the settings inherent to the Post Process Volume
            \item \code{Material Domain} : has to be set to \code{Post Process}
            \item \code{Blendable Location}:
            \begin{itemize}
                \item \code{Before Tonemapping}:  	
                        all lighting is provided in HDR with scene color when Scene Texture expression's Post Process Input 0
                        is used. It fixes issues with temporal anti-aliasing (TAA) and GBuffer lookups. For example,
                        issues that can happen when using depth and normals.
                \item \code{After Tonemapping}:
                    This option indicates that post processing will take place after tonemapping
                    and color grading has been completed. It is the preferred location \textbf{for performance}
                    since the color is LDR and requires less precision and bandwidth. When this option is selected,
                    the SceneTexture expression's Post Process Inputs 2 and 3 are used to control where
                    Scene Color is in the pipeline.
                    Input 2 applies scene color before tonemapping.
                    Input 3 applies scene color after tonemapping.
                \item \code{Before Translucency}: This is even earlier in the pipeline than 'Before Tonemapping' before translucency was combined with the scene color. Note that SeparateTranslucency is composited later than normal translucency.
                \item \code{Replacing the Tonemapper}: PostProcessInput0 provides the HDR scene color, PostProcessInput1 has the SeparateTranslucency (Alpha is mask), PostprocessInput2 has the low resolution bloom input.
            \end{itemize}
            \item \code{Custom Depth} := adds extra draw calls; can be actiuvated on the mesh
        \end{itemize}

        \subsection{Useful nodes}
            \begin{itemize}
                \item \code{SceneTexture}: to access the scene color, depth, normals, etc.
                \item \code{SceneColor}: to access the scene color
                \item \code{SceneDepth}: to access the scene depth
                \item \code{ViewSize}: gives the size of the viewport
            \end{itemize}


        \subsection{Creating a outline}
            \begin{itemize}
                \item create \code{PostProcess Volume}
                \item set \code{Extends Infinite}
                \item 
                \item \code{Actors} need \code{Render Custom Depth Pass} to be enabled
                \item 
                \item Create \code{Material} for the postprocess
                \item use \code{Scene-Texture}-Node
                \item set \code{Scene-Texture-Id} to \code{CustomDepth}
                \item Divide the CustomDepth by a big number
                \item clamp it between 0-1
                \item 
            \end{itemize}


    \section{Tips Tricks}
        \begin{itemize}
            \item you can change parameters using \code{Use Custom Primitive Data} in the mesh
            \item you can name channels on a vector parameter
            \item 
        \end{itemize}

    \section{Decal}
        Project Settings $\rightarrow$ Engine $\rightarrow$ Rendering $\rightarrow$ DBuffer Decals \\
        CameraDirectionVector * Offset into 'World Position Offset' \\
        Blend Modes \\
        Receive Decals Decal \\
        Sort Order \\
        Animated Decal Material \\

    \section{Landscape Material}
        \subsection{Basic Setup}
            \begin{itemize}
                \item create material
                \item create material
            \end{itemize}
        %\label{material_landscape}
        \subsection{Landscape specific nodes}
            \begin{itemize}
                \item LandscapeLayerBlend: enables to blend together multiple textures or material networks to be used as landscape layers.
                \item LandscapeGrassOutput
                \item LandscapeLayerSample
                \item LandscapeLayerSwitch: to exclude material operations if a layer is not present (weight=0)
                \item LandscapeLayerWeight: allows material network blending, based on the weight on the landscape material (layer weight as blending alpha)
                \item LandscapeVisibilityMask: to add holes in a landscape (connected to opacity mask material input)
            \end{itemize}

        \subsection{Blend Modes (LandscapeLayerBlend)}
            \underline{LB Weight Blend:}
            \begin{itemize}
                \item if layer coming from an external program (ex. world machine) 
                \item painting layers independently from another
                \item without layer order
            \end{itemize}
            \underline{LB Alpha Blend:}
            \begin{itemize}
                \item painting in detail
                \item defined layer order
                \item ex. snow over rock and grass
            \end{itemize}
            \underline{LB Height Blend:}
            \begin{itemize}
                \item like LB Weight Blend
                \item adds detail to the transition between layers based on height map
                \item ex. dirt in the gaps of rocks
            \end{itemize}

        \subsection{Grass}
        \subsection{Auto Landscape}
            \begin{itemize}
                \item Select 3 materials from the 'Auto\_Landscape' 'Materials'-Folder
                \item migrate them into the desired project
                \item create a material instance of the 'M\_Auto\_Landscape'
                \item add textures to the materials 'Material\_A' (slopes) 'Material\_B' (Grass) ...
                \item adjust Material properties like tiling, Cell Bombing, Specular, Triplanar Projection (Slopes), Use Normal Map Blend, Z\_Slope\_Bias
                \item Add Foliage:
                \begin{itemize}
                    \item Use imported 'Landscape\_Grass\_Type'-Asset or create a new one
                \end{itemize}
            \end{itemize}
             \includegraphics[width=\textwidth]{Auto_Landscape_1.jpg}

    \section{Physics Material}
        \begin{itemize}
            \item describes the properties of a material
            \item is set on in the details of the material output under 'Physical Material' 
        \end{itemize}

    \section{Example setups}
        \subsection{Moving Distortion (Water effect)}
            The basic concept is that the uv-coordinates move around
            in order to make it look like the surface is moving. \\
            \includegraphics[width=\textwidth]{MovingDistortionShader.png} \\


        \subsection{Swaying Grass}
            \begin{itemize}
                \item Use 'SimpleGrassWind' node
                \begin{itemize}
                    \item WindWeight: grayscale map to specifying influence weight
                    \item AdditionalWPO: takes in additional world position offset network or function
                \end{itemize} 
            \end{itemize}


        \subsection{Glass}
            \uline{High-Quality}
            \begin{itemize}
                \item \code{Blend Mode} = \code{Translucent}
                \item 
                \item 
            \end{itemize}

            \uline{Low-Quality}
            \begin{itemize}
                \item \code{Blend Mode} = \code{Additive} or \code{Modulate}
            \end{itemize}
    \section{Hotkeys}
        \begin{itemize}
            \item T: TextureSample
            \item U: TextureCoordinates
            \item 1-4: Vector
            \item A: Add
            \item M: Multiply
            \item D: Divide
            \item S: Skalar Parameter
            \item CTRL: click on output to drag all connectors
            \item O: 1-x
        \end{itemize}

    \section{Node Reference}
        \begin{table}[H]
            \begin{tabular}{|c|c|}
                \hline
                    Node & Description \\
                \hline
                    Mask & Seperate values (for example rgb or a V3) \\
                    CheapContrast & negative: moves to white, positive: moves to black; works only on one channel \\
                    ChecpContrastRGB & works on all channels \\
                    ComponentMask & specify which channels you want to use \\
                    Clamp & Min-Max \\
                    LinearInterpolate & Output value between A and B depending on the Alpha \\
                    If & ... \\
                \hline
            \end{tabular}
        \end{table}


    \section{Optimization for older hardware and mobile}
        \begin{itemize}
            \item In order to adjust a material to older hardware use the 'FEATURE LEVEL SWITCH'-node
            \item there you can input different calculations for different feature levels
        \end{itemize}


    \section{Substance Designer}
    \begin{itemize}
        \item Create Material in substance designer
        \item expose parameters you want to be editable
        \item export sbsar
        \item import into unreal 
    \end{itemize}


    \section{Curve Atlas (LUT)}
        \begin{itemize}
            \item used to put different gradients into one texture
            \item no runtime support
            \item should use size with power of 2
        \end{itemize}


    \section{Substrate}
        \begin{itemize}
            \item has to be enabled under \code{Plugins}
            \item is a new way to render materials
            \item makes working with translucency more straightforward 
        \end{itemize}

        \subsection{Basics}
            \begin{itemize}
                \item has a root node like the legacy materials where you can set material domain, blend mode, shading model, etc.
                \item a \code{Substrate Shading Models}-Node (Substrate-Slab-BSDF) is the connected to the \code{Front Material}-Input on the root-node
                \item  
                \item a \code{Slab} is the basic building block of a substrate material
                \item it consists of 2 parts:
                \begin{itemize}
                    \item \code{Interface}: is the boundry that interacts with light and is defined by defined by the \code{Roughness}, \code{Normal}, \code{Diffuse} \code{Albedo}, \code{F0} and \code{F90} 
                    \item \code{Medium}: is beneath the interface where light is absorbed and scattered and is defined by Mean-Free-Path
                \end{itemize}
                \item Node-Types:
                \begin{itemize}
                    \item \code{BSDFs}: These nodes represent most types of surfaces, from simple materials to more complex ones like hair, eyes, and water.
                    \item \code{Operators}: These nodes mix and layer multiple Substrate Slab BSDFs to create complex and varied surfaces.
                    \item \code{Building-Blocks}: These nodes translate common material types for use with Substrate, like creating a coated layer or the default legacy material shading model of Unreal Engine.
                    \item \code{Extras}: These nodes define a Material Domain for a Substrate Material, and are directly analogous to their legacy Material Domain namesakes.
                    \item \code{Helpers}: These nodes are used to do some conversion within the material, such as mapping transmittance to Mean Free Path for a Substrate Slab.
                \end{itemize}
            \end{itemize}
            \begin{figure}
                \includegraphics[width=\textwidth]{substrate-slab-composition.png}
                \caption{1 = Interface, 2 = Medium}
                \label{}
            \end{figure}
            
            \subsubsection{BSDFs}
                \begin{figure}[!htb]
                    \includegraphics[width=\textwidth]{substrate-bsdf-nodes.png}
                    \caption{Available bsdf nodes}
                    \label{}
                \end{figure}
                \begin{table}[!htb]
                    \begin{tblr}{p{4cm} | p{12cm}}
                        \hline
                            Substrate-Slab-Input& Description \\
                        \hline
                            Diffuse Albedo      & \makecell[l]{ percentage of light reflected as diffuse from a surface. \\
                                                                similar to the local base color of the medium \\
                                                                default value = 0.18.} \\
                            F0                  & \makecell[l]{ color and brightness of the specular highlight where the surface is \\
                                                                perpendicular to the camera. dielectric Materials (non-metals) = 0 - 0.08 \\
                                                                metallic Materials up to 1. Gemstones are in a range up to around 0.16.} \\
                            F90                 & \makecell[l]{ color of the specular highlight where the surface normal is 90 degrees from the camera \\
                                                                Only hue and saturation are perceived, as brightness is fixed at 1.0 \\
                                                                This fades to black as F0 drops below 0.02.} \\
                            Roughness           & \makecell[l]{ Defines the roughness of the surface. The default value is 0.5.} \\
                            Anisotropy          & \makecell[l]{ Controls the anisotropy direction of the Material \\
                                                                (-1: highlight aligned to the bi-tangent, 1: highlight is aligned with the tangent).} \\
                            Normal              & \makecell[l]{ Defines the normal map of the surface.} \\
                            Tangent             & \makecell[l]{ Take a surface tangent as input. The normal is considered tangent or world space \\
                                                                according to the space properties on the material root node. \\
                                                                This input defines the shading tangent per-pixel.} \\
                            SSS MFP             & \makecell[l]{ Only used when there is no SS-Profile-Asset assigned to the root node \\
                                                                Defines the distance light travels through the medium before being absorbed or scattered. \\
                                                                The default value is 1.} \\
                            SSS MFP Scale       & \makecell[l]{ density of the material $\rightarrow$ influences the absorption and scattering of light by the Material. \\
                                                                defines the average distance at which a photon interacts with a particle of matter \\
                                                                controlled per color channel. } \\
                            SSS Phase Anisotropy& \makecell[l]{ Positive values elongate the phase function along the light direction $\rightarrow$ forward scattering.
                                                                Negative values elongate the function backward along the light direction $\rightarrow$ back scattering.} \\
                            Emissive Color      & \makecell[l]{ Defines the color of the light emitted by the Material.} \\
                            Second Roughness    & \makecell[l]{ Controls the roughness of a secondary specular lobe \\
                                                                does not influence diffuse roughness } \\
                            Second Roughness Weight & \makecell[l]{ The mix factor between the primary and secondary specular lobe \\
                                                                    0 renders the primary lobe only 1.0 renders the secondary lobe only.} \\
                            Fuzz Roughness      & \makecell[l]{ Controls the roughness of the fuzz layer \\
                                                                default value = Roughness-Input} \\
                            Fuzz Amount         & \makecell[l]{ Adds a fuzz-like layer at the interface, causing color retroreflectivity \\
                                                                amount of fuzz applied on top of a surface layer. Usually used to create fabric Materials.} \\
                            Fuzz Color          & \makecell[l]{ Defines the color of the fuzz layer.} \\
                            Glint Density       & \makecell[l]{ The logarithm representation of micro facet density on the surface of a material \\
                                                                Requires r.Substrate.Glints=1 to be set in the ConsoleVariables.ini configuration file } \\
                            Glint UVs           & \makecell[l]{ The UVs used to sample the glint texture \\
                                                                Requires r.Substrate.Glints=1 to be set in the ConsoleVariables.ini configuration file } \\
                        \hline
                    \end{tblr}
                \caption{ *MFP = Mean-Free-Path}  
                \end{table}
            
            \subsubsection{Operators}
                \begin{figure}
                    \includegraphics[width=\textwidth]{substrate-operator-nodes.png}
                    \caption{only for Substrate-Slab-BSDF and Substrate-Simple-Clear-Coat}
                    \label{}
                \end{figure}
                
                \begin{table}[!htb]
                    \begin{tblr}{p{6cm} | p{12cm}}
                        \hline
                            Operator & Description \\
                        \hline
                            Substrate Coverage Weight   & \makecell[l]{ input from a Slab and controls the amount of coverage \\
                                                                        Reducing the weight reduces the coverage of matter of the slab, \\
                                                                        meaning you will see through to the matter underneath \\
                                                                        should be used in conjunction with the Substrate Vertical Layer operator \\
                                                                        to have opaque matter on top of another, like dust and dirt layers\\
                                                                        where you want to control how much they cover the surface below.} \\
                            Substrate Vertical Layer    & \makecell[l]{ input from two Slabs: a Top and Bottom layer. \\
                                                                        The bottom Slab is coated by the top Slab with the bottom layer's appearance \\
                                                                        influenced by the properties of the top layer. \\
                                                                        Use the Top Thickness input to control how thick the top layer is over the bottom. \\
                                                                        ideal for creating car paints, wood varnishes, and wetness on a surface.} \\
                            Substrate Horizontal Blend  & \makecell[l]{input from two Slabs: a Background and Foreground. The Mix input controls how much these two Slabs mix together using a linear interpolation.} \\
                            Substrate Add               & \makecell[l]{input from two Slabs and adds them together. The material created is not physically plausible because it creates more outgoing energy from the surface than incoming energy} \\
                        \hline
                    \end{tblr}
                \caption{ caption }  
                \end{table}

                \begin{itemize}
                    \item \code{Use Parameter Blending}:    Operator nodes include an option to blend their background and foreground into
                                                            a single material when toggling on Use Parameter Blending. Because Substrate Operators
                                                            can create complex material appearances by mixing and layering slabs together, their expense
                                                            at runtime (primarily due to ligthing evaluation) can be costly to performance.
                                                            Parameter blending is an optimization that trades expensive lighting evaluation
                                                            for runtime performance and less costly lighting evaluation.
                \end{itemize}
                
                
            
    \section{HLSL}

        \begin{itemize}
            \item High-Level Shader Language := C-like language
            \item 
        \end{itemize}
\chapter{Niagara}
    \section{VFX-Theory}
        \subsection{Effect Types}
            \begin{itemize}
                \item Paricle System: using only particles 
                \item Mesh: uses meshes as particles for the spawning system
                \item Flipbook/SpriteSheets/Atlases: multiple images in one texture (so you can iterate over them)
                \item Shader: ...
                \item Hybrids: use multiple of the above
            \end{itemize}
        \subsection{VFX Principles}
            \underline{How to convey information}
            \begin{itemize}
                \item Gameplay: effect purpose
                \item Timing:
                \begin{itemize}
                    \item Anticipation: build-up (smooth transition) ex. high damage longer build-up
                    \item Climax:
                    \item Dissipation/Fade-Out:
                \end{itemize}
                \item Shape:
                \item Contrast: to create a focal point (ex. brightest most important)
                \item Color:
            \end{itemize}
            \includegraphics[width=\textwidth]{VFX_Principles.png}
    \section{Niagara Basics}
            \uline{Niagara Components:}
            \begin{itemize}
                \item Niagara-System: contains multiple Niagara-Emitters to create a complex effect
                \item Niagara-Emitter: is an effect that spawns particles
                \item Niagara-Node: one part of an emitter (displayed in the graph)
                \item Niagara-Module: 
                \item Niagara-Module-Script: are the parts that make up the module (functions)
            \end{itemize}
            \uline{Niagara-Modules:}
            \begin{itemize}
                \item add some logic
                \item 
                \item you can create Custom-Modules
            \end{itemize}
            \uline{How to Create a Niagara Emitter:}
            \begin{enumerate}
                \item Create a Niagara System
                \item Add Emitters to the System
                \item Add Modules - that specify the behaviour of the emitter - to the Emitter
                \item Bounds: can be calculated automatically by going to the 'Bounds' dropdown and selecting 'Set fixed bounds' (Emitters)
            \end{enumerate}

        \subsection{GPU vs CPU}
                \begin{itemize}
                    \item GPU can't calculate bounds (Niagara-System might get culled even though it's visible but bounds might be too small)
                    \item $\hookrightarrow$ resize the bounds := above viewport \code{Bounds} $\rightarrow$ \code{Dropdown-menu} $\rightarrow$ \code{Set fixed bounds}
                \end{itemize}
            
            Modules $\rightarrow$ Emitter $\rightarrow$ System \\
            There is a SystemEmitter from which other Emitters can inherit values \\
            'Effect Types' stores advanced settings \\
            'Max time without render' tells how long to look away until the emitter should be removed \\

        \subsection{User Parameters}
            \underline{Parameter-Space in Emitters}
            \begin{itemize}
                \item Emitter: exists in the 'Emitter' namespace
                \item Transient: exists in 'Niagara System' namespace
            \end{itemize}
\bigskip
\bigskip
            \underline{Parameter-Space in Niagara-Systems}
            \begin{itemize}
                \item User: initialized per system;  
                \item 
            \end{itemize}


            In order to use the Value you have to add a 'Set Parameter' Module in the 'Emitter Spawn'



            \begin{itemize}
                \item Every value can be edited through the 'Details Panel' if you 'Make' it a 'Read from new User Parameter'
                \item the DEFAULT value can be set through the 'Niagra Overview Node' of the emitter system
            \end{itemize}

            \underline{Inside of an Emitter}
            \begin{itemize}
                \item Every value can be edited through the 'Details Panel' if you 'Make' it a 
            \end{itemize}
            

\smallskip
            \underline{Parameters and Parameter Types:}
            \begin{itemize}
                \item Primitive: numeric data with varying precision and channel widths
                \item Enum: define fixed named values
                \item Struct: combine Primitives and Enums
                \item Data Interface: defines functinos that provide data from external sources like other UE4 parts or outside applications
            \end{itemize}
\smallskip
            \begin{itemize}
                \item Niagara operates as a stack with modules, executed from top to bottom
                \item Every module is assign to a group to determine when to be executed
                \item 'System Spawn' 'System Update' are executed first handling the shared behaviour between modules ( TODO: add examples of shared behaviour)
                \item Then modules and items from the emitter groups 'Emitter Spawn' and 'Emitter Update'
                \item Parameters in the particle groups execute for each unique particle
                \item finally renderer group items which describe how to render each particle of a emitter
            \end{itemize}                      
\smallskip
            \includegraphics[width=\textwidth]{NiagaraGroups.png} \\
            \begin{itemize}
                \item Emitter Settings
                \item Emitter Spawn: modules that effect emitter on spawn
                \item Emitter Update: modules that effect emitter over time
                \item Particle Spawn: modules that effect the particle spawn
                \item Particle Update: modules that effect the particle over time
                \item Event Handlers: events that define certain data for one or more emitters to trigger a listener and his behaviour
            \end{itemize}

        \section{Data Interface}
            \begin{itemize}
                \item 
            \end{itemize}

        \section{Events}
            \begin{itemize}
                \item only on CPU
                \item needs \code{Requires Persistant IDs}
                \item Can be placed in any module to share data
            \end{itemize}
\bigskip
            \underline{List of Events:}
            \begin{itemize}
                \item Location Events
                \item Death Events
                \item Collision Events
            \end{itemize}

            \subsection{Event Handlers}
                \includegraphics[width=\textwidth]{EventHandlerProperties.png}
                \underline{Event Handler 'Properties':}
                \begin{itemize}
                    \item defines the basics of the Handler
                    \item Source: with a dropdown of all Gernerated Event modules
                    \item Execution Mode: 
                \end{itemize}
\bigskip
                \underline{Event Handler 'Receive Event':}
                \begin{itemize}
                    \item 
                \end{itemize}

            \subsection{Collision example}
                \begin{itemize}
                    \item use \code{Generate Collision event} on one emitter
                    \item add \code{Event Handler} with \code{Source: Collision Event} on another
                    \item $\hookrightarrow$ add \code{Receive: Collision Event} to get actual collision-data
                \end{itemize}

            \subsection{Useful Nodes}
                \begin{itemize}
                    \item Add Velocity
                    \item Add Acceleration
                    \item Drag
                    \item PRESS SPACE BAR TO RESET EMITTER
                \end{itemize}
                
            \subsection{Side Notes}
            \begin{itemize}
                \item Events are only supported by CPU simulations (19.5)
                \item Simulations can be performed once and the result can be shared with other modules
                \item Drag = Widerstand
                \item Masked material gives hard edges that's why translucent is better when doing fx
            \end{itemize}


    \section{Crowds}
        \begin{itemize}
            \item AI with Behavior-Trees is the standard approach
            \item 
            \item Create an empty 'Niagara System
            \item 
            \item add a new empty emitter to it
            \item 
            \item delete 'Initialize Particle'
            \item replace 'Sprite Renderer' with 'Mesh Renderer'
            \item 
            \item under 'Emitter Update' add 'Spawn Burst Instantaneous'
            \item change the spawn Count to 1
            \item 
            \item under 'Emitter State' change 'Life Cycle Mode' = 'Self' $\rightarrow$ 'Loop Behavior' = 'Once' + 'Loop Duration Mode' = 'Infinite'
            \item 
            \item under 'Particle Update' $\rightarrow$ 'Paricle State' $\rightarrow$ disbale 'Kill Particles When Liftime Has Elapsed'
            \item add the 'New Scratch Pad Module' (:= visual scripting for HLSL)
            \item add 'NFS Play Animation'
            \item 
        \end{itemize}


    \section{Some effects}
        \subsection{Ground Crack}
            \begin{itemize}
                \item Create 2048x2048 image in PS
                \item draw with grey circle
                \item add 'cracks' on the edge pointing outwards
                \item highlight with small bright brush
                \item lessen opacity outwards with erase tool
                \item 
                \item copy paste on second layer
                \item rotate and erase some parts
                \item 
                \item export as png
            \end{itemize}


        \subsection{Flipbook}
            \begin{itemize}
                \item create emitter with following modules and set the properties
                \item \code{Emitter-State} := \code{Loop-Behavior} = Once; \code{Loop-duration} = fixed;
                \item \code{Initialize-Particle} := \code{Lifetime} = shorter $\rightarrow$ plays Flipbook faster
                \item \code{Spawn Burst Instantaneous} := \code{Spawn} count = 1
                \item \code{Sprite Renderer}
                    \begin{itemize}
                        \item \code{Material} := your Flipbook
                        \item \code{Sub Image Size} := number of images in the Flipbook
                    \end{itemize}
                \item \code{SubUVAnimation}
                    \begin{itemize}
                        \item \code{Number of frames} := number of images in Flipbook
                    \end{itemize}
            \end{itemize}


    \section{Debugging}
        \begin{itemize}
            \item Niagara Debugger allows to debug the system by:
            \begin{itemize}
                \item showing playing emitter 
                \item number of particles
                \item showing the bounds
            \end{itemize}
        \end{itemize}
 

\chapter{World Creation}
    \section{Landscape Settings}
        \begin{itemize}
            \item \code{Components}: Are the large squares; base unit of rendering, visibility calculation and collision; verts of shared component edges are duplicated
            \item \code{Sections}: the smaller little squares inside the components can made out of of either 1 section or 4 sections (2x2); base unit for LOD calculations
            \item \code{Section Size}: is the number of verts for each section (8, 16, 32, 64, 128, 256)
            \item \code{Quads}:
        \end{itemize}

    \section{Landscape Foliage}
        \underline{Types}
        \begin{itemize}
            \item Painted foliage
            \item Landscape grass output (using material to paint on the layers)
            \item procedural foliage volume
        \end{itemize}
        
    \section{Notes}
        \begin{itemize}
            \item Procedural Foliage has no collision
        \end{itemize}
        \begin{itemize}
            \item Landscape
        \end{itemize}
        \underline{Importing heightmap}
        \begin{itemize}
            \item Z-Scale: 1 $\rightarrow$ (1x256) 512cm height difference; 100 $\rightarrow$ (2x256) 512m height difference
            \item White = high; black low
        \end{itemize}
        \begin{itemize}
            \item 
        \end{itemize}
        \underline{Color coding landscape creation}
        \begin{itemize}
            \item Yellow: Landscape Edge
            \item Light Green: Landscape Component Edge
            \item Medium Green: Landscape Section Edge
            \item Dark Green: Landscape Individual Quad
        \end{itemize}
        Each component = 1 draw call $\rightarrow$ Less Components $\rightarrow$ less draw calls \\
\bigskip
        \underline{CREATING THE LANDSCAPE}
        \begin{itemize}
            \item First create a landscape
            \item form it with a height-map in the creation process or
            \item sculp it
            %\item assign a landscape material \hyperref[material_landscape]{creating a landscape material}
        \end{itemize}



        FOLIAGE: DISABLE ALIGN TO NORMAL \\
    \section{Foliage}
        \underline{Basics}
        \begin{itemize}
            \item Overdraw costs more than tries $\rightarrow$ try to make the leafes fill out the whole area
            \item set dynamic shadow cascades to 1
        \end{itemize}

        \underline{Ways to place foliage}
        \begin{itemize}
            \item Foliage Mode Drag\&Drop static meshes you want to spawn
            \item Enable Procedural Spawner in Editor Settings $\rightarrow$ Procedural Foliage Spawner
            \item Landscape Material with Grass Output Type (assign meshes to array) 'Layer Sample Node' with layer name you want Grass-Type to spawn on
        \end{itemize}
        \begin{itemize}
            \item SHIFT+4
            \item Add Foliage
            \item STATIC MESH (instanced mesh in order to make only 1 draw call)
            \item Magnifying Glass in bottom right to edit Static Mesh
            \item Brush Mode
            \item Filter: where is it placable
        \end{itemize}


    \section{Grass on the landscape}
        \underline{Used Nodes}
        \begin{itemize}
            \item Landscape Layer Blend
            \item Landscape Sample (Must have the same Parameter as Layer you want to spawn the Grass on)
            \item Landscape Grass (is also used for rocks)
            \item Landscape Foliage $\rightarrow$ Landscape Grass Type (Culling settings)
        \end{itemize}
        We can restrict foliage spawning to specific Landscape Layers under 'Placement' $\rightarrow$ Advanced $\rightarrow$ 'Landscape Layers' \\
\bigskip            
        You can change the distance the foliage will stop spawning at with the
        console command(smaller number will spawn further):
\begin{lstlisting}
foliage.MinimumScreenSize 0.00001            
\end{lstlisting}

      
    
\chapter{User Interface (UI)}
    \section{General Notes(math)}
            \subsection{Functions}
                \subsubsection{Lerp}
                    \begin{itemize}
                        \item $Lerp(a, b, c) = result$
                        \item \code{a} and \code{b} := values you want to blend between
                        \item \code{c} :=  \colorbox{lightgray}{$c = 0 \Rightarrow result = a$} and \colorbox{lightgray}{$c = 1 \Rightarrow result = b$}
                        \item returns := mix between the 2 values (is a mask if c is  ${N}=\{0,1\}$)
                    \end{itemize}

                \subsubsection{Inverse Lerp}
                    \begin{itemize}
                        \item can be used to clamp some values
                        \item $InvLerp(a, b, c) = result$
                    \end{itemize}


                \subsubsection{Remap}
                    \begin{itemize}
                        \item is a combination of \code{Lerp} and \code{InvLerp}
                        \item $Remap(iMin, iMax, oMin, oMax, c) = result$
                        \item same as $InvLerp(iMin, iMax, value) = resulat1$ and then $Lerp(oMin, oMax, result1) = result2$
                    \end{itemize}


    \section{UI design}
        \begin{itemize}
            \item some general rules:
            \begin{itemize}
                \item rule of thirds := divide the screen into 3x3 parts and place important elements on the intersections
            \end{itemize}
        \end{itemize}
                


    \section{Resolution independent UI}
        \begin{itemize}
            \item start with resolution of 1920x1080 (<10\% of users have a resolution lower than that)
            \item All text should be legible.
            \item No UI elements should be overlapping.
            \item No text should be breaking out of its containers.
            \item Text should be tested with at least +30\% longer than the base English, to test localization.
            \item 
            \item easy resizing can be done with a \code{SizeBox} inside a \code{ScaleBox} 
        \end{itemize}



\uline{Add snapping to WrapBox inside of Scrollbox}
\begin{lstlisting}
InventoryScrollBox->OnUserScrolled.AddDynamic(this, &ThisClass::OnUserScrolled);

void URPGInventoryPanelWidget::OnUserScrolled(float CurrentOffset)
{
    //Waits StopWheelDelay seconds to fire the OnMouseWheelStop function.
    GetWorld()->GetTimerManager().ClearTimer(MouseWheelStopHandle);
    GetWorld()->GetTimerManager().SetTimer(MouseWheelStopHandle, [this, CurrentOffset]()
        {
            OnMouseWheelStop(CurrentOffset);
        }, StopWheelDelay, false);
}

void URPGInventoryPanelWidget::OnMouseWheelStop(float CurrentOffset)
{
    // Determine which item slot is at the top of the screen
    int32 TopItemIndex = FMath::RoundToInt(CurrentOffset / ItemSlotHeight);

    // Snap the scroll box to the top of the item slot at the top of the screen
    float NewScrollOffset = TopItemIndex * ItemSlotHeight;

    // Set the new scroll offset
    InventoryScrollBox->SetScrollOffset(NewScrollOffset);
}
\end{lstlisting}
\bigskip

\chapter{UMG}
    \subsection{General Notes}
        \begin{itemize}
            \item  settings for textures meant for UI := 
            \begin{itemize}
                \item \code{Level Of Detail} $\rightarrow$ Mip Gen Settings to \code{NoMipmaps}
                \item \code{LOD Group} =  \code{UI}
                \item \code{Compression Setting} = \code{TC Editor Icon}
            \end{itemize}
            \item performance tips:
            \begin{itemize}
                \item smaller widget tree $\rightarrow$ fewer function calls
                \item flatter widget tree $\rightarrow$ less recursion
                \item remove tick-function
                \item use \code{Collapsed} instead of \code{Hidden}
                \begin{itemize}
                    \item \code{Visible} := will be rendered and can be interacted with
                    \item \code{Hidden} := will not be rendered but can be interacted with (bindings are evaluated)
                    \item \code{Collapsed} := will not be rendered and can not be interacted with (bindings aren't evaluated)
                \end{itemize}
            \end{itemize}
        \end{itemize}

        \uline{General-Workflow:}
        \begin{itemize}
            \item There should be
            \begin{itemize}
                \item Main Widget: is used as a container for smaller parts
                \item User Created widgets: is a Category in the designer where you can add any other previously created Blueprint Widget
            \end{itemize}
        \end{itemize}
        \includegraphics[width=\textwidth]{DrawHud.png} \\

        \begin{figure}[!h]
        \begin{minipage}{\textwidth}
            \begingroup \parfillskip=0pt
                    \minipage{\textwidth}
                        \uline{ways to draw an image:}
                        \begin{itemize}
                            \item Box
                            \item Border
                            \item Image
                        \end{itemize}
                    \endminipage\hfill

                    \minipage{0.3\textwidth}
                        \includegraphics[width=\textwidth]{Bilder/9slice_box.png}
                        \caption{9-Slice: Box}
                    \endminipage\hfill
                    \minipage{0.3\textwidth}
                        \includegraphics[width=\textwidth]{Bilder/9slice_border.png}
                        \caption{9-Slice: Border}
                    \endminipage\hfill
                    \minipage{0.3\textwidth}%
                        \includegraphics[width=\textwidth]{Bilder/9slice_image.png}
                        \caption{9-Slice: Image}
                    \endminipage

            \par\endgroup
        \end{minipage}
        \end{figure}
        

\smallskip
\hypertarget{input:UI}{
KeyPressed
\begin{enumerate}
\item Let Preprocessors try to handle it
\item Run through the current focus list from root to outer most child to do Preview keys
\item Run through current focus list from outer most child to root to process KeyDown events
\item Pass input to the PlayerController. We're entirely done with UI now.
\item PlayerController goes through it's own input stack and delegates events out to gameplay classes.
\end{enumerate}
}
At any point, any one of those steps can consume input and stop the chain. This allows UI to not send input to game, a classic case is if you're typing in a  message box, you want to consume those controls so that they don't reach gameplay classes and move the character. \\
\\
CommonUI adds a new Preprocessor, and a new viewport client,
which allows it to catch some specialized inputs for gamepads for UI,
you can use these with any keys, but they're most useful for gamepads.
This is all UI handling and works same as any other UI handling. \\
\\

EnhancedInput specifically overrides the PlayerController's InputComponent class. Which allows it to use the input it normally gets in differing ways, like the hold or tap effects. This is after UI inputs. \\
\\
This is the reason why these two are entirely compatible. They have literally no overlap. \\


Locked just doesn't process the actual interaction
So the click for example
Lockable Widgets are also still visually pressable


    \subsection{Specific Widget notes}
        \subsubsection{RichText}
            \begin{itemize}
                \item Allows to format text
                \item DataTable with RichTextStyle data type:= allows to specify different styles
                \item $\hookrightarrow$ the name is used in the text like \code{<styleName>some text</>}
                \item 
                \item DataTable with RichImageRow data type := used to keep list of img that you want to use
                \item $\hookrightarrow$ the name is used in text like \code{<img id="ImageRowName"/>}
                \item 
                \item decorator classes allow you to add generic things like images, Slate ... inline the text
            \end{itemize}


        \subsubsection{Editable Text}
            \begin{itemize}
                \item has a property called \code{}
            \end{itemize}


        \subsubsection{ListView}
            \begin{itemize}
                \item create a widget containing \code{ListView}
                \item create a widget implmenting \code{UserObjectListEntry}-interface
                \item in the widget with the interface, implement the \code{OnListItemObjectSet} function
                \item 
                \item assign the widget with the interface to \code{Entry Widget Class} in the ListView
                \item use the \code{Set List Items} or \code{Add Item} functions to set/add the items (this will call the \code{OnListItemObjectSet}-function for each item)
            \end{itemize}

            
    \section{CommonUI}

        \begin{itemize}
            \item CommonInputActionDataBase: contains UIInputActions with their keys, names, description and icons
            \item CommonActivatableWidget: is a Widget-BP that can be activated and diactivated. so the top Widget will get focus
            \item CommonInputBaseControllerData: to display Icons for Widgets
            \item CommonUIInputData: universal Confirm/Back button setup
            \item 
            \item CommonActivableWidgetStack: gives automatic focusing and has only one active widget at a time
            \item removing widget from UI := will always deactive (even when destroyed); re-constructing just activates it (\code{auto-activate} is enabled)
            \item auto-activate := will activate the widget if it has been deactivated before
            \item 
            \item FInputMappingContextAndPriority % TODO: get more infos
        \end{itemize}

        Controller Data Info assets defining icon sets for your controllers, plugged into
        Project Settings > Game > Common Input Settings > Controller Data \\
        Gamepad must have the correct name ("Generic") \\

        \subsection{CommonUI with Enhanced Input}
            \begin{itemize}
                \item 
                \item \code{UCommonInputMetadata} :=
                \begin{itemize}
                    \item Inherit from this class for per platform info used by CommonUI
                    \item IMC's can be specified per platform, so each platform may have different Common Input Metadata
                \end{itemize}  
            
                \item \code{bool bIsGenericInputAction} := 
                \begin{itemize}
                    \item \code{Generic} actions (like accept or face button top) will be subscribed to by multiple UI elements. 
                    \item These actions will not broadcast enhanced input action delegates
                    \item 
                    \item \code{Non-generic} actions will fire Enhanced Input events 
                    \item will not fire CommonUI action bindings (Since those can be manually fired in BP).
                \end{itemize}
                \item 
            \end{itemize}


    \subsection{Overview of widgets}
            \begin{table}[!htb]
                \begin{tblr}{p{6cm} | p{12cm}}
                    \hline
                    Widget & Description \\
                    \hline
                    CommonWidgetStack & used when you want to push/pop widgets \\
                    Common Lazy Image & loads images asynchronously and hides loading with throbber \\
                    \hline
                \end{tblr}
                \caption{ caption }
            \end{table}

            extra info on Common Lazy Image: \\
            Streaming textures refers to textures that have mips, and therefore, allow for some or all of the mips to be loaded at any given time.  Depending upon texel density on screen, the renderer can attempt to 'stream' more mips off disk and into memory for the GPU.
            \\
            Virtual Textures are large textures on disk, that utilize GPU readbacks to determine if subsections of the large texture is visible, if so, they read in small chunks of texture into a Virtual texture atlas on the GPU - that enables very high details, without high memory overhead, because only the visible chunks are in memory - and also I think lowest mip level of the virtual texture also stays in memory.
            \\
            Non-Streaming textures, textures without mips (by which i mean they have only 1 - the highest), or those with mips that are marked as never stream, are always resident on the GPU when loaded.  They subtract from the total streaming pool because as mentioned they have no mips or are marked as never stream, so they can not be reduced from the pool until the UTexture is actually unloaded from memory.
            \\
            CommonLazyImage cares about none of this - it's one and only job is to hide the diskload hitch that would accompany a synchronus asset load for an image - by putting it behind an async load and a loading progress material


            
    \section{MVVM (Model-View-Model-View)}
        \subsection{workflow}
        \begin{itemize}
            \item enable MVVM plugin
            \item create blueprint \code{MVVMViewModelBase}
            \item add all the variables it needs
            \item mark them \code{FieldNotify}
            \item 
            \item construct the Viewmodel in some place \code{Construct ...}
            \item add it to the View-Model-Collection in the VMSubsystem \code{Add View Model Instance}
        \end{itemize}
            
            
            
    
    \section{Drag \& Drop}
        \begin{itemize}
            \item Create widget and override \code{OnMouseButtonDown}
            \item use node \code{Detect Drag if Pressed} use \code{Mouse Event} for the \code{Pointer Event}-input and LC as the \code{Drag Key}
            \item  
            \item override \code{OnDragDetected}
            \item use node \code{create Drag \& Drop operation}
            \item 
            \item another created widget overrides the \code{OnDrop}-function
            \item 
            \item the \code{On Drop Cancelled}-function will be fired if it's not droppepd on anything that accepts OnDrop
            \item 
            \item \textbf{extra information}: can be passed by creating a \code{Drag and drop operation} and adding variables to it
        \end{itemize}

            
    \section{Game Settings (Plugin)}            
        \begin{itemize}
            \item GameSettingsCollection := holds multiple settings that should be grouped together (video, audio ...)
            \item can be defined in .cpp files only
            \item 
            \item GameSettingsRegistry := holds the collections
        \end{itemize}

        \subsection{Setting up config files}
            \begin{itemize}
                \item change the GameSettingsClass and LocalPlayerClass using the \code{DefaultEngine.ini}
            \end{itemize}
            
\begin{lstlisting}
GameUserSettingsClassName=/Script/YOUR_PROJECT_NAME.YOUR_CUSTOM_CLASS
LocalPlayerClassName=/Script/YOUR_PROJECT_NAME.YOUR_CUSTOM_CLASS
\end{lstlisting}
        
        \subsection{Setting up a collection}
            \begin{itemize}
                \item create new collection := \code{UGameSettingCollection* Graphics = NewObject<UGameSettingCollection>();}
                \item set (internal)Name and DisplayName:
                \begin{itemize}
                    \item \code{Graphics->SetDevName(TEXT("GraphicsCollection"));}
                    \item \code{Graphics->SetDisplayName(LOCTEXT("GraphicsCollection_Name", "Graphics"));}
                \end{itemize}
            \end{itemize}
        
        
        \subsection{Basic setup and code execution}
            \uline{Widgets to create:}
            \begin{itemize}
                \item GameSettingsScreen
                \item GameSettingsPanel
                \item Widget derived from \code{UCommonTabListWidgetBase}
            \end{itemize}
        
            \begin{figure}
                \includegraphics[width=\textwidth]{GameSettingsSreenHierarchy.png}
                \caption{Widget Hierarchy of the GameSettingsScreen}
                \label{}
            \end{figure}

            \begin{itemize}
                \item First: \code{SettingScreen} $\rightarrow$ \code{RegisterSettingsTab}
                \item \code{RegisterTab} $\rightarrow$ will create a specified button 
            \end{itemize}
            
            \begin{figure}
                \includegraphics[width=\textwidth]{RegisterSettingsTab.png}
                \caption{Tab registration setup}
                \label{}
            \end{figure}
        
        SettingScreen::RegisterTab -> get The SettingCollection from the SettingsCollection::DevName (SettingSreen::)
        
        LyraTabListWidget(that is inside the SettingScreen)::RegisterDynamicTab(C++) !!! this does actually register the settingCollection
        everything before is just fluff
        
        TabListWidget::RegisterDynamicTab -> UCommonTabListWidgetBase::RegisterTab
        
        TabListWidget::HandleTabCreation := takes the tab name ID and a widget
        TabListWidget::HandleTabCreation->TabListWidget::UpdateTabStyles
        TabListWidget::UpdateTabStyles := sets widget as child of horizontalBox (HorizontalBox::AddChildToHorizontalBox), sets padding (HorizontalBoxSlot::SetPadding), sets minimum dimension (CommonButtonBase::SetMinDimensions)
        Refresh Next/Previous (actionButtons visibility)
        
        there is a Buttonstyle specified inside the TabListWidget (that holds the buttonTabs) that can be used to style all buttons on updateTabStyles
        
\chapter{Slate}

    \section{General Notes}
        \begin{enumerate}
            \item Slate uses the \glsdesc{CRTP}
            \item \code{.Pin} return a shared pointer from a weak pointer
            \item if you want have an SVerticalBox, say, and that inherits from SBox, it will returns SVerticalBox as its type name, so checking if it's typename is SBox will fail, even though it's a valid cast.
            If you want to use the Cast-like method, the widget itself must have SLATE\_DECLARE\_WIDGET or SLATE\_IMPLEMENT\_WIDGET in its header.
            \item void Construct(FArguments Args (or whatever it's called), Type1 Arg1, Type2 Arg2)
            SNew(YourType, Arg1, Arg2) 
            Personally I like to make mandatory parameters as construct arguments and anything optional as slate\_argument/attributes/etc. 
            
        \end{enumerate}

    \section{Types}
        \begin{itemize}
            \item \code{FLayout} : contains Areas and Layouts
            \item \code{FArea} : 
            \item \code{FSplitter} : 
        \end{itemize}

    
    \section{Custom Slate Widget}
        \begin{enumerate}
            \item Create a new class that inherits from \code{SCompoundWidget}
            \item Add the slate macros
            \item add the \code{Construct(const FArguments& InArgs)}-function
            \item add widgets to the \code{ChildSlot} member
            \item 
            \item access arguments: \code{InArgs._YourArgument}
        \end{enumerate}

        \begin{lstlisting}
            ChildSlot
        .HAlign( InArgs._HAlign )
        .VAlign( InArgs._VAlign )
        .Padding( InArgs._Padding )
        [
            InArgs._Content.Widget
        ];
        \end{lstlisting}


    \section{Useful code}
\begin{lstlisting}
SetGroup(FToolExampleEditor::Get().GetMenuRoot())
\end{lstlisting}

\begin{lstlisting}
FGlobalTabmanager::Get()->RegisterNomadTabSpawner(TabName, FOnSpawnTab::CreateRaw(this, &FExampleTabToolBase::SpawnTab));
\end{lstlisting}


    
\chapter{Lights}
    \section{Unsorted}
        \begin{itemize}
            \item Sky Light: emit no photon $\rightarrow$ 1 bounce
            \item Directional, Point and Spot Light emit photons $\rightarrow$ multiple bounces
            \item while building the level turn off 'Auto Exposure' $\rightarrow$ Project Settings $\rightarrow$ search Exposure $\rightarrow$ set to false
        \end{itemize}
    \section{Basics}
        \subsection{Mobility}
            \begin{itemize}
                \item \code{Static}: not  moving or updating in any way; built light; use indirect lighting (indirect ligting sample or volumetric lightmaps)
                \item \code{Stationary}: not moving; No built light (lightmass); uses Cached Shadow Map for Movable Light
                \item \code{Movable}: moved and updated; fully dynamic shadow; LIGHT ACTORS only cast dynamic shadows;
            \end{itemize}
        
        \subsection{IES Light}
            Is a file format describing the light distribution \\

        \subsection{Direct/Indirect}

    \section{Sky Atmosphere}
        \subsection{Setup}
            use extra tool for easy setup \code{Window -> Env. Lightmixer} or the following\\
            \begin{itemize}
                \item Add a directional light
                \item Place Directional Light and set Atmosphere/Fog Sun Light in its properties to true
                \item Add 'SkyAthmosphere'
                \item Add Sky Light
            \end{itemize}

            \underline{SkyAtmosphere Properties:}
            \begin{itemize}
                \item Planet:
                \begin{itemize}
                    \item Ground Radius: Size of the planet (if looked at from high altitude/space)
                \end{itemize}                
                \item \code{Rayleigh Exponential Distribution}: to define the altitude (in kilometers) at which Rayleigh scattering effect is reduced to 40\% due to reduced density
                \item \code{Mie Exponential Distribution}: to define the altitude (in kilometers) at which Mie scattering effect is reduced to 40\% due to reduced density
                \item \code{Atmospheric Height}: to define the height of the atmosphere above which we stop evaluating light interactions with the atmosphere.        
            \end{itemize}
            \includegraphics[width=\textwidth]{PlanetaryView.png}

            \underline{Additional notes}
                \begin{itemize}
                    \item Sun at zenith 120000 Lux
                    \item 150000 Lux on a white surface (measure with AO=off and the luminance meter)
                    \item Multiscattering in the Sky atmosphere = 1
                \end{itemize}

        \subsection{Directional Light}
            Has a 'Cascaded Shadow Map Property' which will display near shadows dynamically and use baked lighting for 
            lights at farther distance.
            \begin{table}[H]
                \begin{tabular}{|p{7cm}|p{12cm}|}
                    \hline
                        Property & Description \\
                    \hline
                        Dynamic Shadow Distance & Distance within which you will see cascading shadow maps \\
                        Num Dynamic Shadow Cascades & More levels $\rightarrow$ better shadow resulation at distance with greater performance cost \\
                        Cascade Distribution Exponent &  \\ % TODO:
                    \hline
                \end{tabular}
            \end{table}  

    \section{Lightmass Global Illumination}
        Lightmass creates Lightmaps for 'Stationary' and 'Static' lights.\\
        You have to surround the static light with the 'Lightmass Volume' \\
        don't forget to set ojects mobility static \\

    \section{Screen Space Global Illumination (SSGI)}
        Improves the shadows and directional light. Quality Level 3 for the best balance between performance and quality. \\
        \underline{In order to enable:} \\
        'Project Settings' $\rightarrow$ 'Engine' $\rightarrow$ 'Rendering' $\rightarrow$ under the Lighting category.  \\
        \begin{lstlisting}
    r.SSGI.Quality
        \end{lstlisting}
        \underline{render in half resolution}
        \begin{lstlisting}
    r.SSGI.HalfRes
        \end{lstlisting}

    \section{Lumen}
        \begin{itemize}
            \item has color bleeding
            \item soft shadows
        \end{itemize}

        \begin{itemize}
            \item directional light
            \item sky light
            \item sky atmosphere
        \end{itemize}
   
    \chapter{Collision}
        \href{https://www.unrealengine.com/en-US/blog/collision-filtering}{extensive blog entry}
        \uline{Keywords for collision:}
        \begin{itemize}
            \item \glsdesc{Collision Response (ECR)}
            \item 
        \end{itemize}

        You can define how objects should react to traces or how
        traces should react to object types.

        UE4 has a few ‘built in’ Trace Channels
        (Visibility, Camera) and Object Channels (WorldStatic, WorldDynamic, Pawn, PhysicsBody, Vehicle, Destructible),
        but you can easily add your own under Edit $\rightarrow$ Project Settings $\rightarrow$ Collision $\rightarrow$ 'Collision-Preset' to custom
        though you are limited to 32 in total. \\
        If both actors/components have the 'overlap' property they will go through each other.
        If only one has it, physix will be applied \\

        When two objects intersect, we look at how they respond to each other, and take the least blocking interaction
         \includegraphics[width=\textwidth]{CollisionExample.jpg}
        \underline{Object Types}
        \begin{table}[H]
            \begin{tabular}{|l|l|}
                \hline
                    Property & Description \\
                \hline
                    World Static & Actors that doesn't move. \\
                    World Dynamic & Actors that will move because of animation and code \\
                    Pawn & \\
                    PhysicsBody & \\
                    Vehicle & Vehicles reveice this type by default \\
                    Destructible & Destructible Meshes \\
                \hline
            \end{tabular}
        \end{table}
        
        \includegraphics[width=\textwidth]{Collision-Properties.png} \\

        \section{Create Custom-Collision-Box for modelled objects}
            \begin{itemize}
                \item Create Mesh
                \item Create Additional mesh with following Prefix
            \end{itemize}
            \begin{table}[H]
                \begin{tabular}{|l|l|}
                    \hline
                    Mesh Prefix and Name & Description \\
                    \hline
                    UBX\_[RenderMeshName]\_\#\# & Boxes are created with the Box objects. Don't move or deform \\
                    UCP\_[RenderMeshName]\_\#\# & Capsule (8 segements are good) \\
                    USP\_[RenderMeshName]\_\#\# & Spehere (8 segements) \\
                    UCX\_[RenderMeshName]\_\#\# & Convex shapes only (inner angles less then 180 degree) \\
                    \hline
                \end{tabular}
            \end{table}
            

        
\chapter{Gameplay ability system (GAS)}
    \begin{itemize}
        \item UAbilitySystemComponent := is the main thing
        \item FGameplayAbilitySpecContainer := holds its granted GAs
        \item Gameplay Ability := logic of the gameplay mechanic $\rightarrow$ everything can be gameplay ability
        \item Avatar := \glsdesc{GLS-Avatar}
        \item 
        \item 
        \item FActiveGameplayEffectsContainer := holds the currently active GEs
        \item Gameplay Effects: modifies attributes and tags instantly or for certain amount of time; how is it changing
        \item Gameplay AttributeSet := used on the same Actor that the ASC lives on (PlayerState / Character)
        \item Gameplay Attributes := actor properties(health, damage, state) modified by ASC; modified through gameplay effects; composed of \code{BaseValue} \code{CurrentValue}
        \item 
        \item Gameplay cues := for audio/visual effects 
        \item 
        \item GameplayTagManager := handles Tags globally
        \item GameplayTagContainer := is used inside ASCs $\rightarrow$ one on each Actor with an ASC
        \item GameplayTagMap := stores number of instances of GameplayTags
        \item Gameplay Tags := same principle as Actor Tags; handles intercaten between different gameplay- abilities \& effects; good to think about them as conditions
        \item TagMapCount
        \item 
        \item needs to implement \code{UAbilitySystemComponent* GetAbilitySystemComponent() const}
        \item Character:
        \begin{itemize}
            \item includes:
            \begin{itemize}
                \item "AbilitySystemInterface.h"
                \item "GameplayEffectTypes.h"
            \end{itemize}
            \item derives from 'public IAbilitySystemInterface'
        \end{itemize} 
        \item 
    \end{itemize}

    GAS as a whole is basically just a couple of basic UObject classes and an ActorComponent.
    The Component gets put on the actor that can use the ability. Then there is the actual
    GameplayAbility. This is just a UObject that gets instantiated based on settings. Once per
    user, once ever, or once per ability use I think. The component handles most all of the 
    "CanUse" stuff based on tags on the actor, etc. What you allow in the ability class is 
    entirely up to you. How you mutate the ability based on other factors is entirely up to 
    you. The other classes are basically a visual which is GameplayCues, and then there's 
    GameplayEffects which are what modify the attributes. You'd use an Effect to buff strength, 
    or heal, or damage health, etc. You use Gameplay Abilities to apply effects(though you can 
    apply effects from anywhere). So yeah. At it's core it's just a component that runs small 
    UObjects that can handle the lifetime of the ability.


    \section{Random notes}
    \begin{itemize}
        \item AbilitySystemComponent->RegisterGameplayTagEvent(FGameplayTag::RequestGameplayTag(FName("State.Debuff.Stun")), EGameplayTagEventType::NewOrRemoved).AddUObject(this, \&AGDPlayerState::StunTagChanged);
        \item This is from GASDocumentation (one of the pinned messages here and one of my favorite tools I'm learning GAS with)
        \item In this example the PlayerState has the function: virtual void StunTagChanged(const FGameplayTag CallbackTag, int32 NewCount); that's being bound to when the tag is applied

        \item 
    \end{itemize}
    
    \section{Main}
        \begin{itemize}
            \item should be added to the PlayerState (if presistant Attributes) or the character/pawn
            
            \item \code{ActivatableAbilities.Items}
            \item use \code{ABILITYLIST_SCOPE_LOCK();} to iterate over \code{ActivatableAbilities.Items}
            \item \textbf{remove} Abilities only outside of \code{ABILITYLIST_SCOPE_LOCK();} $\rightarrow$ will only remove if no locks in use
            
            \item IAbilitySystemInterface either on \code{PlayerState} or \code{Actor}
        \end{itemize}

    \section{AttributeSet}
        \begin{table}[!htb]
            \begin{tblr}{p{6cm} | p{6cm} | p{6cm}}
                \hline
                    Macro & Function Signature & Description \\
                \hline
                GAMEPLAYATTRIBUTE \_PROPERTY\_GETTER (UMyAttributeSet, Health) &
                static FGameplayAttribute GetHealth() &
                Static function, returns the FGameplayAttribute struct from the engine's reflection system \\
                
                GAMEPLAYATTRIBUTE \_VALUE\_GETTER (Health) &
                float GetHealth() const &
                Returns the current value of the "Health" Gameplay Attribute \\

                GAMEPLAYATTRIBUTE \_VALUE\_SETTER (Health) &
                void SetHealth(float NewVal) &
                Sets the "Health" Gameplay Attribute's value to NewVal \\

                GAMEPLAYATTRIBUTE \_VALUE\_INITTER (Health) &
                void InitHealth(float NewVal) &
                Initializes the "Health" Gameplay Attribute's value to NewVal \\
            \end{tblr}
        \caption{ caption }  
        \end{table}
        

    \section{Make changes to attributes}
        \begin{itemize}
            \item check if ablitySystem and the used  GameplayEffect exist
            \item use:
            \begin{itemize}
                \item \code{FGameplayEffectContextHandle EffectContext = AbilitySystem->MakeEffectContext();} to create a handle
                \item \code{EffectContext.AddSourceObject(this);}
            \end{itemize} 
            \item 
        \end{itemize}
    
    \section{Gameplay Abilities}
        \uline{Gameplay Ability Functions}
        \begin{table}[!htb]
            \begin{tblr}{p{6cm} | p{12cm}}
                \hline
                    Function & Usage \\
                \hline
                    \code{TryActivateAbility} &
                    typical way to execute abilities calls \code{CanActivateAbility} and then \code{CallActivateAbility} \\
                    
                    \code{CanActivateAbility} &
                    lets the caller know if ability is available (eg.: widgets call to indacate status) \\

                    \code{CallActivateAbility} &
                    executes the ability without prior checks \\

                    & \\

                    \code{ActivateAbility} &
                    main code to do things \\

                    \code{CommitAbility} &
                    applies costs to \code{Attribtues} \\

                    \code{CancelAbility} &
                    allows to cancel an ability from anywhere and broadcasts to \code{OnGamepalyAbilityCancelled} (can be prevented with \code{CanBeCanceled}) \\

                    & \\

                    \code{EndAbility} &
                    shuts down the ability. NOT calling this can lead to serious bugs where ability can't be activated anymore or blocks other abilities. \\

                \hline
            \end{tblr}
        \end{table}
    

        \uline{Gameplay Ability Tag Variables}
        \begin{table}[!htb]
            \begin{tblr}{p{6cm} | p{12cm}}
                \hline
                Gameplay Tag Variable(s) & Purpose\\
                \hline
                Cancel Abilities With Tag &
                Cancels any already-executing Ability with Tags matching the list provided while this Ability is executing. \\
                
                Block Abilities With Tag&
                Prevents execution of any other Ability with a matching Tag while this Ability is executing. \\
                
                Activation Owned Tags &
                While this Ability is executing, the owner of the Ability will be granted this set of Tags. \\
                
                Activation Required Tags &
                The Ability can only be activated if the activating Actor or Component has all of these Tags. \\
                
                Activation Blocked Tags &
                The Ability can only be activated if the activating Actor or Component does not have any of these Tags. \\
                
                Target Required Tags &
                The Ability can only be activated if the targeted Actor or Component has all of these Tags. \\
                
                Target Blocked Tags &
                The Ability can only be activated if the targeted Actor or Component does not have any of these Tags. \\
            \end{tblr}
        \end{table}
    
    \section{GameplayCues}
        \begin{itemize}
            \item 
        \end{itemize}
        
        \subsection{GameplayCueParameters}
            \begin{itemize}
                \item has:
                \begin{itemize}
                    \item \code{Instigator} := who owns the ability-system
                    \item \code{SourceObject} := the object that created the effect
                    \item \code{EffectCauser} := physical object that did the damage (auto set to avatarActor)
                \end{itemize}
            \end{itemize}
    
    \section{Code}
        \uline{define player attributes:}
            \begin{lstlisting}
        UPROPERTY(BlueprintReadOnly, Category="Health", ReplicatedUsing = OnRep_Health)
        FGameplayAttributeData Health;
        ATTRIBUTE_ACCESSORS(UWH_PawnAttributes, Health)
            \end{lstlisting}
            \uline{Give ability to players}
            \begin{lstlisting}
        AbilitySystemComponent->GiveAbility(FGameplayAbilitySpec(AbilityClass, AbilityLevel, InputEnum));
            \end{lstlisting}

            For player controlled characters where the ASC lives on the Pawn, I typically initialize on
            the server in the Pawn's PossessedBy() function and initialize on the client in the
            PlayerController's AcknowledgePossession() function.
            \begin{lstlisting}
void APACharacterBase::PossessedBy(AController * NewController)
{
    Super::PossessedBy(NewController);

    if (AbilitySystemComponent)
    {
        AbilitySystemComponent->InitAbilityActorInfo(this, this);
    }

    // ASC MixedMode replication requires that the ASC Owner's Owner be the Controller.
    SetOwner(NewController);
}
        \end{lstlisting}
        
        \begin{lstlisting}
void APAPlayerControllerBase::AcknowledgePossession(APawn* P)
{
    Super::AcknowledgePossession(P);

    APACharacterBase* CharacterBase = Cast<APACharacterBase>(P);
    if (CharacterBase)
    {
        CharacterBase->GetAbilitySystemComponent()->InitAbilityActorInfo(CharacterBase, CharacterBase);
    }

    //...
}
        \end{lstlisting}

        For player controlled characters where the ASC lives on the PlayerState,
        I typically initialize the server in the Pawn's \code{PossessedBy()} (which gets
        called even before \code{BeginPlay()}) function and
        initialize on the client in the Pawn's \code{OnRep_PlayerState()}-function.
        This ensures that the PlayerState exists on the client.
        \begin{lstlisting}
// Server only
void AGDHeroCharacter::PossessedBy(AController * NewController)
{
    Super::PossessedBy(NewController);

    AGDPlayerState* PS = GetPlayerState<AGDPlayerState>();
    if (PS)
    {
        // Set the ASC on the Server. Clients do this in OnRep_PlayerState()
        AbilitySystemComponent = Cast<UGDAbilitySystemComponent>(PS->GetAbilitySystemComponent());

        // AI won't have PlayerControllers so we can init again here just to be sure. No harm in initing twice for heroes that have PlayerControllers.
        PS->GetAbilitySystemComponent()->InitAbilityActorInfo(PS, this);
    }
    
    //...
}        
        \end{lstlisting}

        \begin{lstlisting}
// Client only
void AGDHeroCharacter::OnRep_PlayerState()
{
    Super::OnRep_PlayerState();

    AGDPlayerState* PS = GetPlayerState<AGDPlayerState>();
    if (PS)
    {
        // Set the ASC for clients. Server does this in PossessedBy.
        AbilitySystemComponent = Cast<UGDAbilitySystemComponent>(PS->GetAbilitySystemComponent());

        // Init ASC Actor Info for clients. Server will init its ASC when it possesses a new Actor.
        AbilitySystemComponent->InitAbilityActorInfo(PS, this);
    }

    // ...
}
        \end{lstlisting}


        \uline{Listen for Tag Changes}
        \begin{lstlisting}
    AbilitySystemComponent->RegisterGameplayTagEvent(
    FGameplayTag::RequestGameplayTag(FName("State.Debuff.Stun")),
    EGameplayTagEventType::NewOrRemoved).AddUObject(this, &AGDPlayerState::StunTagChanged);
        \end{lstlisting}
        \begin{lstlisting}
    virtual void StunTagChanged(const FGameplayTag CallbackTag, int32 NewCount);
        \end{lstlisting}

        \uline{Delegate for Attribute changes}
        \begin{lstlisting}
    AbilitySystemComponent->GetGameplayAttributeValueChangeDelegate(
    AttributeSetBase->GetHealthAttribute()).AddUObject(this, &AGDPlayerState::HealthChanged);
        \end{lstlisting}

        \begin{lstlisting}
    virtual void HealthChanged(const FOnAttributeChangeData& Data);
        \end{lstlisting}


    \chapter{Physics Actor}
        \begin{itemize}
            \item First actor you have to select is the anker
            \item Second actor is the object you want to attach
        \end{itemize}

\chapter{Working with Data}
    \section{Data Table}
        \begin{itemize}
            \item allows to handle Structs in a table
            \item 
            \item \textbf{create} a Struct that inherits from \code{FTableRowBase}
            \item add \code{UPROPERTIES}
            \item use in your classes
        \end{itemize}

        \uline{Specify row to use in class}
            \begin{itemize}
                \item \code{FDataTableRowHandle} : will allow to select a Tabel \& Row
                \item \textbf{specify table type:} \code{meta = (RowType = Item)} will only show tables with the specified type
            \end{itemize}

    \begin{itemize}
        \item 
    \end{itemize}


    \section{Data Assets}
        \begin{itemize}
            \item handle it like static thing
            \item Primary Data Assets Type have to be created first (C++)
            \item Data Asset smt. the data managers knows how to handle
            \item Project Settings -> Game -> Asset Manager
        \end{itemize}

\chapter{Mass Entity-Component-System}
    Normally you have \code{Classes} contain \code{Data} and \code{Methods} \\
    An \code{ECS}, on the other hand, saves \code{Data} and \code{Methods} seperatly. \\
    Additionally, an ECS can \code{multiple Parents}

    \section{Enable}
        Enable plugins
        \begin{itemize}
            \item MassAI
            \item MassCrowd
            \item MassEntity
            \item MassGameplay
        \end{itemize}
         \includegraphics[width=\textwidth]{MassPlugins.png}

    \section{Basics}
        \begin{itemize}
            \item there are Entities
            \item Entities hold Traits
            \item Traits assign Fragments to entities
            \item Data is held in Fragments
            \item Processors are used to manipulate the data
            \item Tags are used to filter for specific Entities
            \item 
            \item Add Fragment requierements
            \item 
            \item Archetype := Entity + Traits
        \end{itemize}
        \includegraphics[width=\textwidth]{EscStructure.png}


    \section{Workflow}
        \begin{itemize}
            \item creating new \code{Trait} := \code{Miscellaneous -> Data Asset} of type \code{MassEntityConfigAsset}
            \item Adding \code{Fragments} := (\code{USTRUCT struct UMassTraitBase StructName : public MassFragment}) to the Entity
            \item Attaching \code{Fragments} to \code{Traits} :=
            \begin{itemize}
                \item override the \code{BuildTemplate} virtual function
                \item call \code{BuildContext.AddFragment<FFragmentName>();}
            \end{itemize}  
            \item Adding \code{Processor} := 
            \begin{itemize}
                \item add \hyperref[MassConstructor]{Constructor}
                \item override \code{protected} \code{ConfigureQueries()} function
                \item override \code{Execute(...)} function 
                \item add a \code{private: FMassEntityQuery EntityQuery}
            \end{itemize} 
            \item Add an Actor to the scene:
            \begin{itemize}
                \item \code{Mass Spawner} := used to spawn Entities (how many, which MassEntityConfigAsset to use )
                \item \code{Mass Bubble Info Base} := 
                \item \code{Mass Crowd Client Bubble Info} := 
                \item \code{Mass Navigation Testing Actor} := 
            \end{itemize}
        \end{itemize}


    \section{Entity}
        \begin{itemize}
            \item Entity :=
            \item lightweight actor
        \end{itemize}

    \uline{include all Entities with a Tag or exclude all Entities with a Flag}
        \begin{lstlisting}
    EntityQuery.AddTagRequirement<FBuildyMassBuilderTag>(EMassFragmentPresence::All);
    EntityQuery.AddTagRequirement<FBuildyMassWarriorTag>(EMassFragmentPresence::None);
        \end{lstlisting}

        \uline{Add Tags}
        \begin{lstlisting}
    USTRUCT()
    struct FBuildyMassBuilderTag : public FMassTag
    {
        GENERATE_BODY()
    };
        \end{lstlisting}

    \section{Traits}
        \begin{itemize}
            \item Traits := 
            \begin{itemize}
                \item lightweigt components 
                \item tell which Fragments to add to an Entity
            \end{itemize}
        \end{itemize}

        \uline{Constructor for }
        \begin{lstlisting}

        \end{lstlisting}
        \label{MassConstructor}


    \section{Fragment}
        real data define smallest structs that define state of a given functionality




    \section{Processors}
        \begin{itemize}
            \item entities will be filtered by the fragments they hold
            \item only the fragments that a processor can manipulate will be forwarded to the processor
            \item is auto registered $\rightarrow$ MassSystem ensures that there will be a processor instance for the Entities
            \item one processor is processing all the data from all Entity instances in the level
        \end{itemize}    
    
        \subsection{MassObserver}
            \includegraphics[width=\textwidth]{MassObserver_Class.png}
            \includegraphics[width=\textwidth]{MassObserver_Registration.png}
            
            \begin{itemize}
                \item \code{UMassObserverProcessor}
                \begin{itemize}
                    \item helps initializing not yet initialized Fragments
                    \item registration must specify the type $\rightarrow$ \code{ObserverType = FAgentRadiusFragment::StaticStruct()}
                \end{itemize}
                \item 
            \end{itemize}

    

\chapter{AI}%TODO:
    \section{Basics}
        \begin{itemize}
            \item A pawn can be controlled by a player or AI
            \item AI needs: a component that specifies what it should do (Behavior tree) \& AIController
            \item AIController: observes the world and makes decisions based on it
            \item 
            \item \code{PawnSensing} := older
            \item \code{Perception-System} := newer
            \item
        \end{itemize}

        \uline{AI-Classification}
            AIs can be differentiated in order to optimize:
            \begin{itemize}
                \item Low AIs:
                \begin{itemize}
                    \item super low poly
                    \item gives the illusion of being fully featured
                \end{itemize}
                \item Medium AIs:
                \begin{itemize}
                    \item clustered
                    \item running simpler animations
                    \item low poly
                \end{itemize}
                \item High AIs:
                \begin{itemize}
                    \item Smart
                    \item complex animations
                \end{itemize}
                \item Culled AIs:
                \begin{itemize}
                    \item are not visible by player
                    \item can stop any dynamic behavior
                \end{itemize}

            \end{itemize}

            \subsection{Stop AI/BT}
                 \begin{itemize}
                    \item best solution: in order to be able to resume the AI you use \code{LockResources}
                    \begin{itemize}
                        \item \code{static void LockResources(UBrainComponent* Brain) { Brain->LockResource(EAIRequestPriority::Logic);}}
                        \item \code{static void LockPathFollowing(AAIController* Controller) { Controller->GetPathFollowingComponent()->LockResource(EAIRequestPriority::Logic); }}
                    \end{itemize}
                    \item can be done by unpossessing the pawn
                    \item OR you use the  \code{Stop Logic}-node from AIController $\rightarrow$ BrainComponent
                 \end{itemize}
                  \includegraphics[width=\textwidth]{StopBT.png}


    \section{NavMesh}
        \begin{itemize}
            \item stop meshes from influencing $\rightarrow$ totally ignore in navigation (invisible for AI)
            \item and also navigation section of the static mesh
            \item put the aiperception on the controller (so it is shared between the pawns?)
        \end{itemize}


    \section{Behavior Trees}
        Stores information about the states you can be in and actions to perform depending on those states.
        There are 2 types of Nodes
        \begin{itemize}
            \item Root: stores properties of the Behavior-Tree(BT); currently only the \code{Blackboard} (BB)
            \item Composite: Define the root of a branch and the base rules for how that branch is executed:
            \begin{itemize}
                \item Selector: executes from left-to-right until an execution is successfull and goes back to parent composite afterwards to continue flow
                \item Sequence: executes branches from left-to-right until it fails
                \item Simple Parallel: has 2 connections; 1. can only be a Task 2. is the "background branch"
            \end{itemize}
            \item Decorator: is added to other nodes and adds conditions under which the node should be executed
            \item Service: runs in the background of Tasks\&Composites at a frequency until the subtree finishes; used for checks and to update the BB
            \item Task: the actual behavior that should be executed (Move, Attack, ...)
            \begin{itemize}
                \item a dedicated graph to add functionality via BPs
                \item "Event Receive Execute AI" := called when this Task is activated
                \item "Event Receive Abort" := called when this task aborts
                \item "Event Receive Tick" := %TODO:
            \end{itemize} 
        \end{itemize}

    \smallskip
        The position of the nodes matters LEFT MOST IMPORTANT RIGHT LEAST IMPORTANT
    \smallskip
        You have to create tasks for some functionality you might want to have in your game:
        \begin{itemize}
            \item Move to a location
            \item 
        \end{itemize}
    \smallskip
        Useful Nodes in Event Graphes for BTTs
        \begin{itemize}
            \item Event Receive Execute AI
            \item GetRandomReachablePointInRadius
        \end{itemize}

    \section{Blackboard}
        In the BB you can specify variables that will influence the behaviour of the AIController

    \section{EQS}%TODO:
        \begin{itemize}
            \item allows to query the level
            \item components:
            \begin{itemize}
                \item Generators := used to produce the locations or actors that will be tested and weighted
                \item Contexts := used as a frame of reference for any Tests or Generators
            \end{itemize}
        \end{itemize}

    \section{Perception}
        \subsection{Basics}
            \begin{itemize}
                \item main parts:
                \begin{itemize}
                    \item \code{UPawnSensingComponent}
                    \item \code{UPawnNoiseEmitterComponent}
                    \item \code{}
                \end{itemize}
            \end{itemize}
        \subsection{Perception Tree}%TODO:
        \subsection{AI Controllers}%TODO:
            \begin{itemize}
                \item is used with a Pawn
                \item calls the actual BT
                \item contains a 'AIPerception'-Component
                \item 
                \item 
            \end{itemize}
        
        \subsection{AIPerception-Component}
            \begin{itemize}
                \item meant for 'AIController'
                \item 
                \item listens for 'Stimuli'
                \item saves 'Stimuli Sources'
                \item Events:
                \begin{itemize}
                    \item On Perception Updated := when stimuli source is registered
                    \item On Target Perception Updated := for specific stimuli
                    \item On Component Activated := ... 
                    \item On Component Deactivated := ...
                \end{itemize}
                \item 
                \item useable Senses:
                \begin{itemize}
                    \item AIDamage := how to react on 'Event Any Damage', 'Event Point Damage', 'Event Radial Damage'
                    \item AIHearing := how to react to detected sounds by 'Report Noise Event'
                    \item AIPrediction := asks Perception System to supply Requestor with PredictedActor's predicted location in PredictionTime seconds
                    \item AISight := ...
                    \item AITeam := tells if someone on the same team is close by
                    \item AITouch := detect when the AI bumps into something or something bumps into it
                \end{itemize}
                \item specifies the Senses the AI should have
                \item currently affiliation can only be added through C++
            \end{itemize}

        \subsection{AIPerceptionStimuliSource-Component}
            \begin{itemize}
                \item meant for 'Actors' that should influence AI
            \end{itemize}

    \section{Debugging}
        \href{https://docs.unrealengine.com/4.27/en-US/InteractiveExperiences/ArtificialIntelligence/AIDebugging/#perception}{Official DOCs}
        \begin{itemize}
            \item apostrophe to enable AI-debugging in the viewport
            \item 
        \end{itemize}

        \subsection{Visual-logger} \label{sec:visuallogger}
            \begin{itemize}
                \item timeline of events
                \item window $\rightarrow$ developer tools $\rightarrow$ visual logger
                \begin{itemize}
                    \item timeline
                    \item info panel (behaviour tree/what state actors are in)
                    \item debug logger for the recorded timeframe
                \end{itemize}
                    \item 
            \end{itemize}
            \underline{Code:} \\
            \includegraphics[width=\textwidth]{VisualLogger.jpg} \\

            \subsubsection{workflow}
                \begin{itemize}
                    \item use \code{DEFINE_LOG_CATEGORY_STATIC( LogSampleVisualLog, Log, All )}
                    \item use a define guard for the actual logging code \code{#if ENABLE_VISUAL_LOG ... #endif}
                    \item check if it's recording \code{FVisualLogger& Vlog = FVisualLogger.Get();}
                    \item check if it's recording \code{if( Vlog.IsRecording() )}
                    \item actually log using:
                    \begin{itemize}
                        \item \code{UE_VLOG_LOCATION}
                        \item \code{UE_VLOG_SEGMENT}
                        \item \code{UE_VLOG_UELOG}
                        \item more in \code{VisualLogger.h}
                    \end{itemize}
                \end{itemize}

        \subsection{Cheat commands}
            \begin{itemize}
                \item can be defined to define commands
            \end{itemize}


    \section{Smart-Objects}
        \begin{itemize}
            \item \code{GameplayBehavior} : defines what should be done ()
            \item \code{GameplayBehaviorConfig} : used to get a CDO of an assigned Gameplay-Behavior and used inside the \code{SmartObjectDefinition}
            \item \code{SmartObjectDefinition} : defines the behavior for each slot in the SO
            \item \code{SmartObject}(Component) : the component that is placed on an \code{Actor}; holds a reference a \code{SmartObjectDefinition}
            \item \code{}
            \item 
            \item has to call \code{EndBehavior} $\rightarrow$ will release the SO (set occupied to false)
            \item 
        \end{itemize}

        \subsection{Gameplay-Behaviour}
            \uline{Overridable Functions}
            \begin{itemize}
                \item \code{OnFinished}
                \item \code{OnFinishedCharacter}
                \item \code{OnFinishedPawn}
                \item \code{OnTriggered}
                \item \code{OnTriggeredCharacter}
                \item \code{OnTriggeredPawn}
            \end{itemize}


        \subsection{SmartObjectSlotDefinitionData}
            \begin{itemize}
                \item is an extension to the SmartObjectDefinition in the way that using it you can add special data to use
                \item can be accessed through \textbf{FStateTreeExecutionContext}
            \end{itemize}

            \subsubsection{SmartObjectSlotAnnotation}
                \begin{itemize}
                    \item adds functionallity to visualize additional data in the viewport
                    \item examples are:
                    \begin{itemize}
                        \item Entry/Exit-points for the SmartObjectSlot
                    \end{itemize}
                \end{itemize}




        \subsection{General workflow}
            \begin{itemize}
                \item create a \code{GameplayBehavior}
                \item create a \code{GameplayBehaviorConfig}
                \item assign the \code{GameplayBehavior} in the \code{GameplayBehaviorConfig}
                \item define the behavior in the \code{GameplayBehavior} inside one of the Overridable-functions and call \code{EndBehavior}
                \item 
                \item create a \code{DataAsset} of type \code{SmartObjectDefinition}
                \item create a slot in the \code{SmartObjectDefinition} and assign a \code{Default Behavior Definition}
                \item 
                \item create an \code{Actor} and add a \code{SmartObject}-Component
                \item assign the \code{GameplayBehavior} inside of the SO-Component
                \item 
                \item create a \code{BT} and \code{Blackboard}
                \item add a variable of type \code{SO Claim Handle} to the Blackboard
                \item create a \code{BT-Task} to 'Find the SO' and 'Use the SO'
                \item in the 'Find the SO' BT-Task use \code{Event Receive Execute AI} and the \code{Smart-Object-Subsystem} to find SmartObjects in a volume around the AI
                \item 
                \item 
            \end{itemize}


    \section{StateTree}
        \begin{itemize}
            \item Is
            \begin{itemize}
                \item \code{StateTreeSchema}: defines which inputs, evaluators and tasks can be used
                \item \code{FGameplayInteractionContext}: holds the data to perform the action
                \item 
            \end{itemize}
            \item Has
            \begin{itemize}
                \item States
                \item Transitions
                \item Tasks
                \item Evaluators
            \end{itemize}
        \end{itemize}

        \textbf{Example schema}
        \begin{lstlisting}
    UGameplayInteractionStateTreeSchema::UGameplayInteractionStateTreeSchema()
    : ContextActorClass(AActor::StaticClass())
    , SmartObjectActorClass(AActor::StaticClass())
    ,ContextDataDescs({
        {UE::GameplayInteraction::Names::ContextActor, AActor::StaticClass(), FGuid(0xDFB93B9E, 0xEDBE4906, 0x851C66B2, 0x7585FA21)},
        {UE::GameplayInteraction::Names::SmartObjectActor, AActor::StaticClass(), FGuid(0x870E433F, 0x99314B95, 0x982B78B0, 0x1B63BBD1)},
        {UE::GameplayInteraction::Names::SmartObjectClaimedHandle, FSmartObjectClaimHandle::StaticStruct(), FGuid(0x13BAB427, 0x26DB4A4A, 0xBD5F937E, 0xDB39F841)},
        {UE::GameplayInteraction::Names::SlotEntranceHandle, FSmartObjectSlotEntranceHandle::StaticStruct(), FGuid(0x283CBA09, 0x95CD42CF, 0xA11F510E, 0x17CB3530)},
        {UE::GameplayInteraction::Names::AbortContext, FGameplayInteractionAbortContext::StaticStruct(), FGuid(0xEED35411, 0x85E844A0, 0x95BE6DB5, 0xB63F51BC)},
    })
{
}
        \end{lstlisting}

        \subsection{Tasks}
            \begin{itemize}
                \item have a:
                \begin{itemize}
                    \item \code{Enter State}
                    \item \code{Tick}
                    \item \code{Exit State}
                \end{itemize}
                \item be carefull with \code{Finish Task}
            \end{itemize}


        \subsection{Random notes}
            \begin{itemize}
                \item in order to stop some logic use \code{LockResource} and \code{ClearResourceLock}
            \end{itemize}

    \section{Contextual Anim Scene}
        problems (with potential fixes):
        \begin{itemize}
            \item motion warping doesn't work 
            \begin{itemize}
                \item make sure you set the \code{role} and \code{Target Name} when providing the location and rotation in the task
                \item the motion warping window is too short (0.7s is the minimum)
            \end{itemize}
            \item contextual anim scene doesn't play the lead in animation when it's a little short $\rightarrow$ having a longer animation fixes it but looks weird
            \item 
            \item you can define \code{IKTargets} in the \code{Sections}
        \end{itemize}

        \begin{figure}
            \includegraphics[width=\textwidth]{ContextualAnimScene_IKTargets.png}
            \caption{ContextualAnimScene IKTargets can be added in sections}
            \label{}
        \end{figure}

    \section{Crowd}
        Use either DetourCrowd AIController or RVO \\
        RVO (reciprocal velocity obstacle) avoidance uses forces to push characters while they are moving to avoid each other. \\
        Detour works by updating the path to do the avoidance. \\

\chapter{Animations}

    \section{Unsorted}
        \begin{itemize}
            \item ANimation Curve := 
        \end{itemize}


    \section{Basic Setup}
        \begin{itemize}
            \item Create 'Animation Blueprint' for the character in question with the appropriate skeleton
            \item Create a blendspace with the character skeleton mesh
            \item create an animation graph
        \end{itemize}


    \section{Basic Assets}
        \subsection{Skeleton Asset}
            \begin{itemize}
                \item equivalent to the blender armature
                \item you can attach skeletal meshes to them
                \item skeletons can be used by multiple meshes
                \item rules:
                \begin{itemize}
                    \item keep order and names
                    \item you can append any number of bones
                    \item you can add any number of bones
                \end{itemize}
            \end{itemize}
            \href{https://docs.unrealengine.com/4.27/en-US/AnimatingObjects/SkeletalMeshAnimation/Skeleton/}{Official Doc}

        \subsection{Animations Sequence}
            \begin{figure}
                \includegraphics[width=\textwidth]{Animation_Sequence.jpg}
                \caption{How the Editor-Window looks like for Animation-Sequences}
            \end{figure}
            \begin{itemize}
                \item are the basic animations that are exported from blender and imported in UE5
                \item Each Animation Sequence asset targets a specific Skeleton and can only be played on that Skeleton.
                \item 'Notify' := will fire events at specific points during the 'Animation Sequence'
                \item there are different types of Notifies:
                \begin{itemize}
                    \item Skeletal Notify := is a generic notify and can be accessed inside of Animation Blueprints and their State Machines E.g.: foot steps, particle effects
                    \item Cloth Simulation Notifies := can change the cloth simulation property to play, pause, resume and reset 
                    \item Play Particle Effect := will play a particle effect and it's not available in Animation Blueprints but the details panel gives some options
                    \item Play Sound := will play a sound and it's not available in Animatin Blueprints but the details panel gives some options
                    \item Reset Dynamics := will reset any AnimDynamics
                \end{itemize}
                \item 'Notify State' := 
                \item 'Sync Marker' :=
                \item 
                \item Can be additive := so they can influence other animations inside your 'Animation Blueprint'
                \item in the 'Animation Sequence' under the section 'Additive Settings' :
                \begin{itemize}
                    \item No additive := 
                    \item Local Space := 
                    \item Mesh Space := 
                \end{itemize}
            \end{itemize}
            \href{https://docs.unrealengine.com/4.27/en-US/AnimatingObjects/SkeletalMeshAnimation/Sequences/}{Offizial Doc}

        \subsection{Animation Curves}
            \begin{itemize}
                \item 
            \end{itemize}

        \subsection{Animation Blueprints}
            \begin{figure}[ht]
                \includegraphics[]{Animation_Blueprint.jpg}
                \caption{Structure of a Animation Blueprint}
            \end{figure}
            \begin{itemize}
                \item Controls animation of ONE skeletal Mesh
                \item Has an 'Anim Graph' \& a 'Event Graph'
                \item Anim Graph := with state machine which is used for constant animations like idle, walking, running ...
                \item Event Graph := 
                \begin{itemize}
                    \item Used to update variables
                    \item 
                    \item 'Event Blueprint Initialize Animation' used for the setup
                    \item 'Event Blueprint Update Animation' is the equivalent to the 'Tick' in normal BPs
                \end{itemize}
                \item You will be able to define the exact behaviour inside the 'Animation Graph' with the help of a state machine
                \item In order for the state machine to transition between states you need to feed it with the needed information
                \item This information gathering will be setup inside of the 'Event Graph' of the 'Animation Blueprint'
            \end{itemize}

            \begin{itemize}
                \item B(lend)S(paces) in the state machine will have input nodes
                \item 'Layered blend per bone'
                \item Base Pose: Animation 
                \item Blend Poses: 
            \end{itemize}
    
        \subsection{Animation Montage}
            \begin{itemize}
                \item 
                \item 
                \item enables you to combine animation sequences and add logic
                \item exposing animation controls to blueprints or code
                \item 
                \item add logic to animations (ex. reloading)
            \end{itemize}

        \subsection{Blendspace}
        \begin{figure}
            \includegraphics[width=\textwidth]{Blend_Space.jpg}
            \caption{How the Editor-Window looks like for 'Blend Spaces'}
        \end{figure}
        \begin{itemize}
            \item Blendspace := takes in input and blends between different animations depending on the input values
            \item the input values are inputs on nodes you add to the Animation Blueprint
            \item 1D-Blendspaces := take in only 1 input
            \item 2D-Blendspaces := take in 2 inputs
        \end{itemize}


    \section{Retargeting}
        \begin{minipage}{0.9\textwidth}
            \begingroup \parfillskip=0pt
                \begin{minipage}[t]{0.49\textwidth}
                    \includegraphics[width=\textwidth]{ShowAllBones.png}

                    \label{}
                \end{minipage}
                \begin{minipage}[t]{0.49\textwidth}
                    \includegraphics[width=\textwidth]{NonRetargetedAnimation.png}

                    \label{}
                \end{minipage}
            \par\endgroup
        \end{minipage}
        
        \begin{itemize}
            \item the skeleton saves the bones and their translation(location) data
            \item in order to fix that $\rightarrow$ retarget the animation/animation-blueprint to another skeleton
        \end{itemize}
        \begin{figure}
            \includegraphics[width=\textwidth]{PreRetargeting.jpg}
            \caption{}
        \end{figure}
        \begin{figure}
            \includegraphics[width=\textwidth]{PostRetargeting.jpg}
            \caption{}
        \end{figure}
        \uline{Workflow:}
        \begin{itemize}
            \item Select animation you want to retarget
            \item RC $\rightarrow$ 'Retarget Anim Asset'
            \item Source Skeleton will be selected automatically
            \item Search for the Skeleton you want to retarget to ('Target')
            \item the location of the bones will be adjusted and a new 'Anim Asset' will be created
            \item the rotation does NOT get adjusted $\rightarrow$ the 'Base Pose' has to be adjusted in the 'Retarget Manager'
            \item if the skeletons have differences you might have to change the bone mappings e.g.: greystone
        \end{itemize}
        \uline{some settings to set before retargeting}
        \begin{itemize}
            \item in the skeleton you can set the type of \glqq TranslationRetargeting\grqq
            \begin{itemize}
                \item Animation := bone translation comes from the animation data, unchanged
                \item Skeleton := bone translation comes from the target skeleton's bind pose
                \item Animation scaled := comes from the animation data but is scaled by the skeleton proportions
                \item Animation relative
            \end{itemize}
        \end{itemize}

    
    \uline{copy pose from mesh}  
    \begin{itemize}
        \item feeds in skeletal information from one skeleton to another
        \item takes in source mesh component
        \item 'use attached parent' to get the parent in the ABP
        \item skeletons don't have to match
        \item name and hierarchy match
    \end{itemize}
	
    \uline{retarget pose from mesh}
    \begin{itemize}
        \item uses an IK-Retargetter
        \item -> check compatability by opening the retargetter and set own character as preview
        \item -> check animations
        \item 'use attached parent'
    \end{itemize}
    
    \subsection{IK-Retargetter}
        \begin{itemize}
            \item made with IK-Rigs for easier retargeting
            \item 
        \end{itemize}
        \subsubsection{Notes from experience (solutions to problems) }
            \begin{itemize}
                \item finger retargeting is bad $\rightarrow$ disable \code{FK} retargeting on the metacarpals
                \item 
            \end{itemize}


    \section{Skeleton LODs}
        

    \section{Addative Animation}
        \begin{itemize}
            \item animation is added on top of another
            \item examples
            \begin{itemize}
                \item breathing
                \item compression after jump
                \item leaning forwards while running
            \end{itemize}
        \end{itemize}
        
        \subsection{Nodes}

        
        \href{https://www.youtube.com/watch?v=flHL3qJB3_I}{YT explenation}

    \section{ALS}
        \subsection{Custom character}
            \begin{itemize}
                \item create an IK-Rig for the character
                \item create a retargetter depending for the custom character
                \item create ABP for your character with \code{Retarget Pose from Mesh} and set
                \begin{itemize}
                    \item \code{Use attached Parent} to \code{true}
                    \item \code{IKRetargeter Asset} to the one you created
                \end{itemize}
                \item add a \glqq Sekeletal Mesh component\grqq for your custom character
                \item set
                \begin{itemize}
                    \item \code{Hidden in Game} to \code{true}
                    \item under Optimization set \code{Visibility based anim tick option} to \code{Always Tick Pose and Refresh Bones}
                \end{itemize}
                
            \end{itemize}

    \section{Blend Poses depending on a Boolean value}
        \href{https://www.youtube.com/watch?v=TiQjjcvM7Eo}{YT explenation}
    

    \section{AnimDynamics}
        \begin{itemize}
            \item is a node inside the AnimGraph
            \item can be used to add physically based secondary animations for pouches, necklaces, backpacks ...
            \item detailed video under \href{https://www.youtube.com/watch?v=5h5CvZEBBWo}{https://www.youtube.com/watch?v=5h5CvZEBBWo}
        \end{itemize}

    \section{Animation Modifier}
        Enable users to define a sequence of actions for a given animation sequence or skeleton which enables Animation Sync Markers

    \section{State Machine vs Animation Montage}
        Something that happens upon a button press (attack) \\
        Also interruptable animations are placed here. You will specify a specific frame on which the animation can
        branch out playing different animations depending on some condition \\
        Slots 

        Nodes for the CharacterBP:
        \begin{itemize}
            \item Play Anim Montage (simple)
            \item Play Montage (way more complex)
            \item Layered blend per bone
            \item blend poses by bool
        \end{itemize}
    \smallskip


    \section{Blending Animations with Blendspace \& Animation-Graph}
        \uline{Blendspaces}

    \smallskip
        \uline{Animationsgraph}
        \begin{itemize}
            \item AnimationsGraphs have Animation-States between which you can switch depending on conditions
            \item 
        \end{itemize}

    \section{Add muscle progress or Head/Body Rotation}
        \begin{itemize}
            \item the 'Transform Bone'-Node in the 'AnimGraph' can manipulate specific bones
            \item e.g.: rotate the head depending on cursor location || scale a body part depending on muscle mass ...
        \end{itemize}

    \section{Example LeftRightDab}
        \begin{figure}
            \includegraphics[width=\textwidth]{LeftRightDab.jpg}
            \caption{Example setup for leaning where bool values are set in the character BP and used in the AnimBP}
        \end{figure}

    \section{Clothing}
        \href{https://docs.unrealengine.com/4.27/en-US/InteractiveExperiences/Physics/Cloth/Overview/}{Official Doc}

    \section{Veretex-Animation}
        \begin{itemize}
            \item create 'Animation Sequences'
            \item combine all animations into a 'Animation Composite'
            \item use maya with 'vertex animation tools' to create vertex animation textures %TODO: Blender?
            \item 
        \end{itemize}

    \section{Skeleton LODs}
         \includegraphics[width=\textwidth]{Skeleton_Editor.jpg}
        how to add LODs for the skeleton:
        \begin{itemize}
            \item go into the open the 'Skeleton Editor' of the skeleton
            \item go into the 'Mesh' section
            \item specify the bones you want to remove for each LOD in 'Asset Details'
        \end{itemize}
        \includegraphics[width=\textwidth]{Asset_Details.jpg}
        \href{https://www.youtube.com/watch?v=ti8NopRIgFs}{YT ref}



    \section{Motion Warping}
            \begin{itemize}
            \item Motion Warping := character animations will be adjusted depending on prerequisites (move to certain location before playing the next section of an animation)
            \item \code{RootMotion} must be enabled on the Animation
            \item 
            \item Setup:
            \begin{itemize}
                \item Create Notifies 'Motion Warping' == Motion Windows (1 for rotation and 1 for translation)
                \item before you play the animation montage, use 'Add or Update Sync Point'-Nodes for every 'Motion Warping'-Notify
                \item use 'Make Motion Warping Sync Point'-Node to specify where it should warp to
                \item 'Warp Point Anim Provider' will be in the next UE version
            \end{itemize}
        \end{itemize}

    \section{ControlRig}
        \begin{figure}
            \includegraphics[width=\textwidth]{EditorLayout.png}
            \caption{Recommended layout (used by a lot of epic employees)}
            \label{}
        \end{figure}

        

        \textbf{\uline{basic events in BP}}        
        \begin{itemize}
            \item Construction: where we can create the constrols and extra bones
            \item Forwards-Solve: where the logic for Ctrl $\rightarrow$ Bone is setup
            \item Backwards-Solve: where 
            \item Interaction-Event: any time the controls als moved
        \end{itemize}
        \textbf{\uline{some concepts}}
        \begin{itemize}
            \item Indirect-Controls / Proxy-Control: drives multiple things
            \item dynamic hierarchy: where controls can
            \item 
        \end{itemize}
        
        \begin{figure}
            \includegraphics[width=\textwidth]{Animation_UsefulOptions.png}
            \caption{Some useful options for the viewport when in AnimationMode}
            \label{}
        \end{figure}

        \textbf{\uline{Useful shortcuts}}
        \begin{itemize}
            \item Ctrl + G: reset transform on selected
            \item Ctrl + Shift + G: Reset all control transform
            \item 1 Create Get Transform Node
            \item 2 Create Set Transform Node
        \end{itemize}

        \subsection{Full-Body IK}
            \begin{itemize}
                \item create a ControlRig IK
                \item add the 'Full Body IK'-Node and set the 'Root' to the first skinned bone (most cases pelvis)
                \item Add effectors = $\rightarrow$ RC the bones $\rightarrow$ 'New' $\rightarrow$ 'New Control'
                \item Unparent the effectors
                \item Drag effectors into the 'Rig Graph' $\rightarrow$ connect the 'Transform' to 'Full Body IK'-Node
                \item set bones
                \item add the constraints under 'Bone Settings'
                \item 
                \item Bottom 'Full Body IK'-Settings:
                \begin{itemize}
                    \item 'Iterations': rel. cheap (100 not that bad)
                    \item 'Mass Multiplier': bones rotate depending on their length $\rightarrow$ higher Mass-value = less rotation
                    \item 
                \end{itemize}
            \end{itemize}

    \section{Anim Link Layers}
        \begin{itemize}
            \item is a way to splilt huge anim\_BPs into smaller ones and link them together
            \item 
            \item are instances of animation blueprints
            \item hold functions for every pose
        \end{itemize}

        \uline{SETUP}
        \begin{itemize}
            \item Create a seperate ABP
            \item copy\&paste the state machines you want to split from the main graph
            \item add \code{Linked Anim Graph}-Node to your main graph
            \item select the ABP you created
            \item assign variables that you update in the main ABP to the ones used in the linked graph
            \item 
            \item if the additional ABP needs the output of another ABP use \code{Input Pose}-Node in the using graph
            \item $\hookrightarrow$ adds an input pin when it's used in a \code{Linked Anim Graph}-Node
        \end{itemize}
            \includegraphics[width=\textwidth]{Bilder/LinkedGraphDetails.png}

        \uline{ANIMATION LAYER INTERFACE}
        \begin{itemize}
            \item used for dynamic linked graphs
        \end{itemize}

        \uline{SETUP}
        \begin{itemize}
            \item create an \code{Animtion Layer Interface}
            \item add \code{Animation Layers}
            \item 
            \item create a default implementation in the main graph that will use the linked layer
            \item eg.: use default logic as input\&result
            \item 
            \item create ABPs that implement that interface
            \item implement the actual Layer (double click the animation layer in the \code{My Blueprint}-tab)
            \item 
            \item Add the specific \code{Linked Anim Layer}-Node where you want to use it
            \item 
            \item Set the linked graph from your characters Mesh-Component
            \item 
            \item unlink the anim
        \end{itemize}
            \includegraphics[width=\textwidth]{Bilder/LinkedLayerInterface.png}


    \section{Usefull commands}
        \code{showflag.bones 1}


    \section{Motion Matching}

        \subsection{Basic Concepts}
            \begin{enumerate}
                \item \code{Pose History} := stores the last X poses of the character
                \item \code{Pose-Search-Database} := contains a list of animation assets and a schema
                \item \code{Pose-Search-Schema} := describes what features/properties you care about (eg.: trajectory, position, velocity)
                \item \code{Chooser-Table} := specifies what motion matching can currently use; uses many variables to reduce number of possible animations
                \item \code{Interrupt Mode} := can be specified to either interrupt and change Pose-Search-Database or keep the currently running animation a little longer
                \item 
            \end{enumerate}


        \subsection{Pose Search Database}
            \begin{enumerate}
                \item every channel has a weight, the higher the weight the more important the channel is
            \end{enumerate}
        


        \subsection{}
            \begin{enumerate}
                \item each movement should have it's own pose search database $\rightarrow$ gives more precise pose selection
                \item otherwise the chooser might stay longer in the wrong pose
            \end{enumerate}


        \subsection{Nodes}
            \begin{table}[!htb]
                \begin{tblr}{p{6cm} | p{12cm}}
                    \hline
                        Node & Description \\
                    \hline
                        Pose History & \makecell[l]{collects information for the specified bones, trajectories ...} \\
                        Motion Matching & \makecell[l]{takes input (pose search database) \\ finds the best matching pose from the database and returns it} \\
                        Evaluate Chooser & \makecell[l]{takes the output of the Motion Matching \\ and decides which animation to play} \\
                    \hline
                \end{tblr}
            \caption{ caption }  
            \end{table}

        \subsection{Setup}

        \begin{itemize}
            \item first add a \code{Motion Matching}-Node to the AnimGraph
            \item hook it up to the \code{Pose Histroy}-Node
            \item 
            \item create a \code{Motion Matching Data Asset}
            \item add the animations you want to use
            \item 
            \item create a code{Space-Search-Schema}-Asset := describes what features/properties you care about
            \item first select the skeleton it should be used on
            \item add \code{Channels}
            \item the \code{Samples}
            \begin{itemize}
                \item \code{Offset} := character-location on the trajectory at X seconds (negative := X seconds ago; positive := X seconds in the future)
                \item \code{Weight} := how important is the sample
            \end{itemize}
        \end{itemize}

        pose-search-feature-channel
        \begin{itemize}
            \item those have functions you can override
            \item 
        \end{itemize}

        \subsection{Debugging}
            tools:
            \begin{itemize}
                \item \code{Game Animation}
                \item \code{Rewind Debugger}
            \end{itemize}

            \begin{itemize}
                \item \code{ShowFlag.MotionMatching 1}
                \item \code{ShowFlag.MotionMatching 2}
            \end{itemize}

            \subsubsection{Rewind Debugger}
                \begin{itemize}
                    \item \code{Cost} := 
                    \item 
                \end{itemize}

\chapter{Blueprints}
    \section{Notes}
        \begin{itemize}
            \item What to use:
            \begin{itemize}
                \item Sender $\rightarrow$ Receiver = Direct communication || Interface
                \begin{itemize}
                    \item When you know everything about the communication except the time
                    \item which objects are part of the communication
                \end{itemize}
                \item Listening Receiver = Event Dispatcher
            \end{itemize}
            \item BlueprintPure := just executes the function without the need for input of timing
            \item 
            \item Removing objects from an array will break in normal forEachLoop (abcde $\rightarrow$ remove c $\rightarrow$ you skip d \& end up at e)
        \end{itemize}


    \section{Blueprint Communication}
        Basics of BP communication:
        \begin{itemize}
            \item UE4 has no broadcast functionality
            \item there is always a sender \& receiver involved
            \item $\rightarrow$ at least one participant has to know about the other via REFERENCE
        \end{itemize}
        Types of communication
        \begin{itemize}
            \item Direct-Communication: one to one with known participants and functionality
            \item Interfaces: functionality without of Implementation 
        \end{itemize}
\smallskip
        CASTING: is a way to test if some object has class specific functionality/properties \\
\smallskip
    \subsection{Interface}
        Are assets themself \\
        \begin{itemize}
            \item RC in content browser Blueprints $\rightarrow$ Blueprint-Interface
            \item You can't make anything in the graph except selecting the function
            \item You'll add functions and properties on the right side
            \item Click the specific function you want to add Input/Output to it
            \item 
            \item Add to BPs: under 'Class Settings' $\rightarrow$ 'Details' $\rightarrow$ 'Interfaces' $\rightarrow$ add your interfaces
            \item 
        \end{itemize}


    \subsection{Event Dispatcher (Publish-Subscribe)}
        \begin{itemize}
            \item Create a dispatcher
            \item Call the dispatcher on an event (ex. key press)
            \item Bind something to the dispatcher (adding a delegate) so it will be executed when dispatcher gets called
            \item Hint: it has only to be bound once $\rightarrow$ ex. construct event
        \end{itemize}

        You attach a disptacher to the sender. It can have inputs to send with the message \\
        When the sender wants to send a message he has to call the dispatcher \\
        The receiver makes a reference to the sender (In contrast to the other communication types) \\
        Then anyone can subscribe(bind)/unsubscribe(unbind) to the sender, to acknowledge the message and react to it. \\
        In an Event-Dispatcher-Map := are the rules defined to handle the event \\
        \includegraphics[width=\textwidth]{EventDispatcher.png} \\
        After creating dispatcher you can
        \begin{itemize}
            \item Add nodes:
            \item Bind nodes: bind a specific 'Custom Event' so it fires in response to the Dispatcher
            \item Unbind nodes: unbind a specific 'Custom Event' from a Dispatcher so you're not longer listening
                \begin{itemize}
                    \item from Class Blueprint $\rightarrow$ will unbind all
                    \item from Level Blueprint $\rightarrow$ unbind just for the target
                \end{itemize}
            \item Unbind all: will unbind all Events from the Dispatcher
            \item Call: will broadcast/publish to every receiver
            \item Event: adds a custom event with a signature matching the event dispatcher
            \item Assign: bind + assign/create new event
        \end{itemize}
\smallskip
        \underline{Event Types:}
        \begin{itemize}
            \item OnClick
            \item OnOverlap
            \item  
        \end{itemize}


    \section{Useful BP Nodes}    
        \begin{itemize}
            \item Add Controller Yaw (Rechts-Links) \& Add Controller Pitch(Up-Down) 
            \item Branch = if-else
            \item Gate := 
            \item Timeline
            \item CastToGenericCharacter
            \item SetTimerByFunction / SetTimerByEvent \href{https://docs.unrealengine.com/en-US/Gameplay/HowTo/UseTimers/Blueprints/index.html}{Additional Info and nodes}
        \end{itemize}
        \includegraphics[width=\textwidth]{CameraControl.png} \\
        \includegraphics[width=\textwidth]{MovementControl.png} \\
        Zu beachten ist hier das beim ''CameraControl'' in Details des ''BP\_Character''
        unter den Details des ''Camera Component'' ''Camera Options'' $\rightarrow$ ''Use Pawn Control Rotation'' angekreuzt werden muss \\

        \subsection{SetActorLocation}
        \subsection{SetActorRotation}
        \subsection{SetActorLocationAndRotation}
        \subsection{Construct Object from Class}
        \subsection{Add Force}
        \subsection{Events}
            There a a lot of different Events that can be used in BPs \\
            \includegraphics[width=\textwidth]{Events.png} \\
        \subsection{Timeline}
            The Timeline can output values depending on the time.
            \begin{itemize}
                \item Use Last Keyframe := will set the last Keyframe as the END
            \end{itemize}

        \subsection{Open Level}
            Will open the level you specify in the name box \\
        \subsection{Load Level Instance}
        \subsection{IsValid}
            Checks if the specified return value is NULLPTR
        \subsection{All trace functions}
    
    \section{Traces/Ray-Casting}
        \begin{itemize}
            \item Create a node for (Multi)Line-Traces (ByChannel, ByObject,ByProfile)
            \item Get the location of the thing you want the line trace to be cast from and plug into start
            \item get rotation of it and 'GetForwardVector' which creates a unified Vector which needs to be multiplied by the distance
            \item then add unified vector multiplied by the distance to the start location and plug into end
            \item Draw Debug Type For One Frame
        \end{itemize}
        There are different types of Ray-Casting \\
        \subsection{Single Ray-Cast-by-channel}


\chapter{C++}

    \section{Unsorted}
        \begin{itemize}
            \item disable optimisation for a specific file \\ \code{#pragma message("*****Optimisation disabled on " __FILE__) PRAGMA_DISABLE_OPTIMIZATION}
            \item $\hookrightarrow$ and disable \code{unity build} in \code{.build.cs} with \code{bUseUnity = false;}
            \item 
            \item the FSlateApplication tracks a LastUserInteractionTime double property for any input to the application
            \item \code{TObjectPtr} can just be replaced with rawPointer
            \item Block-Comments above \code{UCLASS/UPROPERTY} will create documentation in the editor
            \item 
            \item \code{AddUObject} vs \code{AddUFunction} add uobject is a native delegate binding that works similar to add static, except checks for object alive first
            add ufunction uses reflection to invoke it and is thus extremely slow
            \item \code{OnConstruction} is called after \code{AActor::UserConstructionScript} is run first
            \item 
            \item code you need to include engine modules \code{System.IO.Path.GetFullPath(Target.RelativeEnginePath) + "Source\\Editor\\Blutility\\Private"}
            \item 
            \item \code{TSharedFromThis} is required to use \code{SharedThis} or \code{AsShared()}, but you can also use \code{MakeShareable} for a type that doesn't use it 
                    \code{MakeShared} creates a new reference counter
                    \code{MakeShareable} does not
            \item can't use UPROPERTY in MACROS
        \end{itemize}

    \section{Naming Conventions}
        \begin{itemize}
            \item Types := are nouns
            \item Methods := are verbs or describe return value of a method that has no effect
            \item IsVisible/ShouldClearBuffer := Functions returning booleans
            \item Procedure should use a strong verb followed by an object
            \item Out := Parameters passed by reference and changed inside function
            \item const := marked pointer or reference arguments that will not be changed inside / methods !modify the object
        \end{itemize}
        \uline{Examples}
        \begin{itemize}
            \item float TeaWeight;
            \item int32 TeaCount;
            \item bool bDoesTeaStink;
            \item FName TeaName;
            \item FString TeaFriendlyName;
            \item UClass* TeaClass;
            \item USoundCue* TeaSound;
            \item UTexture* TeaTexture;
        \end{itemize}
    \smallskip
        \uline{Type Names:}
        \begin{itemize}
            \item U := UObject
            \item A := AActor
            \item S := SWidget
            \item I := Abstract Interfaces
            \item E := Enums
            \item b := Boolean
            \item F := most other classes
        \end{itemize}
        
\bigskip

        \uline{Engine Structure for modules (Engine/Source)}
        \begin{itemize}
            \item Developer: for any application but used during development only
            \item Editor: for use in unreal Editor
            \item Runtime: for any application at any time
            \item ThirdParty: code and libraries from external third parties
        \end{itemize}

    \section{Smart Pointers}
    \href{https://docs.unrealengine.com/5.0/en-US/smart-pointers-in-unreal-engine/}{off. Doc}
        \begin{table}[!htb]
            \begin{tblr}{p{6cm}|p{12cm}}
                \hline
                    Pointer Type & Properties \\
                \hline
                    Shared Pointer (TSharedPtr) & {owns the object it references \\ prevents deletion of the object \\ nullable} \\
                    Shared Reference (TSharedRef) & {owns the object it references \\ non-nullable -> ensures valid object} \\
                    Weak Pointer (TWeakPtr) & {don't own the object \\ do not increase ref count \\ nullable} \\
                    Unique Pointer (TUniquePtr) & {is the only pointer that owns an object \\ can transfer ownership \\ cannot share ownership \\ will delete referenced object when pointer gets out of scope} \\
                \hline
            \end{tblr}
        \caption{ caption }  
        \end{table}

        \begin{table}[!htb]
            \begin{tblr}{p{6cm}|p{12cm}}
                \hline
                    Helper Class/Function & Description \\
                \hline
                    : public TSharedFromThis & Class that derives from TSharedFromThis adds the AsShared SharedThis functions -> enables you to acquire a TSharedRef to your object \\
                    MakeShared / MakeShareable & {creates a shared pointer from a raw C++ pointer \\ MakeShared allocates a new object instance and the reference controller in a single memory block but requires the object to have a public constructor \\ MakeShareable works on private constructors, lets you take ownership \\ supports customized behaviour when deleting the object} \\
                    {StaticCastSharedRef \\ StaticCastSharedPtr} & typically used to downcast to a derived type \\
                    {ConstCastSharedRef \\ ConstCastSharedPtr} & converts a const SmartRef to a mutable SmartRef / SmartPtr \\
                \hline
            \end{tblr}
        \caption{ caption }  
        \end{table}
        Up-casting is implicit. \code{StaticCastSharedPtr} to downcast to a derived class
        \begin{lstlisting}
    // This assumes we validated that the FDragDropOperation is actually an FAssetDragDropOp through other means.
    TSharedPtr<FDragDropOperation> Operation = DragDropEvent.GetOperation();
    // We can now cast with StaticCastSharedPtr.
    TSharedPtr<FAssetDragDropOp> DragDropOp = StaticCastSharedPtr<FAssetDragDropOp>(Operation);
        \end{lstlisting}

    \section{Important Modules}
        \begin{itemize}
            \item \uline{\textbf{Core:}}
            \begin{itemize}
                \item basic framework for unreal modules to communicate
                \item standard set of types (math library, container library, HAL library)
            \end{itemize}
            \item \uline{\textbf{CoreUObject:}} defines UObject
            \item \uline{\textbf{Engine:}}
            \begin{itemize}
                \item contains functionality associate with a game (game world, actors, characters)
                \item Actor, Pawn, Controller, Components, Gameplay, Assets
            \end{itemize}
            \item \uline{\textbf{Toolkits:}}
            \begin{itemize}
                \item FAssetEditorToolkit: base class for toolkits that are used for asset editing
            \end{itemize}
        \end{itemize}

    \section{Unreal Types}
        \begin{itemize}
            \item ANSICHAR
            \item FVector
            \item FRotator
            \item FTransform
            \item FMatrix
            \item FArchive
            \item FOutputDevice
            \item 
            \item Text-Types:
            \begin{itemize}
                \item FString :=
                \begin{itemize}
                    \item mutable, created with \colorbox{mygray}{\lstinline{FString MyStr = TEXT("Hello, Unreal 4!")}}, lots of methods to enhance usability \href{https://docs.unrealengine.com/en-US/API/Runtime/Core/Containers/FString}{off. Docs} (16 Bytes)
                    \item FString objects store their own character arrays, while FName and FText objects store an index to a shared character array, and can establish equality based purely on this index value
                \end{itemize} 
                \item \code{FText} := meant for localization \href{https://docs.unrealengine.com/en-US/API/Runtime/Core/Internationalization/FText}{off. Docs}
                \item \code{FName} := stores strings as an identifier and thus saves memory (8 Bytes)
                \item \code{TCHAR} := ==\code{wchar_t} \href{https://docs.unrealengine.com/en-US/API/Runtime/Core/Misc/TChar}{off. Docs}
                \item \href{https://docs.unrealengine.com/4.27/en-US/ProgrammingAndScripting/ProgrammingWithCPP/UnrealArchitecture/StringHandling/}{further Information}
            \end{itemize}
            \item 
            \item Container \href{https://www.unrealengine.com/en-US/blog/ue4-libraries-you-should-know-about}{more info}
            \begin{itemize}
                \item TArray
                \begin{itemize}
                    \item dynamically resizible array
                    \item holds ptr to objects by default
                    \item can be any Type ex. \colorbox{mygray}{\lstinline{TArray<FVector>}}
                \end{itemize}
                \item TSet := like Array but unique
                \item TList
                \item TMap := key-value eg.: \colorbox{mygray}{\lstinline{TMap<FIntPoint, FPiece> Data}}
            \end{itemize}
            \item iterator $\rightarrow$ \colorbox{mygray}{\lstinline{for (auto EnemyIterator =a EnemySet.CreateIterator(); EnemyIterator; ++EnemyIterator)}}
            \item for-each $\rightarrow$ \colorbox{mygray}{\lstinline{for (AActor* OneActor : ActorArray)}} (TMap return key-value-pair)
            \item integers
            \begin{itemize}
                \item int8 / uint8
                \item int16 / uint16
                \item int32 / uint32
                \item int64 / uint64
            \end{itemize}
        \end{itemize}
            \subsection{Name vs Text vs String}:
            \begin{itemize}
                \item FName: 
                \begin{itemize}
                    \item simplest way to work with characters in a data table with fast key lookup
                    \item case insensitive
                    \item immutable
                    \item \href{https://docs.unrealengine.com/en-US/ProgrammingAndScripting/ProgrammingWithCPP/UnrealArchitecture/StringHandling/FName/index.html}{more}
                    \item \code{EName} is an enum that holds indecis to entries in the FName table
                \end{itemize}
                \item FText:
                \begin{itemize}
                    \item All user-facing text should use
                    \item supports text localization
                    \item supports formatting
                    \item generate text from dates and times
                    \item text from numbers
                    \item \href{https://docs.unrealengine.com/en-US/ProgrammingAndScripting/ProgrammingWithCPP/UnrealArchitecture/StringHandling/FText/index.html}{more}
                \end{itemize}
                \item FString:
                \begin{itemize}
                    \item only one that is mutable
                    \item $\rightarrow$ can be searched, modified and compared
                    \item \href{https://docs.unrealengine.com/en-US/ProgrammingAndScripting/ProgrammingWithCPP/UnrealArchitecture/StringHandling/FString/index.html}{more}
                \end{itemize}
                \item \href{https://docs.unrealengine.com/en-US/ProgrammingAndScripting/ProgrammingWithCPP/UnrealArchitecture/StringHandling/index.html}{Conversions}
            \end{itemize}

            \subsection{Localization}
                \begin{itemize}
                    \item Macros:
                    \begin{itemize}
                        \item \code{NSLOCTEXT} := 
                        \item \code{LOCTEXT(key, sourceString)} := localized piece of text;  namespace has to be defined with \code{#define LOCTEXT_NAMESPACE "nameSpace"}
                    \end{itemize}
                    \item String Tables can be referenced in C++ using either the LOCTABLE macro, or the static FText::FromStringTable function. The underlying logic is identical, although the macro is easier to type but will only work with literal values, whereas the function will work with both literal and variable arguments. 
                    \item 
                \end{itemize}

    \section{Classes provided by the engine}
        \subsection{UObject}
            \begin{itemize}
                \item IS THE BASE BUILDING BLOCK
                \item UClass \& UObject together are root for everything that a gameplay object does during lifetime
                \item provides:
                \begin{itemize}
                    \item reflection of properties and methods
                    \item serialization of properties
                    \item garbage collection
                    \item finding a UObject by name
                    \item configurable values for properties
                    \item Type information available at runtime 
                    \item Automatic updating of default property changes
                    \item Automatic property initialization
                    \item Automatic editor integration
                    \item networking support
                \end{itemize}
                \item a UClass singleton is created for every class that derives from UObject, containing all the metadata about the class instance
            \end{itemize}
            \uline{\textbf{Interate over all UObjects and Children}}
            \begin{lstlisting}
    // Will find ALL current UObject instances
    // or specify a different class
    for (TObjectIterator<UObject> It; It; ++It)
    {
        UObject* CurrentObject = *It;
        UE_LOG(LogTemp, Log, TEXT("Found UObject named: %s"), *CurrentObject->GetName());
    }
            \end{lstlisting}

        \subsection{AActor}
            \href{https://docs.unrealengine.com/4.27/en-US/ProgrammingAndScripting/ProgrammingWithCPP/UnrealArchitecture/Actors/}{off. Docs}
            \begin{itemize}
                \item AActor : UObject
                \item all objects that can be placed in a level derive from AActor
                \item 
            \end{itemize}

            \uline{\textbf{Iterate over all Actors and Children}}
            \begin{lstlisting}
    //Get the current World-instance with the PlayerController
    UWorld* World = MyPC->GetWorld();
    // Like object iterators, you can provide a specific class to get only objects that are
    // or derive from that class
    for (TActorIterator<AEnemy> It(World); It; ++It)
    {
        // ...
    }
            \end{lstlisting}


        \subsection{UActorComponent}
            \begin{itemize}
                \item contains the individual tasks of an actor:
                \begin{itemize}
                    \item particle effects
                    \item sounds
                    \item visual mesh
                    \item physics interaction
                    \item \code{meta=(BlueprintSpawnableComponent)} to make it possible to add it to actors
                \end{itemize}
                \item ticked as part of the owning Actors tick $\rightarrow$ call Super::Tick in the component
            \end{itemize}

        \uline{Movement and MovementComponents}
        \begin{itemize}
            \item any movement makes use of the MovementComponent
        \end{itemize}

        \uline{ProjectileMovementComponent}
            \begin{itemize}
                \item updates the position of another compoment every tick
                \item usually root component of the owning actor is moved but can be changed
            \end{itemize}

        \uline{RotatingMovementComponent}
            \begin{itemize}
                \item performs continuous rotation of a component
                \item special values like: pivot point, rotation rate
                \item no collision testing
            \end{itemize}

        \uline{CharacterMovementComponent}
            \begin{itemize}
                \item adds movement modes like walking, running, jumping, flying, falling, and swimming
                \item modes have special values to alter the behaviour like falling and walking friction, speeds for travel through air and water and across land, buoyancy, gravity scale, and the physics forces
                \item 
            \end{itemize}
        
    \section{Metadata Specifier}
        Specifies the behaviour when declaring something. \\
        \begin{itemize}
            \item UPROPERTY
            \item UFUNCTION
            \item UINTERFACE
            \item UCLASS
            \item USTRUCT
            \item UENUM
        \end{itemize}

        \subsection{UPROPERTY}
            Will include the porperty to the garbage collection and set some porperties
            \begin{itemize}
                \item Visibility: EditAnywhere, BlueprintReadWrite, BlueprintReadOnly, BlueprintCallable, 
                \item Transient := it will not be saved on disk and any value that is derived from it (eg. damagePerSecond = damage/damgeInSeconds)
                \item BlueprintImplementableEvent := lets you define the actual behaviour in the blueprint/LevelBlueprint
                \item BlueprintNativeEvent := designed to be overriden by BP but also has a default C++ implementation named '[FUNCTIONNAME]\_Implementation'
                \item Meta specifier:
                \begin{itemize}
                    \item meta = (MakeEditWidget = "true") := 
                \end{itemize}
                \item values can be edited after they are initialized with the constructor with \colorbox{mygray}{\lstinline{void AMyActor::PostInitProperties(){ }}}
                \item anything not designed for the packed game should be inside of \colorbox{mygray}{\lstinline{#if WITH_EDITOR ... #endif}}
            \end{itemize}

\smallskip
            \uline{Function that can be implemented through BPs but otherwise executes the CPP implementation}
            \begin{lstlisting}
    /** CPP-File */
    UFUNCTION(BlueprintNativeEvent)
    void SpawnOurPawn(UClass* ToSpawn, const FVector& Location);
    /** the CPP implementation needs the _IMPLEMENTATION suffix
    if BP implements the function, CPP needs to be called manually
    by calling the parent function
    */
    void ASpawnActor::SpawnOurPawn_Implementation(UClass* ToSpawn, const FVector& Location)
    /** and the PARENT function has to be called in the BP in order to execute CPP */
            \end{lstlisting}

        \subsection{UCLASS}
            \begin{itemize}
                \item Abstract: prevents from adding Actors of this class in the level
                \item Blueprintable: allow user to create Blueprints that derive from this class
                \item BlueprintType: expose the class as a type that can be used for variables in Blueprints
                \item Metadata Specifiers:
                \begin{itemize}
                    \item \colorbox{mygray}{\lstinline{[Short]ToolTip=""}} : will override the automatically generated tooltip (created from code comments)
                    \item \colorbox{mygray}{\lstinline{ChildCanTick/ChildCannotTick}}
                \end{itemize}
            \end{itemize}
            \href{https://docs.unrealengine.com/en-US/Programming/UnrealArchitecture/Reference/Classes/Specifiers/index.html}{Complete list}


        \subsection{UENUM}
            \begin{enumerate}
                \item BlueprintType: makes the enum available in Blueprints
                \item Meta: can be used to add metadata to the enum
                \item DisplayName: changes the name of the enum in the editor
                \item ToolTip: adds a tooltip to the enum
                \item switch case: each case must be like \code{case EnumName::CaseName}
            \end{enumerate}

    \section{Usefull functions\&Variables for classes}
        \subsection{Pawn}
            \uline{Variables}
            \begin{itemize}
                \item AutoPossesPlayer
                \item AutoPossesAI
                \item bUseControllerRotationPitch, bUseControllerRotationYaw, bUseControllerRotationRoll
                \item Controller
                \item LastHitBy
            \end{itemize}
\smallskip
            \uline{Functions}
            \begin{itemize}
                \item GetVelocity()
                \item AddMovementInput()
                \item ConsumeMovementInputVector()
                \item CreatePlayerInputComponent()
                \item DestroyPlayerInputComponent()
                \item GetBaseAimRotation
                \item GetController
            \end{itemize}
\smallskip

\smallskip
    
\smallskip

    \section{Asserts}
        \href{https://dev.epicgames.com/documentation/en-us/unreal-engine/asserts-in-unreal-engine?application_version=5.3}{off. docs}
        \begin{table}[!htb]
            \begin{tblr}{p{6cm} | p{12cm}}
                \hline
                    function & description \\
                \hline
                    \SetCell[c=2]{l} \makecell[l]{CHECK := in Debug, Development, Test, and Shipping Editor builds, \\ except those ending in "Slow", which only operate in Debug builds \\ Defining USE\_CHECKS\_IN\_SHIPPING makes Check macros operate in all builds.} \\
                    check | checkslow & \makecell[l]{Halts execution if Expression is false} \\
                    checkf | checkfslow & \makecell[l]{Halts execution if Expression is false and prints the message} \\
                    checkcode & \makecell[l]{Executes Code within a do-while loop structure that runs \\ useful to prepare information that another Check requires} \\
                    checkNoEntry & \makecell[l]{Halts execution if the line is ever hit, similar to check(false),\\ but intended for code paths that should be unreachable} \\
                    checkNoReentry & \makecell[l]{Halts execution if the line is hit more than once} \\
                    checkNoRecursion & \makecell[l]{Halts execution if the line is hit more than once\\ on the same call stack (without leaving scope)} \\
                    unimplemented & \makecell[l]{Halts execution if the line is ever hit, similar to check(false),\\ but intended for virtual functions that should be overridden and not called} \\
                    \SetCell[c=2]{l} \makecell[l]{VERIFY := Verify macros evaluate their expressions even in builds where Check macros are disabled} \\
                    verify & \makecell[l]{Halts execution if Expression is false in debug builds} \\
                    \SetCell[c=2]{l} \makecell[l]{ENSURE := works with non-fatal errors} \\
                    ensure & \makecell[l]{Halts execution if Expression is false in debug builds} \\
                \hline
            \end{tblr}
        \caption{ caption }  
        \end{table}
        

    \section{Components}
        \uline{Major classes:}
        \begin{itemize}
            \item Actor Components/UActorComponent: for abstract behavior like movement, inventory, attribute managment ...
            \item Scene Components/USceneComponent (child of UActorComponent): supports location-based behavior wihtout a geometric representation.
            \item Primitive Components/UPrimitiveComponent (child of USceneComponent):  location-based behavior with geometric representation $\rightarrow$ static/skeletal mesh, sprites/billboards, particle systems ...
        \end{itemize}
        \uline{Examples}
        \begin{itemize}
            \item UCameraComponent
            \item USpringArmComponent
            \item UBoxComponent
            \item UStaticMeshComponent
            \item ...
        \end{itemize}
        \uline{No ticking unless:}
        \begin{itemize}
            \item 1. set \code{PrimaryComponentTick.bCanEverTick = true}
            \item 2. call \code{PrimaryComponentTick.SetTickFunctionEnable(true)}
            \item (3. disable it) \code{PrimaryComponentTick.SetTickFunctionEnable(false)}
        \end{itemize}

    \section{Adding additional things into automatically added header files}
        \begin{itemize}
            \item open: \code{[ProjectName].Build.cs}
            \item public dependency names in \code{PublicDependenyModuleNames.AddRange} are \code{\"Core\", \"CoreUObject\", \"Engine\", \"InputCore\"}
            \item and to 'PrivateDependencyModuleNames.AddRange' to be only useable in current module ie.: \code{\"Slate\", \"SlateCore\"}
        \end{itemize}

\bigskip

    \section{Gameplay tags}
        \begin{itemize}
            \item are used for nested functionallity like DamgeTypes, 
            \item can be set in \code{GameName/Config/DefaultGameplayTags.ini}
            \item can be set in \code{GameName/Config/Tags}
            \item can be set through DataTables of the type GameplayTagTableRow
            \item 
            \item are added to blueprints or native types as GameplayTag or GameplayTagContainer
            \item 
            \item Container.HasTagExact() returns:
            \begin{itemize}
                \item “Weapon.Melee”
                \item ”Spell.Fireball” 
            \end{itemize}
            \item Container.HasTag() return:
            \begin{itemize}
                \item “Weapon”
                \item “Weapon.Melee” 
                \item “Spell” 
                \item “Spell.Fireball”
            \end{itemize} 
        \end{itemize}

    \section{Memory Managed Object instances(NewObject SpawnActor)}
        Creating an UObject instance is done using
        \begin{lstlisting}
    NewObject< >
        \end{lstlisting}
        Creating an UActor instance is done using
        \begin{lstlisting}
    SpawnActor< >
        \end{lstlisting}

    \section{Subsystems}
        \glsdesc{Subsystem}
        Is a way to extend Engine-Classes while avoiding to override them \\
        \uline{Currently supported subsystems}
        \begin{itemize}
            \item UEngineSubsystem: Access through \code{UMyEngineSubsystem* MySubsystem = GEngine->GetEngineSubsystem<UMyEngineSubsystem>();}
            \item \code{UEditorSubsystem} : Access through \code{UMyEditorSubsystem* MySubsystem = GEditor->GetEditorSubsystem<UMyEditorSubsystem>();}
            \item \code{UGameInstanceSubsystem} Access through \code{UGameInstance* GameInstance = ...; UMyGameSubsystem* MySubsystem = GameInstance->GetSubsystem<UMyGameSubsystem>();}
            \item \code{ULocalPlayerSubsystem} Access through \code{ULocalPlayer* LocalPlayer = ...; UMyPlayerSubsystem * MySubsystem = LocalPlayer->GetSubsystem<UMyPlayerSubsystem>();}
            \item UWorldSubsystem
        \end{itemize}
        \uline{Neccessary code for subsystems}
        \begin{lstlisting}
public:
    // Begin USubsystem
    virtual void Initialize(FSubsystemCollectionBase& Collection) override;
    virtual void Deinitialize() override;
    // End USubsystem
        \end{lstlisting}

    \section{Gameplay Timers}
        \begin{itemize}
            \item Schedule actions to perform
            \item timers are managed in a global Timer Manager
            \item Global Timer Manager: exists in Game Instance and in every WorldSettings
        \end{itemize}
        \begin{itemize}
            \item create a \code{FTimerHandle} in the class you want to use a timer in
            \item set timer with the GlobalTimerManager
        \end{itemize}
        \uline{Simple Timer that calls a function}
        \begin{lstlisting}
    FTimerHandle TH_PrimaryAttack;
    GetWorldTimerManager().SetTimer(TH_PrimaryAttack, this, &AVCharacter::FireProjectile, 0.2f);
        \end{lstlisting}

        \textbf{\uline{Timer that takes in parameters}}
        \begin{lstlisting}
    /* .h */
    FTimerDelegate TimerDel;
	FTimerHandle ShootHandle;
    
    /* .cpp */
    TimerDel.BindUFunction(this, FName("Shoot"), TargetCharacter);
	GetWorldTimerManager().SetTimer(ShootHandle, TimerDel, AttackSpeed, true);
        \end{lstlisting}

    \section{Delegates}
        \begin{table}[!htb]
            \begin{tblr}{p{6cm} | p{12cm}}
                \hline
                    Function signature & Declaration macro \\
                \hline
                    \code{void Function()} & \makecell{\code{DECLARE_DELEGATE(DelegateName)}} \\
                    \code{void Function(Param1)} & \makecell{\code{DECLARE_DELEGATE_OneParam(DelegateName, Param1Type)}} \\
                    \code{void Function(Param1, Param2)} & \makecell{\code{DECLARE_DELEGATE_TwoParams(DelegateName, Param1Type, Param2Type)}} \\
                    \code{void Function(Param1, Param2, ...)} & \makecell{\code{DECLARE_DELEGATE_[PARAM\#](DelegateName, Param1Type, Param2Type, ...)}} \\
                    \code{[RET_VALUE] Function()} & \makecell{\code{DECLARE_DELEGATE_RetVal(RetValType, DelegateName)}} \\
                    \code{[RET_VALUE] Function(Param1)} & \makecell{\code{DECLARE_DELEGATE_RetVal_OneParam(RetValType, DelegateName, Param1Type)}} \\
                    \code{[RET_VALUE] Function(Param1, Param2)} & \makecell{\code{DECLARE_DELEGATE_RetVal_TwoParams(RetValType, DelegateName, Param1Type, Param2Type)}} \\
                    {\makecell{[RET\_VALUE] \\ Function(Param1, Param2 ...)}} & \makecell{\code{DECLARE\_DELEGATE\_RetVal\_[PARAM\#](RetValType, DelegateName, Param1Type, Param2Type, ...)}} \\
                \hline
            \end{tblr}
        \end{table}

         \includegraphics[width=\textwidth]{DelegateTypes.png}

    \section{Create custom Config files}
        \begin{itemize}
            \item add UCLASS specifiers:
            \begin{itemize}
                \item 'Config=ConfigName' $\rightarrow$ UPROPERTY values can be set from ini file
                \item 'PerObjectConfig' $\rightarrow$ if every instance should write to .ini file 
                \item 'UPROPERTY(Config)' $\rightarrow$ property value can be set from ini file
            \end{itemize}
            \item Configuration Categories:
            \begin{itemize}
                \item Compat
                \item DeviceProfiles
                \item Editor
                \item EditorGameAgnostic
                \item EditorKeyBindings
                \item EditorUserSettings
                \item Engine
                \item Game
                \item Input
                \item Lightmass
                \item Scalability
            \end{itemize}
        \end{itemize}

    \section{Interfaces}
        \begin{table}[H]
            \begin{tblr}{p{6cm} | p{12cm}}
                \hline
                    UFUNCTION type & Description \\
                \hline
                    \textbf{NO} UFUNCTION &
                    C++ only $\rightarrow$ can have \textbf{default implementation}

                    BlueprintCallable &
                    Functions using the BlueprintCallable specifier can be called in C++ or Blueprint using
                    a reference to an object that implements the interface \\

                    BlueprintImplementableEvent &
                    can not be overridden in C++, but can be overridden in any Blueprint class that implements
                    or inherits your interface. \\
                    
                    BlueprintNativeEvent &
                    Functions using BlueprintNativeEvent can be implemented in C++ by overriding a function
                    with the same name, but with the suffix \_Implementation added to the end. \\
            \end{tblr}
        \caption{Types of UFUNCTIONs for Interfaces}
        \end{table}

        \subsection{Basic setup}
            \begin{itemize}
                \item check if the actor implements the interface \code{->Implements<UInterfaceName>()}
                \item cast the actor to the interface \code{Cast<IINTERFACE_NAME>(Actor)}
                \item call function from the interface \code{Execute_InterfaceFunction(ActorImplementingTheInterface)}
            \end{itemize}
            \uline{Example implementation}
            \begin{lstlisting}
void UBaseAttributeSet::PostGameplayEffectExecute(const FGameplayEffectModCallbackData& Data)
{
	if (Data.Target.GetAvatarActor()->Implements<UIDeath>())
	{
		if (Data.EvaluatedData.Attribute.GetNumericValue(this) <= 0)
		{
			Cast<IIDeath>(Data.Target.GetAvatarActor())->Execute_Died(Data.Target.GetAvatarActor());
		}
	}
}
        \end{lstlisting}

        \subsection{Making pointers an output}
        \begin{itemize}
            \item make the normal pointer a \code{Pointer reference}
        \end{itemize}
        \begin{lstlisting}
UFUNCTION(BlueprintImplementableEvent)
void GetHintAndIcon(FString& Text, UTexture2D*& Texture);
        \end{lstlisting}


    \section{JSON}
        \begin{itemize}
            \item 
        \end{itemize}
        \begin{lstlisting}
    FString Out;
	FString Path = FPaths::ProjectContentDir();
	Path.Append("TestJson.json");
	FJsonObjectConverter::UStructToJsonObjectString( Struct, Out, 0, 0);
	FFileHelper::SaveStringToFile(Out, *Path);
        \end{lstlisting}    

    \section{Usefull C++ Code}
        \uline{Moving a Pawn}
            \begin{itemize}
            \item Assign 'Action/Axis Mappings' in 'Project Settings'
                \item include Components/InputComponent.h
                \item Function map Functions to Keys in the SetupPlayerInputComponent 
                \item declare and define the Functions
            \end{itemize}
            \uline{Example:}
\begin{lstlisting}
void AMainCharacter::SetupPlayerInputComponent(UInputComponent* PlayerInputComponent)
{
    Super::SetupPlayerInputComponent(PlayerInputComponent);

    UEnhancedInputComponent* EnhancedInputComponent = Cast<UEnhancedInputComponent>(PlayerInputComponent);

    if (ensureAlwaysMsgf(EnhancedInputComponent != nullptr, TEXT("**********\nNeed to change to enhanced input\n**********")))
    {
        EnhancedInputComponent->BindAction(IA_Move, ETriggerEvent::Triggered, this, &AVCharacter::Move);
        EnhancedInputComponent->BindAction(IA_Look, ETriggerEvent::Triggered, this, &AVCharacter::Look);
        EnhancedInputComponent->BindAction(IA_Interact, ETriggerEvent::Triggered, this, &AVCharacter::Interact);
    }
}           
\end{lstlisting}

        \uline{Setup the Camera}
            \begin{itemize}
                \item declare a camera-component in the header-file
                \item define it in the CPP-file
                \item attach to root
                \item set LocRot
                \item 
                \item ADVANCED:
                \begin{itemize}
                    \item declare springarm-component in the header-file
                    \item define it in the CPP-file
                    \item setup attachment to root
                    \item set targetarmlength
                    \item set bUsePawnControlRotation to true
                \end{itemize}
            \end{itemize}

            \uline{Example:}
            \begin{lstlisting}
    FollowCamera = CreateDefaultSubobject<UCameraComponent>(TEXT("FollowCamera"));
    FollowCamera->SetupAttachment(CameraBoom, USpringArmComponent::SocketName);
    FollowCamera->bUsePawnControlRotation = false;
            \end{lstlisting}

            \uline{Advanced: added springarmcomponent}
            \begin{lstlisting}
                CameraBoom = CreateDefaultSubobject<USpringArmComponent>(TEXT("CameraBoom"));
                CameraBoom->SetupAttachment(GetRootComponent());
                CameraBoom->TargetArmLength = 600.f;
                CameraBoom->bUsePawnControlRotation = true;
            \end{lstlisting}
        \uline{Add addtional navigation keys in UI (add this in the GameInstance or PlayerController)}
\begin{lstlisting}
FSlateApplication::Get().GetNavigationConfig()->KeyEventRules.Add(EKeys::W, EUINavigation::Up);
FSlateApplication::Get().GetNavigationConfig()->KeyEventRules.Add(EKeys::A, EUINavigation::Left);
FSlateApplication::Get().GetNavigationConfig()->KeyEventRules.Add(EKeys::S, EUINavigation::Down);
FSlateApplication::Get().GetNavigationConfig()->KeyEventRules.Add(EKeys::D, EUINavigation::Right);
\end{lstlisting}

        \uline{Push a StaticMesh Component}
\begin{lstlisting}
StaticMesh->AddForce([FVECTOR])
\end{lstlisting}
\smallskip
        \uline{create a random number}
        \begin{lstlisting}
    FMath::FRand();
    FMath::FRandRange([START],[END]);
        \end{lstlisting}
\smallskip
        \uline{Clamp values}
        \begin{lstlisting}
    FMath::Clamp([INPUTVARIABLENAME], [MINVALUE], [MAXVALUE]);
        \end{lstlisting}
\smallskip
        \uline{Camera Setup for the player}
        \begin{lstlisting}
    FRotator NewRotation = GetActorRotation();
    NewRotation.Yaw += CameraInput.X;
    SetActorRotation(NewRotation);

    FRotator NewSpringArmRotation = SpringArm->GetComponentRotation();
    NewSpringArmRotation.Pitch = FMath::Clamp(NewSpringArmRotation.Pitch += CameraInput.Y, -80.f, -15.f);
    SpringArm->SetWorldRotation(NewSpringArmRotation);
        \end{lstlisting}
\smallskip
        \begin{lstlisting}
    void AMyActor::PostInitProperties()
    {
        Super::PostInitProperties();
        DamagePerSecond = TotalDamage / TimeInSeconds;
    }
        \end{lstlisting}
\smallskip
        \uline{Add a delegate}
\begin{lstlisting}
DECLARE_DYNAMIC_MULTICAST_DELEGATE(FOnInventoryUpdated);

UPROPERTY(BlueprintAssignable) // assignable in BPs (example AnimBP)
FOnInventoryUpdated OnInventoryUpdated; //Add the actual delegate

// IN THE CPP FILE
OnInventoryUpdated.Broadcast();
\end{lstlisting}

        \uline{Save to file:}
\begin{lstlisting}
bool result = FFileHelper::SaveArrayToFile(ToBinary, TEXT(SAVEDATAFILENAME))
\end{lstlisting}
        \uline{Setup TriggerBox with collision channels:}
\begin{lstlisting}
ATriggerBox::ATriggerBox()
{
    TriggerBox->SetCollisionEnabled(ECollisionEnabled::QueryOnly);
    TriggerBox->SetCollisionObjectType(ECollisionChannel::ECC_WorldStatic);
    TriggerBox->SetCollisionResponseToAllChannels(ECollisionResponse::ECR_Ignore);
    TriggerBox->SetCollisionResponseToChannel(ECollisionChannel::ECC_Pawn, ECollisionResponse::ECR_Overlap);

    TriggerBox->SetBoxExtent(FVector(62.f, 62.f, 32.f));
}
ATriggerBox::BeginPlay()
{
    TriggerBox->OnComponentBeginOverlap.AddDynamic(this, &AFloorSwitch::OnOverlapBegin);
    TriggerBox->OnComponentEndOverlap.AddDynamic(this, &AFloorSwitch::OnOverlapEnd);
}
\end{lstlisting}

    \uline{Print Debug Message}
\begin{lstlisting}
UE_LOGFMT(LogTemp, Display, "Text {0}", Variable);
\end{lstlisting}

    \uline{Logging}
    \begin{lstlisting}
UE_LOG(LogTemp, Warning, TEXT("My Text"))
    \end{lstlisting}
    \uline{LogTypes:}
    \begin{itemize}
        \item Fatal: printed to console and files even if logging is disabled
        \item Error: to console and logs (red)
        \item Warning: to conosle and files (yellow)
        \item Display: to console and files
        \item Log: NOT to console but files
        \item Verbose:
    \end{itemize}
    \uline{Define custom category}
    \begin{lstlisting}
    DEFINE_LOG_CATEGORY(CategoryName);
    \end{lstlisting}
    
\uline{BeginPlay equivalent for animations:}
\begin{lstlisting}
void NativeIntializeAnimation() override;
\end{lstlisting}

\uline{Get the World from viewport}
\begin{lstlisting}
FWorldContext* world = GEngine->GetWorldContextFromGameViewport(GEngine->GameViewport);
world->World();
\end{lstlisting}

\uline{Set the CPP-Standard}
\begin{lstlisting}
CppStandard = CppStandardVersion.Cpp20;
\end{lstlisting}


\uline{Marker}
\begin{lstlisting}
FVector MarkerWorldPosition = ObjectiveMarker.GetWorldLocation();

int32 ViewportWidth;
int32 ViewportHeight;
GetOwningPlayer()->GetViewportSize(ViewportWidth, ViewportHeight);
FVector2D ViewportSize(ViewportWidth, ViewportHeight);

FVector2D ScreenPosition;
bool bHasScreenPos = false;

ULocalPlayer* LP = GetOwningPlayer()->GetLocalPlayer();
if (LP && LP->ViewportClient)
{
    // TODO: Should probably cache this per-frame instead of having each widget calculate this for itself.
    FSceneViewProjectionData ProjectionData;
    if (LP->GetProjectionData(LP->ViewportClient->Viewport, ProjectionData))
    {
        FMatrix const ViewProjectionMatrix = ProjectionData.ComputeViewProjectionMatrix();
        FPlane Result = ViewProjectionMatrix.TransformFVector4(FVector4(MarkerWorldPosition, 1.f));
        FIntRect ViewRect = ProjectionData.GetViewRect();

        // the result of this will be x and y coords in -1..1 projection space
        // TODO: This is very hacky - in theory we should only be using Result.W here (see ProjectWorldToScreen) but this seems to give good results.
        // If we just use `Result.W` we get bad values with W < 0 (which ProjectWorldToScreen outright ignores). `FMath::Abs(Result.W)` gives good results
        // but only when W isn't close to 0 (-300 < W < 300 all seems to give bad results with the position ending up in a far corner of the screen).
        // Setting a minimum value of W based on the viewport height seems to give reasonable results, with the hackyness only being noticeable when watching carefully.
        const float RHW = 1.0f / FMath::Max(FMath::Abs(Result.W), ViewRect.Height() / 2.f);
        FPlane PosInScreenSpace = FPlane(Result.X * RHW, Result.Y * RHW, Result.Z * RHW, Result.W);

        // Move from projection space to normalized 0..1 UI space
        const float NormalizedX = ( PosInScreenSpace.X / 2.f ) + 0.5f;
        const float NormalizedY = 1.f - ( PosInScreenSpace.Y / 2.f ) - 0.5f;

        FVector2D RayStartViewRectSpace(
            ( NormalizedX * (float)ViewRect.Width() ),
            ( NormalizedY * (float)ViewRect.Height() )
            );

        ScreenPosition = RayStartViewRectSpace + FVector2D(static_cast<float>(ViewRect.Min.X), static_cast<float>(ViewRect.Min.Y));
        ScreenPosition -= FVector2D(ProjectionData.GetConstrainedViewRect().Min);

        bHasScreenPos = true;
    }
}
\end{lstlisting}

\uline{Iterate TMap}
\begin{lstlisting}
TMap<int32, AActor*> exampleIntegerToActorMap;
for (const TPair<int32, AActor* >& pair : exampleIntegerToActorMap)
{
    pair.Key;
    pair.Value;
}
\end{lstlisting}
    
\uline{FilterByPredicate}
\begin{lstlisting}
TArray<FVector> Things { FVector::ZeroVector, { 1 } };
FVector* FoundXPositiveThing = Things.FindByPredicate([](const FVector& Item) { return Item.X > 0.f; });    
\end{lstlisting}

\uline{Save package}
\begin{lstlisting}
//FAssetRegistryModule::AssetCreated(NewWorld);
//NewWorld->MarkPackageDirty();
//FString FilePath = FString::Printf(TEXT("%s%s%s"), *ContentDirPath, *PackageName.RightChop(1), *FPackageName::GetAssetPackageExtension());
//FSavePackageArgs SaveArgs;
//SaveArgs.TopLevelFlags = RF_Public | RF_Standalone;
//bool bSuccess = UPackage::SavePackage(MyPackage, NewWorld, *FilePath, SaveArgs);
\end{lstlisting}

\uline{Iterate over all UFUNCTIONs}
\begin{lstlisting}
TFieldIterator<UFunction>

for ( TFieldIterator<UFunction> FuncIt(FromClass, EFieldIteratorFlags::IncludeSuper); FuncIt; ++FuncIt )



USTRUCT(BlueprintType)
struct FCppPostStruct
{
GENERATED_BODY()

UPROPERTY()
int32 userId;

UPROPERTY()
int32 id;

UPROPERTY()
FString title;

UPROPERTY()
FString body;
};

FCppPostStruct* test = new FCppPostStruct();
test->id = 0;
test->userId = 0;
test->title = "hello";
test->body = "world";

for (TFieldIterator<UProperty> It(test->StaticStruct()); It; ++It)
{
UProperty* Property = *It;
FString VariableName = Property->GetName();
}

\end{lstlisting}

\uline{Notes}
\begin{itemize}
\item Or often TDelegate<void()> (void is the return type, and you can specify the params in the parentheses)
\item Multicast just means it holds an array of delegates, multiple delegates bound to different functions and objects
\item Dynamic are also quite a bit slower
\end{itemize}


\begin{lstlisting}
UClass* Class = GetClass();
const bool bIsBlueprint = Class->bGeneratedBy;
if (!bIsBlueprint) return;
for (TFieldIterator<FProperty> PropertyIterator(this->GetClass()); PropertyIterator; ++PropertyIterator)
{
}
\end{lstlisting}
use Property->ContainerPtrToValuePtr, pass the object instace to its params
UClass is shared between same archetypes
but ContainerPtrToValuePtr gives you the correct variable from the Uobject passed to it


\begin{lstlisting}
GameplayTagEditorModule::AddNewGameplayTagToINI <.>
\end{lstlisting}

\uline{Possess on server}
\begin{lstlisting}
PossessedBy() // on server
APlayerController::AcknowledgePossession() // on client
OnRep::Controller // any
APawn::Restart // IDK
\end{lstlisting}


\uline{Possess on Client}
\begin{lstlisting}
\end{lstlisting}


\uline{Force Garbage Collection}
\begin{lstlisting}
GetWorld()->ForceGarbageCollection( true );
\end{lstlisting}

\uline{CreateDefaultSubobject vs NewObject}
    \begin{itemize}
        \item \code{CreateDefaultSubobject}
        \begin{itemize}
            \item can ONLY be called in constructor
        \end{itemize}
    \end{itemize}
    \begin{itemize}
        \item \code{NewObject}
        \begin{itemize}
            \item can be called outside constructor
        \end{itemize}
    \end{itemize}

\uline{TSoftObjPtr TWeakObjPtr}
\begin{itemize}
    \item \code{TSoftObjPtr}
    \begin{itemize}
        \item if you want to control when something is loaded
        \item you can decide when to load
        \item == TWeakObjPtr with a path
    \end{itemize}
\end{itemize}

\begin{itemize}
    \item \code{TWeakObjPtr}
    \begin{itemize}
        \item if you want to reference something and not keep it alive for GC
    \end{itemize}
\end{itemize}

\uline{OnPossess before BeginPlay}
    \begin{itemize}
        \item \code{OnPossess} will be executed even before the \code{BeginPlay}
    \end{itemize}


    

    \subsection{Developer settings}
        \begin{itemize}
            \item set:
            \begin{itemize}
                \item \code{CategoryName} (Editor, Game )
                \item \code{SectionName}
            \end{itemize}
        \end{itemize}

        \begin{figure}[H]
            \includegraphics[width=\textwidth]{DeveloperSettings.png}
            \caption{DeveloperSettings}
            \label{}
        \end{figure}


    \subsection{Check for MetaData in UCLASS UPROPERTY}
        \uline{is editor only}
        \begin{lstlisting}
    Property->HasMetaData(TEXT("MyMetaTag"))
        \end{lstlisting}


    \section{Class Notes}
        \subsection{ULocalPlayer}
            \begin{itemize}
                \item  Each player that is active on the current client has a LocalPlayer. It stays active across maps
                \item  There may be several spawned in the case of splitscreen/coop.
                \item  There may be 0 spawned on servers.
                \item 
            \end{itemize}


        \subsection{UCommonUIActionRouterBase}
            \begin{itemize}
                \item The nucleus of the CommonUI input routing system
                \item @todo DanH: Explain what that means more fully 
                \item FlushInput
            \end{itemize}

        \subsection{UInputComponent}

        \subsection{IInputProcessor}
            \begin{itemize}
                \item Interface for a Slate Input Handler
                \item 
            \end{itemize}
             \includegraphics[width=\textwidth]{Bilder/IInputProcessor.jpg}

        \subsection{FCommonInputPreprocessor}
            \begin{itemize}
                \item Helper class that is designed to fire before any UI has a chance to process input so that we can properly set the current input type of the application.
                \item 
            \end{itemize}

        \subsection{FEnhancedInputEditorProcessor}
            \begin{itemize}
                \item used to pass InputKey events to the Enhanced Input Editor Subsystem
                \item will not steal input, and all the functions here will return "False"
                \item so that other Input Processors still run with all their normal considerations
            \end{itemize}

        
        \subsection{UIActionRouterTypes}
            \begin{itemize}
                \item // Note: Everything in here should be considered completely private to each other and CommonUIActionRouter.
                //They were all originally defined directly in CommonUIActionRouter.cpp, but it was annoying having to scroll around so much.
            \end{itemize}



\chapter{Networking}
    Networking in UE4 is called 'Replication'

    \section{Basics}
        \begin{itemize}
            \item is Actor-centric
            \item how to create 3 instances and connect them to the server (local debugging) :NEXT LINE:
            \item Editor Preferences $\rightarrow$ type 'Multi' $\rightarrow$ tick 'Enable Network Emulation'
            \item 
            \item replication goes only SERVER $\rightarrow$ CLIENT but CLIENT $\nrightarrow$ SERVER
            \item 
            \item set \colorbox{mygray}{\lstinline{NetUpdateFrequency}} to an appropriate value ex.: 10 = $\frac{1}{5}$ $\rightarrow$ update every 5th of a second (0.2s)
            \item \colorbox{mygray}{\lstinline{net.UseAdaptiveNetUpdateFrequency}} combined with \colorbox{mygray}{\lstinline{MinNetUpdateFrequency}}(minimum update attempts) and \colorbox{mygray}{\lstinline{NetUpdateFrequency}}(maximum update attempts) can improve perfermonce
            \item 
            \item clients must reconnect to the server for some actions, this is called Non-seamless-travel and occurs when:
            \begin{itemize}
                \item loading a map for the first time
                \item connecting to a server for the first time as a client
                \item you want to end a multiplayer game and start a new one
            \end{itemize}
        \end{itemize}

        \includegraphics[width=\textwidth]{DomainExample.png}
        \uline{}
        \begin{itemize}
            \item Server Only: Exist on the server (ex.: AGameMode)
            \item Server \& Client: Exist on the Server and on all Clients (ex.: AGameState, APlayerState, APawn )
            \item Server \& Owning Client: Exist on the Server and the owning Client (ex.: APlayerController)
            \item Owning Client Only: Exist only on the Client (ex.: AHUD, UMG Widgets)
        \end{itemize}
    \begin{itemize}
        \item \uline{Network Modes:} is a property of the world and can have the following values:
        \begin{itemize}
            \item NM\_Standalone := for singleplayer and local-multiplayer
            \item NM\_Client := runs the game without any server-logic
            \item NM\_ListenServer := runs the game and accepts connections
            \item NM\_DedicatedServer := hosts a network multiplayer session without graphics
        \end{itemize}
        \item \uline{Replication-Features:}
        \begin{itemize}
            \item Creation \& Destruction := when an authoritative version of a repicated Actor is spawned on a server, it automatically generates remote proxis
            \item Movement-Replication := bReplicateMovement = true and the Actor will replicate \textit{Location,Rotation,Velocity}
            \item Variable-Replication := any variable that is 
            \item Component-Replication := 
            \item Remote-Procedure-Calls := 
        \end{itemize}
        \item \uline{Network Roles:}
        \begin{itemize}
            \item None := the actor has no role in a network game and does not replicate
            \item Authority := the actor is authoritativ and replicates its information to remote proxies of it on other machines
            \item Simulated Proxy (this are most actors) := the actor is a remote proxy that is controlled entirely by an authoritative actor on another machine
            \item Autonomous Proxy (ex. player character) := the actor is a remote proxy that is capable of performing some functions locally, but receivs corrections from an authoritative actor
        \end{itemize}
    \end{itemize}
\includegraphics[width=\textwidth]{NetworkLayout.jpg}

        \uline{GameState}
        \begin{itemize}
            \item AGameStateBase \& AGameState
            \item replicated to all Clients
            \item most important class for shared information between Server \& Clients
            \item keeps track of the Game State ex.: GameMode contains total KillCount GameState the Kills for every player
            \item 
        \end{itemize}

        \uline{PlayerState}
        \begin{itemize}
            \item most important class for a specific Player
            \item each Player has a PlayerState
            \item access through PlayerArray inside of GameState
            \item 
            \item good for ex.:
            \begin{itemize}
                \item PlayerName
                \item Score
                \item Ping
                \item GuildID
            \end{itemize}
        \end{itemize}

        \uline{Player Controller:}
        \begin{itemize}
            \item exists on the Client \& Server
            \item Clients don't know of each other PlayerControllers
            \item GerPlayerController(0):
            \begin{itemize}
                \item Listen-Server: gets the Listen-Servers PlayerController
                \item Client: gets the Clients PlayerController
                \item Dedicated-Server: gets the first Clients \
            \end{itemize}
        \end{itemize}

        \uline{Userful Classes:}
        \begin{itemize}
            \item AInfo := is an actor used to store information (is an actor in order to use replication)
            \item 
        \end{itemize}

        \uline{RPCs invoked from Server}
        \begin{table}[!htb]
            \begin{tabular}{|p{3cm}|p{3cm}|p{3cm}|p{3cm}|p{3cm}|}
                \hline
                    Actor Ownership & Not replicated & NetMulticast & Server & Client \\
                \hline
                    Client-owned Actor & Runs on Server & Runs on Server and all Clients & Runs on Server & Runs on Actors owning Client \\
                    Server-owned Actor & Runs on Server & Runs on Server and all Clients & Runs on Server & Runs on Server \\
                    Unowned Actor & Runs on Server & Runs on Server and all Clients & Runs on Server & Runs on Server \\
                \hline
            \end{tabular}
        \caption{ caption }  
        \end{table}
        
        \uline{RPCs invoked from Server}
        \begin{table}[!htb]
            \begin{tabular}{|p{3cm}|p{3cm}|p{3cm}|p{3cm}|p{3cm}|}
                \hline
                    Actor Ownership & Not replicated & NetMulticast & Server & Client \\
                \hline
                    Owned by invoking Client & Runs on invoking Client & Runs on invoked Client & Runs on Server & Runs on invoking Client \\
                    Owned by a different Client & Runs on invoking Client & Runs on invoked Client & Dropped & Runs on invoking Client \\
                    Server owned actor & Runs on invoking Client & Runs on invoked Client & Dropped & Runs on invoking Client \\
                    Unowned Actor & Runs on invoking Client & Runs on invoked Client & Dropped & Runs on invoking Client \\
                \hline
            \end{tabular}
        \caption{ caption }  
        \end{table}
        \begin{lstlisting}
    UFUNCTION(Client, unreliable)
    void ClientRPCFunction();

    UFUNCTION(Client, reliable)
    void ReliableClientRPCFunction();

    UFUNCTION(NetMulticast, unreliable)
    void MulticastRPCFunction();

        \end{lstlisting}


        
        \subsection{Network prediction and reconciliation}
        \begin{itemize}
            \item prediction: player's next moves are tried to be predicted
            \item reconciliation: corrects errors that occured during prediction
        \end{itemize}
        
        \subsection{Bandwidth Optimization}
        \begin{itemize}
            \item network relevancy: 
            \item replication conditions: 
        \end{itemize}
        
        
        \subsection{Workflow}
            \begin{itemize}
                \item 
            \end{itemize}


    \section{IRIS}
        \textbf{\uline{Is the new networking plugin}}
        \begin{itemize}
            \item a new API to improve networking with paralization and decoupling gameplay code and network code
        \end{itemize}

        \textbf{\uline{Main parts}}
        \begin{itemize}
            \item \textbf{\uline{Replication Bridge}}
            \begin{itemize}
                \item begins/end replication for actor or Objects
                \item builds descriptors and protocols for replicated data
            \end{itemize}
            \item \textbf{\uline{Net Object}}
            \begin{itemize}
                \item internal representation of an actor/object consisting of
                \item \uline{Replication protocol}
                \item \uline{replication instance protocol}
                \item \uline{buffer to store quantized data}
            \end{itemize}
            \item \textbf{\uline{Replication protocol}}
            \begin{itemize}
                \item per object type and shared between all instances of the same type
                \item contains all replication state descriptors $\rightarrow$ net object can be restored with
            \end{itemize}
            \textbf{\uline{replication fragment}}
            \begin{itemize}
                \item component responsible for carrying replication states back-and-forth between gameplay code and replication system
            \end{itemize}
            \item \textbf{\uline{Net handle}}
            \begin{itemize}
                \item object that API-functions operate on
                \item \uline{handle of} an actor
                \item \uline{handle for} replication system
                \item \uline{generated} by the replication system
                \item \uline{returned} when \code{BeginReplication} is called on an actor
            \end{itemize}
        \end{itemize}


        \textbf{\uline{replication state + replication state descriptor}}
        \begin{itemize}
            \item \textbf{\uline{replication state}}
            \begin{itemize}
                \item bridge between replication system and gameplay code
                \item basic form is a struct containing data to be replicated
            \end{itemize}
            \item 
            \item \textbf{replication state descriptor}
            \begin{itemize}
                \item memory layout
                \item conditionals
                \item filtering
                \item prioritization
                \item serialization
            \end{itemize}
        \end{itemize}


        \subsection{workflow and dataflow}
            \begin{itemize}
                \item \textbf{Registration:}
                \begin{itemize}
                    \item \uline{manually}
                    \item register object with replication system
                    \item declare replication state and replication state descriptor (header properties / explicitly %TODO how do yuo do that?)
                    \item 
                    \item \uline{automatically}
                    \item constructs a replication protocol and replication instance protocol using all replication states defined by the object and its components
                    \item creates a net object corresponding to the newly registered object using the previously constructed replication protocol and replication instance protocol
                    \item assigns the replicated object a unique net handle
                \end{itemize}
            \end{itemize}
            
            \begin{itemize}
                \item \textbf{\uline{Sender}}
                \begin{itemize}
                    \item pre-send-update: polls all replicated objects for state changes if running in legacy mode
                    \item quantizes all dirty state data with the apropriate net serializer
                    \item updates the filtering status and prioritization of all net objects
                    \item 
                    \item send: create and fill packets by ticking all iris data streams
                \end{itemize}
                \item \textbf{\uline{Net-token-data-stream}}: serializes any new tokens with the apropriate net serializer
                \item \textbf{\uline{Replication-data-stream}}: for data read from the replication data stream
                \begin{itemize}
                    \item deserializes and reads the received state data
                    \item instantiates new objects immediately since they are required to build replication protocols
                \end{itemize}
                \item 
            \end{itemize}
            


        \subsection{General notes}
            \begin{itemize}
                \item handle unexpected network behavior with \code{UEngine::OnNetworkFailure}
                
            \end{itemize}


    \section{CPP}
        \subsection{Replicate variables}
            \begin{itemize}
                \item any variable you want to replicate needs \colorbox{mygray}{\lstinline{UPROPERTY(replicated)}} 
                \item and they need to implement the 'GetLifetimeReplicatedProps'-Function
            \end{itemize}
            \begin{lstlisting}
        void ATestPlayerCharacter::GetLifetimeReplicatedProps(TArray<FLifetimeProperty>& OutLifetimeProps) const {
            Super::GetLifetimeReplicatedProps(OutLifetimeProps);

            // Here we list the variables we want to replicate + a condition if wanted
            DOREPLIFETIME(ATestPlayerCharacter, Health, COND_[...]);
        }
            \end{lstlisting}

            \uline{Possible Conditions:}
            \begin{itemize}
                \item COND\_InitialOnly := only attempt to send on the initial bunch
                \item COND\_OwnerOnly := only send to the owner
                \item COND\_SkipOwner := send to every client except the owner
                \item COND\_SimulatedOnly := only send to simulated actors
                \item COND\_AutonomousOnly := only send to autonomous actors
                \item COND\_SimulatedPhysics := send to simulated || bRepPhysics actors
                \item COND\_InitialOrOwner := send to initial packet || actors owner
                \item COND\_Custom := 
            \end{itemize}

            \uline{Replication-Methods}
            \begin{itemize}
                \item replicated := 
                \item ReplicatedUsing=[FunctionName] := will call a function after a value is replicated
            \end{itemize}
            
    \section{The Replicataion System}

        !!! \textbf{not replicated features of Actors, Pawns, Characters} !!!
        \begin{itemize}
            \item Skeletal Mesh and Static Mesh component
            \item Materials
            \item Animation Blueprints
            \item Particle Systems
            \item Sound Emitters
            \item Physics Objects
            \item 
            \item $\rightarrow$ so replicating the variables that drive them $\rightarrow$ every client creates a similar effect
        \end{itemize}



    \section{Blueprints}
        \begin{itemize}
            \item Any BP-class that should replicate $\rightarrow$ set 'Replicates'-Property to true
            \item 
            \item 'IsStandalone'-Node := to check if it's an offline session use 
            \item 
            \item 'HasAuthority'-Node := tells who has authority over the actor \& if it's replicated or not \& mode of replication \url{https://docs.unrealengine.com/4.27/en-US/InteractiveExperiences/Networking/Actors/Roles/}{Official Docs}
            \item $\rightarrow$ ROLE\_SimulatedProxy will update 'AActor' based on it's 'NetUpdateFrequency' and predict ActorMovement etc. between those intervals based on it's velocity
            \item $\rightarrow$ writing custom code for replication can be beneficial
            \item 
        \end{itemize}


    \section{C++}
        \begin{itemize}
            \item you can add specifiers to 'UPROPERTY' in order to enable replication
            \begin{itemize}
                \item 'Replicate' := enables replication
                \item 'ReplicateUsing = FunctionName' := will replicate and execute the specified function on changes
                \item \colorbox{mygray}{\lstinline{GetLifetimeReplicatedProps(TArray <FLifetimeProperty> & OutLifetimeProps) const}}:
                \begin{itemize}
                    \item is the function that is responsible to replicate the properties
                    \item 'Super::GetLifetimeReplicatedProps(OutLifetimeProps);' must be called
                \end{itemize}
            \end{itemize}
            \item 
        \end{itemize}

        \uline{Needed Headers:}
        \begin{itemize}
            \item \colorbox{mygray}{\lstinline{#include "Net/UnrealNetwork.h"}}
            \item 
        \end{itemize}


    \section{Crossplay}
        \begin{itemize}
            \item Game Services: 
            \begin{itemize}
                \item works with any account system
                \item matchmaking
                \item lobbies
                \item peer-to-peer
                \item Voice Chat
                \item 
                \item account linking
                \item player data storage
                \item stats
                \item achievements
                \item leaderboards
                \item inventory
                \item 
                \item Anti-Cheat
                \item Player Reports
                \item Player Sanctions
            \end{itemize}
            \item Account Services: 
        \end{itemize}

    \chapter{Modular Character}

            \section{Comparison of different ways}
                 \includegraphics[width=\textwidth]{ModularCharacterComparison.jpg}

            \section{Master Pose Component}
                The core is the 'Master Pose Component', enabling to set 'Skinned Mesh Component Object' as children
                to another 'Skinned Mesh Component' that is considered to be the master \\
                Children do not use any Bone Transform Buffer and don't run any animations \\
\smallskip \\
                This setup will be done in the construction script of a BP \\
                \includegraphics[width=\textwidth]{ConstructionScript.jpg} \\
                Children of the Master Bone has to be a subset with exact matching structure. \\
                No extra joints or skip any joints. Since there are no Bone Buffer data for extra joints,
                it will render using the reference pose. \\
                Cannot run any other animations or physics on any children. \\


                \underline{Components needed} (thirdpersoncharacter)
                \begin{itemize}
                    \item CapsuleComponent
                    \item ArrowComponent
                    \item SpringArmComponent
                    \item Camera
                \end{itemize}
\smallskip            
                And then you need a 'Skeletal Mesh' for every clothing part:
                \begin{itemize}
                    \item Hat
                    \item Upper Body (seperate underwear and shirt)
                    \item Arms
                    \item Hands
                    \item Lower Body (Seperate underwear and pants)
                    \item Feet
                \end{itemize}
\smallskip 
                \begin{itemize}
                    \item Copy Location/Rotation from torso to rest
                    \item Assign same BP for the clothings
                    \item 'Copy Pose From Mesh' is an AnimGraph node used on the Animation Blueprint of the child (only matching bones everything else will use reference pose (from parent))
                \end{itemize}
\smallskip
                Note: Copy Pose From Mesh is more expensive than Master Pose Component because this runs the animation on each child \\
                Additionally, if you want to use physics on the child, you may want to use the Rigid Body  or AnimDynamics  skeletal control nodes instead \\
                \includegraphics[width=\textwidth]{CopyPoseFromMesh.jpg} \\
\smallskip \\
                While PREVIEWING in the animation editor additional assigned meshed will automatically use copy pose from mesh \\
                OR use a CUSTOM PREVIEW MESH COLLECTION \\
\smallskip\\
                MERGE multiple Skeletal Meshes at runtime into a single Skeletal Mesh through code with FSKELETALMESHMERGE. \\
                High INITIAL cost of creating the Skeletal Mesh, the rendering cost is cheaper renderinga a single Skeletal Mesh instead of multiple meshes. \\
                For example character comprised of three Components (head, body and legs).  50 characters on screen $\rightarrow$ 50 draw calls. \\
                Without Skeletal Mesh Merge: each Component own draw call + three calls per character $\rightarrow$ 150 draw calls.  \\
\smallskip \\
                When using FSKELETALMESHMERGE the Master Pose Component (body) has to contain all the animations \\
                If extra joints for certain body parts, still need all animations on the body. \\
                Can only run one animation on the merged mesh and transferring Morph Targets to the merged mesh is not supported. \\
                With a mergedMesh $\rightarrow$ FSkeletalMeshMerge::GenerateLODModel create your Morph Targets by calculating the FMorphTargetDelta between your base mesh and any morphs. \\
\smallskip \\
                FSkeletalMeshMerge build your content in a specific way from the start.
                use one common Material and decide on an atlas for your Textures (for example, boots go in this region while gloves go in this region and so on)
                so you can cut up and put together your textures to make new ones and render your whole character as one section \\

\smallskip

\chapter{PixelStreaming}
\label{PixelStreaming}
    \section{Introduction}
        \begin{itemize}
            \item allows you to run a packaged application on a desktop PC and stream the viewport to clients using WebRTC
            \item clients interact with it
        \end{itemize}

    \section{Setup}
        \href{https://docs.unrealengine.com/en-US/Platforms/PixelStreaming/GettingStarted/index.html}{Official Docs}
        \begin{itemize}
            \item install node.js
            \item enable the plugin \code{PixelStreaming}
            \item under \code{Editor Preferences} add \code{Additional Launch Parameters} \\
            \code{-AudioMixer -PixelStreamingIP=localhost -PixelStreamingPort=8888} 
            \item after building the package you have to add the same parameters to the .exe \\
            you can do this by creating a shortcut and adding the parameters to the end of \code{Target}
            \item adding \code{-RenderOffScreen} will enable streaming even if the window gets minimized
            \item 
            \item \href{https://github.com/EpicGamesExt/PixelStreamingInfrastructure.git}{download the server from github} \\
            or use the .sh script inside \code{\\Engine\\Plugins\\Media\\PixelStreaming\\Resources\\WebServers}
            \item 
            \item inside \code{SignallingWebServer\\platform\_scripts\\cmd} use the \code{Start_SignallingServer.ps1}
            \item the command line will inform about the status of the server
            \item 
            \item now launch the application from the created shortcut
        \end{itemize}

    \section{Controls}
        \begin{itemize}
            \item the controls for the application are inside of the \code{Frontend} directory of the downloaded PixelStreaming directory
            \item 
        \end{itemize}

\chapter{Optimization}
    \section{General}
    \begin{itemize}
        \item DISABLE EVENT TICK ON EVERY CLASS in the class defaults
        \item 
        \item Unreal engine has a SetTransform and SetActorLocationAndRotation calls, that let you change rotation and location in 1 call instead of 2. Calling SetActorLocationAndRotation is twice faster than calling SetLocation and SetRotation separately.
        \item 
        \item check what the current bottleneck is with 'stat unit' command BUT
        \item unreal insight gives a much better view
        \item 
        \item easiest way to tell is to just look at stat unitgraph from the packaged game
        \item 
    \end{itemize}

    \section{Profiling (Identify bottlenecks/problems)}
        \subsection{Texture streaming metrics}
                \begin{lstlisting}
            stat streaming sortby=name maxhistoryframes=1                
                \end{lstlisting}

        
            \begin{table}[!htb]
                \begin{tblr}{p{6cm} | p{12cm}}
                    \hline
                        Command & Description \\
                    \hline
                    memreport -full & \makecell{will print all currently loaded assets} \\
                    obj list Class=AnimSequence & \makecell{used to list all loaded assets of a class} \\
                    'stat unit' Game(altering position/rotation)(==CPU)\{Animations, Physics, Collision, AI, Spawning/Destroying ,...\} GPU(any rendering) \{Lighting, Redering of models, Refelctinos, shaders, ...\}
                    't.maxfps = [NUMBER]' & set fps limit \\
                    'stat rhi' & shows things like triangles drawn, draw primitive calls ... \\
                    'stat r' & \\
                    'stat scenerendering' & \\
                    'stat fps' & \\
                    'r.shadowquality [NUMBER]' && sets the shadow quality \\
                    'stat init views' & \\
                \end{tblr}
            \end{table}
        
                
                
    \section{Insights}
        \begin{itemize}
            \item is located in the \code{Engine/Binaries/[Platform]/UnrealInsights.exe}
            \item or binary build in \code{Engine/Build/BatchFiles/RunUBT.bat UnrealInsights Win64 Development}
            \item 
            \item \code{Trace} :=
            \begin{itemize}
                \item structured logging framework
                \item 
            \end{itemize}
            \item 
        \end{itemize}
        
        \subsection{adding arguments to visual studio}
            \begin{itemize}
                \item \code{RC} on the game project $\rightarrow$ \code{Properties}
                \item \code{Debugging} $\rightarrow$ \code{Command Arguments}
                \item $\hookrightarrow$ add your arguments
                \item default is: \code{\$(SolutionDir)FightForFame.uproject" -skipcompile}
                \item you can add something like: \\
                \code{\"\$(SolutionDir)FightForFame.uproject\" -skipcompile -trace=default,memory}
            \end{itemize}

TRACE\_CPUPROFILER\_EVENT\_SCOPE();
SCOPE\_CYCLE
TRACE\_BOOKMARK


Editor Preferences -> Play in Standalone Game -> Additional Launch Parameters "-cpuprofilertrace" "-loadtimetrace"


    \section{Instanced Static Mesh\& Hierarchical Instanced Mesh}
        \includegraphics[width=\textwidth]{InstancedStaticMesh.png} \\
        \underline{General}
        \begin{itemize}
            \item First create a single mesh from multiple meshes (RC $\rightarrow$ convert actors to static mesh)
        \end{itemize}
        !!! Transforms have to be the same for all instances !!!
        \begin{itemize}
            \item Have no LOD
        \end{itemize}
        \underline{That's where HLOD come into play}
        \begin{itemize}
            \item Creates an instance for every LOD
            \item scale has always to be 1.0 ?
        \end{itemize}

    \section{HLOD}
        \begin{itemize}
            \item Groups multiple objects into one single object
        \end{itemize}

    \section{Code Profiling}


    \section{Reference Viewer}
        \begin{itemize}
            \item \code{open:} 'RC' on an asset and select \code{Reference Viewer} in the content browser
        \end{itemize}


    \section{Obejct/Class/Hard Reference VS SoftReference}
        \begin{itemize}
            \item SoftReferences have to be cast into the class they will be used as
            \item 
        \end{itemize}


    \section{Common problems and solutions}
        \begin{table}[!htb]
            \begin{tabular}{p{5cm}|p{5cm}}
                \hline
                    Problem & Solution \\
                \hline
                too many polygons & LODs \\
                too many skeletal meshes are animating & Animation budget allocator \\
                & \\
                memory & use pooling to reduce object creation/deletion \\
                too many unit calculations & cluster the units \\
                & Animation Sharing Manager \\
                Memory pool & check loaded textures \\
                \hline
            \end{tabular}
        \caption{ caption }  
        \end{table}




    \section{Textures}
        \begin{itemize}
            \item streaming virtual textures are good to use with big textures of which some part will probably not be visible in the near future
            \item Enable Crunch compression: will reduce the memory usage (use probably always)
            \item can be enabled automatically for new imported textures
            \begin{itemize}
                \item Project Settings > Texture Import > Virtual Textures category 
                \item Auto Virtual Texturing Size
            \end{itemize}
            \item 
        \end{itemize}


    \section{Monitor performance}
        \begin{itemize}
            \item put a QUICK\_SCOPE\_CYCLE\_COUNTER(STAT\_MyRotationFunction)
            \item $\rightarrow$ use \code{stat quick} to check it
        \end{itemize}


    \section{Significance Manager}
        \begin{itemize}
            \item only provides the framework for determining the significance of an Object $\rightarrow$ actual calculation to be defined by the project
            \item functions that will be called on the Object during Significance Manager updates and must be registered:
            \begin{itemize}
                \item \code{FSignificanceFunction}: primary evaluation function; takes \code{Object} and a single \code{Transform};  calculates and returns the significance as a float. During the Significance Manager's update process, this function will be called once for each Transform that was passed in. The final result will be determined by the Significance Manager's Update function; by default, it will be the highest value. Each registered Object is required to be associated with a function of type FSignificanceFunction when it is registered. 
                \item \code{FPostSignificanceFunction}: 
            \end{itemize}
            \item ways to get significance of an object:
            \begin{itemize}
                \item \code{GetSignificance}: returns a significance value
                \item \code{QuerySignificance}: returns \code{false} if the object is not registered
            \end{itemize}
            \item \code{Update}:
            \begin{itemize}
                \item does not run automatically 
                \item good place to call it in an overridden version of UGameViewportClient
                \item 
            \end{itemize}
        \end{itemize}


    


    \begin{lstlisting}
    #include "MyGameViewportClient.h"
    #include "SignificanceManager.h"
    #include "Kismet/GameplayStatics.h"
    void UMyGameViewportClient::Tick(float DeltaTime)
    {
        // Call the superclass' Tick function.
        Super::Tick(DeltaTime);
        // Ensure that we have a valid World and Significance Manager instance.
        if (UWorld* World = GetWorld())
        {
            if (USignificanceManager* SignificanceManager = FSignificanceManagerModule::Get(World))
            {
                // Update once per frame, using only Player 0's world transform.
                if (APawn *PlayerPawn = UGameplayStatics::GetPlayerPawn(World, 0))
                {
                    // The Significance Manager uses an ArrayView. Construct a one-element Array to hold the Transform.
                    TArray<FTransform> TransformArray;
                    TransformArray.Add(PlayerPawn->GetTransform());
                    // Update the Significance Manager with our one-element Array passed in through an ArrayView.
                    SignificanceManager->Update(TArrayView<FTransform>(TransformArray));
                }
            }
        }
    }
    \end{lstlisting}


\chapter{Gameplay Mechanics}
    \section{Screenshots/Thumbnails}
        \begin{itemize}
            \item add \code{UScreenCaptureComponent2D} to the player BP
            \item \code{Capture Source}: \code{Final Color (LDR)}
            \item 
            \item 
            \item create \code{RenderTarget}
            \item set \code{Gamma} to 2.2
            \item set the 
            \item specify it in the \code{Texture Target} of \code{UScreenCaptureComponent2D}
        \end{itemize}

    
            

    \chapter{Cooking \& Packaging}
        \begin{itemize}
            \item is the process of: converting internal format to the platform specific format
            \item assets are stored in a custom format inside unreal
        \end{itemize}

    \chapter{Debugging}


        \section{Basic methods}
            \begin{itemize}
                \item Print string and an additional Macro with variables (ex. bPrintDebugMessages)
                \item \hyperref[sec:visuallogger]{visual logger} (ex. heatmaps showing )
                \item console commands
                \item execute console commands from BP (for example UI to toggle things)
                \item get platform name
            \end{itemize}
        

        \section{Output Log}
            \begin{itemize}
                \item tells what is going on in UE4
                \item 
            \end{itemize}


    \chapter{Console Commands}
        In order to open cmd hit \^ \\
        \underline{Show collision boxes}
        \begin{lstlisting}
        show Collision
        \end{lstlisting}
        \begin{table}[H]
            \begin{tabular}{|l|c|}
                \hline
                    Command & Description \\
                \hline
                    [COMMAND] ? & help for the command \\
                    stat scenerendering & diplsays rendering counts \\
                    stat memory & shows the memory usage \\
                    stat none & remove all stat info \\
                    stat unitgraph & graph gpu usage \\
                    obj list class=skeletalmesh & will display all objects of a type in memory \\
                    show Collision & shows the collision boxes for meshes \\
                    r. & are rendering commands \\
                    r.SkyAtmosphere.Visualize 1 & turn on the Visualization for debugging \\
                    r.VT.Borders 1 & shows which textures are virtual textures \\
                    ShowDebug Bones & will show the Skeletal Mesh Bones \\
                    ShowDebug Animation & will show information about the animation states ... \\
                    
                \hline
            \end{tabular}
        \end{table}
\smallskip

    \chapter{Create Custom Template}
        \href{https://www.youtube.com/watch?time_continue=119&v=MYM7iSh-uac&feature=emb_logo}{YT-Video} explaining the process \\
        \begin{itemize}
            \item Create new project
            \item setup everything you want to be in the template
            \item regenerate project id: Project Settings $\rightarrow$ Description $\rightarrow$ Project ID
            \item Copy:
            \begin{itemize}
                \item Config
                \item Content
                \item Source
                \item Saved
                \item *.uproject
            \end{itemize}
            \item from the 'ProjectFolder' to 'Engine/Templates/...'
            \item 
            \item Create a 'Media'-Folder and place an image there
            \item Copy the 'TempalteDefs.ini' from an existing template
        \end{itemize}


\chapter{Plugins}

    \section{Unsorted Notes}
        \begin{itemize}
            \item a game feature plugin "knows" about the game but the game doesn't know about game features
            \item a game knows about plugins but plugins don't know about the game
            \item a game feature knows about a plugin since it can depend on
        \end{itemize}

    \section{Enable/Disable Default plugin}
        \begin{itemize}
            \item simply edit the \code{.uplugin-file}
            \item add/remove \code{\"EnabledByDefault\": true,}
        \end{itemize}

    \section{Plugin Structure}
        \uline{\textbf{In File-Explorer:}}
        \begin{itemize}
            \item Binaries: compiled binaries (dll and debug databases)
            \item Content:
            \item Intermediat: contains temporary files
            \item Source: contains the source code
            \item Resources: files that the plugin needs (for example the icon \colorbox{mygray}{\lstinline{Icon128.png}} with a size of 128x128)
            \item 
            \item $[PluginName].uplugin$: JSON-file containing the information of the plugin and used by the engine to register the plugin
        \end{itemize}

        \uline{\textbf{Plugin-Locations:}}
        \begin{itemize}
            \item Engine: $[UE4 Root]/Engine/Plugins/[Plugin Name]/$
            \item Game: $[Project Root]/Plugins/[PluginName]/$
        \end{itemize}

        \uline{\textbf{Plugin-Configuration-Location:}}
        \begin{itemize}
            \item Engine plugins: $[PluginName/]/Config/Base/[PluginName].ini$
            \item Game plugins: $[PluginName]/Config/Default/[PluginName].ini$
        \end{itemize}

        \uline{\textbf{Possible Plugin Dependencies}} \\
        \includegraphics[width=\textwidth]{PluginDependencies.jpg}

        \section{Create Plugins}
            How to create new plugins: \\
            \includegraphics[width=\textwidth]{CreateNewPlugin.png} \\
            Everything will be setup if created this way \\
            \underline{Filestructure:}
            \begin{itemize}
                \item Will be placed in the Plugin folder of the project
                \item has the basic structure
                \item code itself is placed in 'Source'
                \item 'Source' $\rightarrow$ 'Public' + 'Private'
                \item every module needs it's own build file
                \item \textbf{Content-Plugin:} "CanContainContent" setting within the Plugin's descriptor must be set to "true"
            \end{itemize}
            
            \uline{\textbf{Types:}}
            \begin{itemize}
                \item Runtime: will always be loaded (even in shipped games)
                \item RuntimeNoCommandlet:
                \item Developer: development runtime or editor builds but never in shipped games
                \item Editor: only when editor is starting up
                \item EditorNoCommandlet:
                \item Program:
            \end{itemize}

    \section{Plugin-Descriptor-File}
        \begin{itemize}
            \item has the \colorbox{mygray}{\lstinline{.uplugin}}-extension
            \item JSON-format
            \item 
        \end{itemize}
        \uline{\textbf{Example of a plugin descriptor-file}}
        \begin{lstlisting}
    {
        "FileVersion" : 3,
        "Version" : 1,
        "VersionName" : "1.0",
        "FriendlyName" : "UObject Example Plugin",
        "Description" : "An example of a plugin which declares its own UObject type.  This can be used as a starting point when creating your own plugin.",
        "Category" : "Examples",
        "CreatedBy" : "Epic Games, Inc.",
        "CreatedByURL" : "http://epicgames.com",
        "DocsURL" : "",
        "MarketplaceURL" : "",
        "SupportURL" : "",
        "EnabledByDefault" : true,
        "CanContainContent" : false,
        "IsBetaVersion" : false,
        "Installed" : false,
        "Modules" :
        [
            {
                "Name" : "UObjectPlugin",
                "Type" : "Developer",
                "LoadingPhase" : "Default"
            }
        ]
    }
\end{lstlisting}

        \subsection{Module-Descriptors}
            \href{https://docs.unrealengine.com/4.27/en-US/API/Runtime/Projects/FModuleDescriptor/}{off. API}
            \begin{itemize}
                \item valid type options are:
                \begin{itemize}
                    \item Runtime := even in shipped games
                    \item RuntimeNoCommandlet
                    \item Developer := development build or editor build
                    \item Editor
                    \item EditorNoCommandlet
                    \item Program
                \end{itemize}
            \end{itemize}
        \uline{\textbf{Example of a Module-Descriptor}}
        \begin{lstlisting}
    {
        "Name" : "UObjectPlugin",
        "Type" : "Developer"
        "LoadingPhase" : "Default"
    }
        \end{lstlisting}

    \section{Bare Minimum}
        \begin{itemize}
            \item EditorModule has to derive from IModuleInterface
            \item implement 'StartupModule()' \& 'ShutdownModule()'
            \item 
        \end{itemize}

    \section{Useful Modules}
        \begin{itemize}
            \item UnrealEd:
            \item SlateCore
            \item Slate
            \item ContentBrowser
            \item PropertyEditor
            \item LevelEditor
            \item DetailCustomizations
        \end{itemize}

    \section{Create a Asset Type}
        \begin{itemize}
            \item First declare a assset types C++ class
            \item add factories to create instances of the asset
            \item customize asset appearence
            \item add asset specific content browser action
            \item add custom asset editor UI
        \end{itemize}

    \section{Add Factory}
        \begin{itemize}
            \item UFactory
            \item different types: Content Browser Context Menu, Content Browser Drag \& Drop
        \end{itemize}

    \section{Publish a Plugin}
        \begin{itemize}
            \item files/folders you will have to send to epic:
            \begin{itemize}
                \item \code{YourPlugin.uplugin}-file
                 \item \code{Resources}-folder
                 \item \code{Source}-folder
                 \item no build files
            \end{itemize}
        \end{itemize}

\chapter{Develop tools and extend the editor}
    \section{extra notes}
        \begin{itemize}
            \item we need to know the \code{Extension points} when we add menu entries
            \item for that we can enable the option to \code{display ui extension points} in editor settings
            \item 
            \begin{itemize}
                \item \code{Blutility}
				\item \code{EditorScriptingUtilities}
                \item \code{ContentBrowser}
            \end{itemize}
            \item headers:
            \begin{itemize}
                \item \code{EditorAssetLibrary.h}
                \item \code{Framework/Notifications/NotificationManager.h}
                \item \code{Widgets/Notifications/SNotificationList.h}
                \item \code{Misc/MessageDialog.h}
                \item \code{ObjectTools.h}
                \item \code{}
            \end{itemize}
            \item 
            \item \code{FAssetData}
            \begin{itemize}
                \item \code{FAssetData::AssetName} := The name of the asset
                \item \code{FAssetData::PackageName} := path to asset from Content/Game
                \item \code{FAssetData::AssetClassPath::GetAssetName()} := the type of the asset eg. \glqq StaticMesh\grqq
            \\ \end{itemize}
            \item \code{Viewport} := 
            \item \code{ViewportClient} := processes the input to the viewport and draws the viewport
            \item \code{Toolkits} := are supposed to hold the UI
            \item 
            \item \code{UAssetEditor::Initialize();}
            \begin{itemize}
                \item This will do a variety of things including registration of the asset editor, creation of the toolkit
                \item (via CreateToolkit()), and creation of the editor mode manager within the toolkit.
                \item The asset editor toolkit will do the rest of the necessary initialization inside its PostInitAssetEditor.
            \end{itemize}
            \item improve asset loading
            \begin{itemize}
                \item \code{UToolMenu} is used to add Right-Click entries
                \item added tools to the context menu of asset will benefit from using \code{Asset Tag Data} because loading can be more specific
                \item run the editor with -WarnIfAssetsLoaded (ConfigurationProperties -> Command Arguments)
            \end{itemize}

            
        \end{itemize}
        AssetDefinitions.h : line 280
        \begin{lstlisting}
* 
* Unfortunately I can't prevent people from backsliding, at least for now.  Even after fixing the APIs to not require
* loading, people need to be cleverer (Use Asset Tag Data) in how they provide right click options for assets.  But to
* help, you can run the editor with -WarnIfAssetsLoaded on the command line.  I've added a new utility class to the
* engine called, FWarnIfAssetsLoadedInScope, it causes notifications with callstacks to be popped up telling you what
* code is actually responsible for any asset loads within earmarked scopes that should NOT be loading assets.
* 
* 
* Registration
* Registering your Asset Definition is no longer required like it was for Asset Actions.  The UObjects are automatically
* registered with the new Asset Definition Registry (UAssetDefinitionRegistry).
* 
* It's very important that you do not load the asset in your CanExecuteAction callback or in this self callback, you should save that until you finally get Executed.
* If you're looking for examples, there are tons you'll find by searching for \glqq namespace MenuExtension\_\grqq
        \end{lstlisting}
        FROM THE UVEDITOR
        \begin{lstlisting}
* The actual asset editor class doesn't have that much in it, intentionally. 
* 
* Our current asset editor guidelines ask us to place as little business logic as possible
* into the class, instead putting as much of the non-UI code into the subsystem as possible,
* and the UI code into the toolkit (which this class owns).
*
* However, since we're using a mode and the Interactive Tools Framework, a lot of our business logic
* ends up inside the mode and the tools, not the subsystem. The front-facing code is mostly in
* the asset editor toolkit, though the mode toolkit has most of the things that deal with the toolbar
* on the left.
        \end{lstlisting}

    \section{ways to extend the editor}
        \begin{itemize}
            \item Editor Utility
            \begin{itemize}
                \item \code{Utility Actor}
                \item \code{Utility Object}
                \item \code{Utility Asset Action}
            \end{itemize}
        \end{itemize}

        
    \section{Create Utility-Actor-Action (ScriptedAction in RightClickMenu in level viewport)}
        
        \begin{itemize}
            \item create an \code{Actor-Action-Utility}
            \item add \code{Custom-Event} for each action you want to add
            \item enable \code{Call in Editor} on the Event
        \end{itemize}    
        
        \uline{Making \code{Actions} context sensitive (for specific asset classes)}
        \begin{itemize}
            \item use \code{GetSupportedClass}
        \end{itemize}
        \includegraphics[width=\textwidth]{GetSupportedClassExample.jpg}


    \section{Add Right Click entries}
        \begin{itemize}
            \item so we have to:
            \item \code{access array of menu entries:} use those $\downarrow$
            \item \code{GetAllPathViewContextMenuExtenders()} \code{GetAllAssetViewContextMenuExtenders()}
            \item 
            \begin{lstlisting}
                ContentBrowserModuleMenuExtension.Add(FContentBrowserMenuExtender_SelectedPaths::CreateRaw(this, &FAssistingUtilsModule::CustomCBMenuExtender));
            \end{lstlisting}
            \item 
            \item when we build the menu we will specify/create another delegate that will be called when our menu entry is selected
            \item \code{AddMenuExtension()}
            \item 
            \item the \code{FUICommandList} in the menu-builder is used to specify custom shortcuts
        \end{itemize}
        
        \begin{enumerate}
            \item FNewModule::StartupModule
            \begin{enumerate}
                \item load \code{ContentBrowserModule}
                \item get the array of menu entries delegates
                \item create new entry
                \item \code{BindRaw} a function to the new entry
                \item and \code{Add()} the new entry to the array
            \end{enumerate}
            \item TSharedRef<FExtender> FNewModule::ReturnCreatedExtender(const TArray<FAssetData>\& AssetInfo)
            \begin{enumerate}
                \item create \code{TSharedRef<FExtender>}
                \item add a MenuExtension
                \item 
            \end{enumerate}
        \end{enumerate}

        \begin{lstlisting}
    /* NewModule.cpp 
    *  FNewModule::StartupModule */
    FContentBrowserModule LoadedCBM = FModuleManager::Get().LoadModuleChecked<FContentBrowserModule>("ModuleManager") // 1.1
    TArray<FContentBrowserMenuExtender_SelectedAssets>& AssetExtender_A = LoadedCBM.GetAllAssetViewContextMenuExtenders(); // 1.2
    FContentBrowserMenuExtender_SelectedAssets MyAssetExtender; // 1.3
    MyAssetExtender::BindRaw(this, &FNewModule::ReturnCreatedExtender); // 1.4
    AssteExtender_A.Add(MyAssetExtender); // 1.5
        \end{lstlisting}

    \section{UICommands}
        \begin{itemize}
            \item basic setup
            \begin{itemize}
                \item create new class that inherits from \code{public TCommands<FYourClassNameCommands>}
                \item   
                \item in \code{.h} 
                \item \code{virtual void RegisterCommands() override;}
                \item create constructor
                \begin{lstlisting}
    FSlateQuickstartWindowCommands()
    : TCommands<FbCommands>(
        TEXT("SlateQuickstartWindow"),
        NSLOCTEXT("Contexts", "SlateQuickstartWindow", "SlateQuickstartWindow Plugin"),
        NAME_None,
        FSlateQuickstartWindowStyle::GetStyleSetName())
                \end{lstlisting}
                \item add \code{TSharedPtr< FUICommandInfo > commandName} for each command
                \item  
                \item in \code{.cpp} 
                \item inside \code{RegisterCommands}
                \begin{lstlisting}
    UI_COMMAND(
        TestCommand,
        "TestCommand", 
        "This is test command", 
        EUserInterfaceActionType::Button, 
        FInputGesture()
    );
                \end{lstlisting} 
                \item 
                \item 
                \item in the module (that inherits from \code{IModuleInterface})(\code{.h}) add
                \item \code{TSharedPtr<class FUICommandList> PluginCommands;}
                \item 
                \item call the \code{.Register()}-function of the your \code{TCommandList}-class
                \item and map the commands to actions
                \begin{lstlisting}
    PluginCommands = MakeShareable(new FUICommandList);
    PluginCommands->MapAction(
        FTutorialPluginEditorCommands::Get().TestCommand,
        FExecuteAction::CreateRaw(this, &FTutorialPluginEditorModule::TestAction)
    );
                \end{lstlisting}
            \end{itemize}
        \end{itemize}


    \section{Create a tab}
        \begin{itemize}
            \item has to be called on in the \code{StartupModule}
            \item access the \code{FGlobalTabManager::Get()}
            \item and register a new tab \code{RegisterNomadTabSpawner()}
        \end{itemize}



    \section{Creating a custom editor}
        \begin{itemize}
            \item \code{IToolkit} := is there for the visual representation
            \item \code{FToolkitManager} := keeps track of all the toolkits
            \item \code{Editor:} :=
        \end{itemize}

    \section{Code snippets}
        \uline{Get selected assets in content browser:}
        \begin{lstlisting}
    TArray<FAssetData> AssetInfoArray;
    AssetSelectionUtils::GetSelectedAssets(AssetInfoArray);
        \end{lstlisting}

        \uline{Get asset type of the selected assets}
        \begin{lstlisting}
    // after getting an array of FAssetData from the selected items
    // in the content browser, we can use the AssetInfo
    
    // Will return a FName like "StaticMesh" "Texture2D"
    AssetInfo.AssetClassPath.GetAssetName()
        \end{lstlisting}
    
    \section{K2Nodes}
        \subsection{Basics}
            \begin{itemize}
                \item dynamic functionality
                \item inputs and outputs of a Node change depending on other inputs and outputs
                \item functionality change input types
                \item 
                \item One way to think of it is a Blueprint Node that has two or more functions
                      inside of it; one that takes the inputs, passes it around through N functions,
                      then another function that returns the output, like a subgraph in Blueprints
                \item \code{K2Node_SpawnActorFromClass} and \code{K2Node_GetDataTableRow} have solid comments and straightforward outcomes
                \item 
            \end{itemize}

\smallskip            
\smallskip
\newpage
\begin{landscape}
    \chapter{Unreal-CPP Cheatsheet}
    \begin{tcolorbox}[title=Basics, colframe=violet!95, colback=violet!45, center title]
        \begin{multicols}{2}
            PREFIXES:
            \begin{lstlisting}
F // Struct
T // Template
U // UObject
A // AActor
S // SWidget
I // Abstract interface
E // Enums
b // Booleans
GA_ // GameplayAbility
GE_ // GameplayEffect
AN_ // AnimNotify
W_ // Widget
EUW_ // EditorUserWidget
IA_ // InputAction
PMI_ // PlayerMappableInput
IMC_ // InputMappingContext
FFE_ // ForceFeedbackEffect
CS_ // CameraShake



LastMouseCoordinates // displayed as 'Last Mouse Coordiants' in editor
Last_Mouse_Coordinates // WRONG
bIsFalling // 'Is Falling' checkbox in editor
            \end{lstlisting}

            STRINGS:
            \begin{lstlisting}
FString
FName OneDataBaseEntry;
FText TextLocalization;
MyDogPtr->DogName = FName(TEXT("Samson));
            \end{lstlisting}
            
            Conversion:
            \begin{lstlisting}
TestHUDString = TestHUDName.ToString();
TestHUDText = FText::FromName(TestHUDName);
TestHUDName = FName(*TestHUDString);
TestHUDText = FText::FromString(TestHUDString);
TestHUDString = TestHUDText.ToString();
            \end{lstlisting}

            UPROPERTY SPECIFIERS:
            \begin{lstlisting}
BlueprintReadOnly //Variable available from BPs
BlueprintReadWrite //Variable available from BPs
Category="TopCategory|SubCategory|..."
Config #.ini file -> BlueprintReadOnly
GlobalConfig
EditAnywhere
EditInstanceOnly //any visible instance in editor
EditDefaultsOnly //Only the base
VisibleInstanceOnly
VisibleDefaultsOnly
Category "Category1"
Category "Category2 | SubCategory"
            \end{lstlisting}
            TARRAY FUNCTIONS:
            \begin{lstlisting}
Add() //might relocate the array in order to make it fit
AddUnique() 
Contains()
CountBytes() // count bytes needed to serialize
GetAllocatedSize()
Heapify()

            \end{lstlisting}    


        \end{multicols}
        \end{tcolorbox}
\end{landscape}

\newpage

\begin{landscape}
        \begin{tcolorbox}[title=Useful Actor Functions, colframe=violet!95, colback=violet!45, center title]
            \begin{multicols}{2}
            \begin{lstlisting}
#include "Components/StaticMeshComponent.h"
GetActorLocation()
SetActorLocation(VeCtOr)
RootComponent = CreateDefaultSubobject<USceneComponent>(TEXT("Root Component"));
MeshComponent = CreateDefaultSubobject<UStaticMeshComponent>(TEXT("NaMe"));
MeshComponent->SetupAttachment(GetRootComponent());
Camera = CreateDefaultSubobject<UCameraComponent>(TEXT("Camera"));
Camera->SetupAttachment(GetRootCOmponent());
Camera->SetRelativeLocation();
Camera->SetRelativeRotation();
MeshComponent->AddForce(FVector(200000000.f,0.0f,0.0f));
StaticMesh->AddTorque(FVector(FMath::FRandRange(2000000.f, 200000000.f),0.0f,FMath::FRand()));

AutoPossessPlayer = EAutoReceiveInput::Player0;
            \end{lstlisting}
            PAWN functions
            \begin{lstlisting}
PlayerInputComponent->BindAction("Jump", IE_Pressed, this, &ACharacter::Jump);
PlayerInputComponent->BindAction("Jump", IE_Released, this, &ACharacter::StopJumping);

            \end{lstlisting}
            UPROPERTY SPECIFIERS:
            \begin{lstlisting}

            \end{lstlisting}
        \end{multicols}
        \end{tcolorbox}
\end{landscape}


\newpage

\begin{landscape}
        \begin{tcolorbox}[title=Debugging, colframe=violet!95, colback=violet!45, center title]
            \begin{multicols}{2}
            \begin{lstlisting}
stat game // BP overall performance
Dumpticks // show all ticking actors
stat rhi //
stat memory //
stat scenerendering //
stat engine //
stat none //
stat fps //
DisableAllScreenMessages // ...
SHIFT+L // toggle all stats

            \end{lstlisting}
            PAWN functions
            \begin{lstlisting}

            \end{lstlisting}
            UPROPERTY SPECIFIERS:
            \begin{lstlisting}

            \end{lstlisting}
        \end{multicols}
        \end{tcolorbox}
\end{landscape}



    \chapter{Additional}
        \section{Add sockets to static meshes inside blender}
         \includegraphics[width=\textwidth]{AddSocketOnMesh.jpeg}



\newpage
    \glsaddall
    \printglossary[title=Glossar, toctitle=Glossar]
    \printglossaries      

\end{document}