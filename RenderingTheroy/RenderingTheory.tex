\chapter{Rendering}
    \section{Best practices}
        \begin{itemize}
            \item eliminate noise while testing
            \item turn off real time
            \item standalone mode
            \item stat unit (show primary threads)
            \item Draw = CPU
            \item GPU = GPU
            \item 
            \item turn off smooth 'frame rate'
            \item 'Project Settings'
            \item 
            \item Pause button = pause game
            \item 
            \item stat unit graph
        \end{itemize}


        \paragraph{GPU} \mbox{} \\
        \begin{itemize}
            \item 'profile gpu' grabs all computation for one frame
            \item 
            \item 'r.' are the rendering commands
            \item 
            \item stat scenerenderin
        \end{itemize}

        \paragraph{CPU} \mbox{} \\
        \begin{itemize}
            \item Window $\rightarrow$ Developer Tools $\rightarrow$ Session Forntend
            \item Profiler is most useful tab
            \item Load to load the written file from stat startfile and stat stopfile commands
            \item 
            \item 'stat startfile'
            \item "stat stopfile"
            \item 
            \item Blue tag on the side will give more information on the event
            \item "Avg num of calls" will give indicator on how often the blueprint gets called
            \item 
        \end{itemize}
    \section{Rendering Overview}
        \begin{itemize}
            \item rendering := is the process of generating a 2D image from a 3D scene
            \item the steps involved make up the \code{graphics pipeline}
            \item and the processing of the data is done by the \code{GPU}
            \item 
            \item point = pixel
            \item line = vector
            \item plane = filled polygon
        \end{itemize}

        \begin{table}[!htb]
            \begin{tblr}{p{8cm} | p{8cm}}
                \hline
                    Resolution & 24 bits per pixel \\
                \hline
                    128 x 128 & 393.216 \\
                    256 x 256 & 1.572.864 \\
                    512 x 512 & 6.291.456 \\
                    1024 x 1024 & 25.165.824 \\
                \hline
            \end{tblr}
        \caption{ caption }  
        \end{table}

        \subsection{Antialiasing}
            \begin{itemize}
                \item is a technique used to reduce the effect of aliasing
                \item aliasing := is the distortion; the jagged or stair-step effect on the edges of objects
                \item is based on sampling := representing a large set of values by an appropriate smaller set
                \item simple antialiasing techniques: 
                \begin{itemize}
                    \item average the color of the pixel
                    \item average the color of the pixel with its neighbors
                \end{itemize}
                \item 
            \end{itemize}
        
        \section{Rendering Pipeline}
        


        \subsection{Stages}
            \subsubsection{Application Stage}
                \begin{itemize}
                    \item is where CPU does the following work:
                    \begin{itemize}
                        \item Scene setup: organize and mage all the objects and lights in the scene
                        \item Animation and physics: calculate the position, rotation and state of objects
                        \item Culling: determine what is visible and what is not
                        \item State management: setting up various states like shaders, textures, etc.
                        \item Batching: combine multiple objects into one draw call
                        \item Draw call submission: send the draw calls to the GPU
                    \end{itemize}
                \end{itemize}

            \subsubsection{Geometry Processing Stage}
                \begin{itemize}
                    \item is where the data starts to transition from CPU $\rightarrow$ GPU:
                    \begin{itemize}
                        \item Vertex processing: process each vertex
                        \item Tessellation: optional stage that adds more vertices to the geometry
                        \item Geometry shader: process each primitive
                        \item Rasterization: convert the primitives into fragments
                        \item Fragment shader: process each fragment
                    \end{itemize}
                \end{itemize}

    \section{Rendering Pass}
        \begin{itemize}
            \item draw calls: is a command from CPU to the GPU to render something
            \item a pass: are multiple draw calls
        \end{itemize}