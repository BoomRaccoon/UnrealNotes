\chapter{Develop tools and extend the editor}
    \section{extra notes}
        \begin{itemize}
            \item we need to know the \code{Extension points} when we add menu entries
            \item for that we can enable the option to \code{display ui extension points} in editor settings
            \item 
            \begin{itemize}
                \item \code{Blutility}
				\item \code{EditorScriptingUtilities}
                \item \code{ContentBrowser}
            \end{itemize}
            \item headers:
            \begin{itemize}
                \item \code{EditorAssetLibrary.h}
                \item \code{Framework/Notifications/NotificationManager.h}
                \item \code{Widgets/Notifications/SNotificationList.h}
                \item \code{Misc/MessageDialog.h}
                \item \code{ObjectTools.h}
                \item \code{}
            \end{itemize}
            \item 
            \item \code{FAssetData}
            \begin{itemize}
                \item \code{FAssetData::AssetName} := The name of the asset
                \item \code{FAssetData::PackageName} := path to asset from Content/Game
                \item \code{FAssetData::AssetClassPath::GetAssetName()} := the type of the asset eg. \glqq StaticMesh\grqq
            \\ \end{itemize}
            \item \code{Viewport} := 
            \item \code{ViewportClient} := processes the input to the viewport and draws the viewport
            \item \code{Toolkits} := are supposed to hold the UI
            \item 
            \item \code{UAssetEditor::Initialize();}
            \begin{itemize}
                \item This will do a variety of things including registration of the asset editor, creation of the toolkit
                \item (via CreateToolkit()), and creation of the editor mode manager within the toolkit.
                \item The asset editor toolkit will do the rest of the necessary initialization inside its PostInitAssetEditor.
            \end{itemize}
            \item improve asset loading
            \begin{itemize}
                \item \code{UToolMenu} is used to add Right-Click entries
                \item added tools to the context menu of asset will benefit from using \code{Asset Tag Data} because loading can be more specific
                \item run the editor with -WarnIfAssetsLoaded (ConfigurationProperties -> Command Arguments)
            \end{itemize}

            
        \end{itemize}
        AssetDefinitions.h : line 280
        \begin{lstlisting}
* 
* Unfortunately I can't prevent people from backsliding, at least for now.  Even after fixing the APIs to not require
* loading, people need to be cleverer (Use Asset Tag Data) in how they provide right click options for assets.  But to
* help, you can run the editor with -WarnIfAssetsLoaded on the command line.  I've added a new utility class to the
* engine called, FWarnIfAssetsLoadedInScope, it causes notifications with callstacks to be popped up telling you what
* code is actually responsible for any asset loads within earmarked scopes that should NOT be loading assets.
* 
* 
* Registration
* Registering your Asset Definition is no longer required like it was for Asset Actions.  The UObjects are automatically
* registered with the new Asset Definition Registry (UAssetDefinitionRegistry).
* 
* It's very important that you do not load the asset in your CanExecuteAction callback or in this self callback, you should save that until you finally get Executed.
* If you're looking for examples, there are tons you'll find by searching for \glqq namespace MenuExtension\_\grqq
        \end{lstlisting}
        FROM THE UVEDITOR
        \begin{lstlisting}
* The actual asset editor class doesn't have that much in it, intentionally. 
* 
* Our current asset editor guidelines ask us to place as little business logic as possible
* into the class, instead putting as much of the non-UI code into the subsystem as possible,
* and the UI code into the toolkit (which this class owns).
*
* However, since we're using a mode and the Interactive Tools Framework, a lot of our business logic
* ends up inside the mode and the tools, not the subsystem. The front-facing code is mostly in
* the asset editor toolkit, though the mode toolkit has most of the things that deal with the toolbar
* on the left.
        \end{lstlisting}

    \section{ways to extend the editor}
        \begin{itemize}
            \item Editor Utility
            \begin{itemize}
                \item \code{Utility Actor}
                \item \code{Utility Object}
                \item \code{Utility Asset Action}
            \end{itemize}
        \end{itemize}

        
    \section{Create Utility-Actor-Action (ScriptedAction in RightClickMenu in level viewport)}
        
        \begin{itemize}
            \item create an \code{Actor-Action-Utility}
            \item add \code{Custom-Event} for each action you want to add
            \item enable \code{Call in Editor} on the Event
        \end{itemize}    
        
        \uline{Making \code{Actions} context sensitive (for specific asset classes)}
        \begin{itemize}
            \item use \code{GetSupportedClass}
        \end{itemize}
        \includegraphics[width=\textwidth]{GetSupportedClassExample.jpg}


    \section{Add Right Click entries}
        \begin{itemize}
            \item so we have to:
            \item \code{access array of menu entries:} use those $\downarrow$
            \item \code{GetAllPathViewContextMenuExtenders()} \code{GetAllAssetViewContextMenuExtenders()}
            \item 
            \begin{lstlisting}
                ContentBrowserModuleMenuExtension.Add(FContentBrowserMenuExtender_SelectedPaths::CreateRaw(this, &FAssistingUtilsModule::CustomCBMenuExtender));
            \end{lstlisting}
            \item 
            \item when we build the menu we will specify/create another delegate that will be called when our menu entry is selected
            \item \code{AddMenuExtension()}
            \item 
            \item the \code{FUICommandList} in the menu-builder is used to specify custom shortcuts
        \end{itemize}
        
        \begin{enumerate}
            \item FNewModule::StartupModule
            \begin{enumerate}
                \item load \code{ContentBrowserModule}
                \item get the array of menu entries delegates
                \item create new entry
                \item \code{BindRaw} a function to the new entry
                \item and \code{Add()} the new entry to the array
            \end{enumerate}
            \item TSharedRef<FExtender> FNewModule::ReturnCreatedExtender(const TArray<FAssetData>\& AssetInfo)
            \begin{enumerate}
                \item create \code{TSharedRef<FExtender>}
                \item add a MenuExtension
                \item 
            \end{enumerate}
        \end{enumerate}

        \begin{lstlisting}
    /* NewModule.cpp 
    *  FNewModule::StartupModule */
    FContentBrowserModule LoadedCBM = FModuleManager::Get().LoadModuleChecked<FContentBrowserModule>("ModuleManager") // 1.1
    TArray<FContentBrowserMenuExtender_SelectedAssets>& AssetExtender_A = LoadedCBM.GetAllAssetViewContextMenuExtenders(); // 1.2
    FContentBrowserMenuExtender_SelectedAssets MyAssetExtender; // 1.3
    MyAssetExtender::BindRaw(this, &FNewModule::ReturnCreatedExtender); // 1.4
    AssteExtender_A.Add(MyAssetExtender); // 1.5
        \end{lstlisting}

    \section{UICommands}
        \begin{itemize}
            \item basic setup
            \begin{itemize}
                \item create new class that inherits from \code{public TCommands<FYourClassNameCommands>}
                \item   
                \item in \code{.h} 
                \item \code{virtual void RegisterCommands() override;}
                \item create constructor
                \begin{lstlisting}
    FSlateQuickstartWindowCommands()
    : TCommands<FbCommands>(
        TEXT("SlateQuickstartWindow"),
        NSLOCTEXT("Contexts", "SlateQuickstartWindow", "SlateQuickstartWindow Plugin"),
        NAME_None,
        FSlateQuickstartWindowStyle::GetStyleSetName())
                \end{lstlisting}
                \item add \code{TSharedPtr< FUICommandInfo > commandName} for each command
                \item  
                \item in \code{.cpp} 
                \item inside \code{RegisterCommands}
                \begin{lstlisting}
    UI_COMMAND(
        TestCommand,
        "TestCommand", 
        "This is test command", 
        EUserInterfaceActionType::Button, 
        FInputGesture()
    );
                \end{lstlisting} 
                \item 
                \item 
                \item in the module (that inherits from \code{IModuleInterface})(\code{.h}) add
                \item \code{TSharedPtr<class FUICommandList> PluginCommands;}
                \item 
                \item call the \code{.Register()}-function of the your \code{TCommandList}-class
                \item and map the commands to actions
                \begin{lstlisting}
    PluginCommands = MakeShareable(new FUICommandList);
    PluginCommands->MapAction(
        FTutorialPluginEditorCommands::Get().TestCommand,
        FExecuteAction::CreateRaw(this, &FTutorialPluginEditorModule::TestAction)
    );
                \end{lstlisting}
            \end{itemize}
        \end{itemize}


    \section{Create a tab}
        \begin{itemize}
            \item has to be called on in the \code{StartupModule}
            \item access the \code{FGlobalTabManager::Get()}
            \item and register a new tab \code{RegisterNomadTabSpawner()}
        \end{itemize}



    \section{Creating a custom editor}
        \begin{itemize}
            \item \code{IToolkit} := is there for the visual representation
            \item \code{FToolkitManager} := keeps track of all the toolkits
            \item \code{Editor:} :=
        \end{itemize}

    \section{Code snippets}
        \uline{Get selected assets in content browser:}
        \begin{lstlisting}
    TArray<FAssetData> AssetInfoArray;
    AssetSelectionUtils::GetSelectedAssets(AssetInfoArray);
        \end{lstlisting}

        \uline{Get asset type of the selected assets}
        \begin{lstlisting}
    // after getting an array of FAssetData from the selected items
    // in the content browser, we can use the AssetInfo
    
    // Will return a FName like "StaticMesh" "Texture2D"
    AssetInfo.AssetClassPath.GetAssetName()
        \end{lstlisting}
    
    \section{K2Nodes}
        \subsection{Basics}
            \begin{itemize}
                \item dynamic functionality
                \item inputs and outputs of a Node change depending on other inputs and outputs
                \item functionality change input types
                \item 
                \item One way to think of it is a Blueprint Node that has two or more functions
                      inside of it; one that takes the inputs, passes it around through N functions,
                      then another function that returns the output, like a subgraph in Blueprints
                \item \code{K2Node_SpawnActorFromClass} and \code{K2Node_GetDataTableRow} have solid comments and straightforward outcomes
                \item 
            \end{itemize}