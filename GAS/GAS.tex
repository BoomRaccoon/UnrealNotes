\chapter{Gameplay ability system (GAS)}
    \begin{itemize}
        \item UAbilitySystemComponent := is the main thing
        \item FGameplayAbilitySpecContainer := holds its granted GAs
        \item Gameplay Ability := logic of the gameplay mechanic $\rightarrow$ everything can be gameplay ability
        \item Avatar := \glsdesc{GLS-Avatar}
        \item 
        \item 
        \item FActiveGameplayEffectsContainer := holds the currently active GEs
        \item Gameplay Effects: modifies attributes and tags instantly or for certain amount of time; how is it changing
        \item Gameplay AttributeSet := used on the same Actor that the ASC lives on (PlayerState / Character)
        \item Gameplay Attributes := actor properties(health, damage, state) modified by ASC; modified through gameplay effects; composed of \code{BaseValue} \code{CurrentValue}
        \item 
        \item Gameplay cues := for audio/visual effects 
        \item 
        \item GameplayTagManager := handles Tags globally
        \item GameplayTagContainer := is used inside ASCs $\rightarrow$ one on each Actor with an ASC
        \item GameplayTagMap := stores number of instances of GameplayTags
        \item Gameplay Tags := same principle as Actor Tags; handles intercaten between different gameplay- abilities \& effects; good to think about them as conditions
        \item TagMapCount
        \item 
        \item needs to implement \code{UAbilitySystemComponent* GetAbilitySystemComponent() const}
        \item Character:
        \begin{itemize}
            \item includes:
            \begin{itemize}
                \item "AbilitySystemInterface.h"
                \item "GameplayEffectTypes.h"
            \end{itemize}
            \item derives from 'public IAbilitySystemInterface'
        \end{itemize} 
        \item 
    \end{itemize}

    GAS as a whole is basically just a couple of basic UObject classes and an ActorComponent.
    The Component gets put on the actor that can use the ability. Then there is the actual
    GameplayAbility. This is just a UObject that gets instantiated based on settings. Once per
    user, once ever, or once per ability use I think. The component handles most all of the 
    "CanUse" stuff based on tags on the actor, etc. What you allow in the ability class is 
    entirely up to you. How you mutate the ability based on other factors is entirely up to 
    you. The other classes are basically a visual which is GameplayCues, and then there's 
    GameplayEffects which are what modify the attributes. You'd use an Effect to buff strength, 
    or heal, or damage health, etc. You use Gameplay Abilities to apply effects(though you can 
    apply effects from anywhere). So yeah. At it's core it's just a component that runs small 
    UObjects that can handle the lifetime of the ability.


    \section{Random notes}
    \begin{itemize}
        \item AbilitySystemComponent->RegisterGameplayTagEvent(FGameplayTag::RequestGameplayTag(FName("State.Debuff.Stun")), EGameplayTagEventType::NewOrRemoved).AddUObject(this, \&AGDPlayerState::StunTagChanged);
        \item This is from GASDocumentation (one of the pinned messages here and one of my favorite tools I'm learning GAS with)
        \item In this example the PlayerState has the function: virtual void StunTagChanged(const FGameplayTag CallbackTag, int32 NewCount); that's being bound to when the tag is applied

        \item 
    \end{itemize}
    
    \section{Main}
        \begin{itemize}
            \item should be added to the PlayerState (if presistant Attributes) or the character/pawn
            
            \item \code{ActivatableAbilities.Items}
            \item use \code{ABILITYLIST_SCOPE_LOCK();} to iterate over \code{ActivatableAbilities.Items}
            \item \textbf{remove} Abilities only outside of \code{ABILITYLIST_SCOPE_LOCK();} $\rightarrow$ will only remove if no locks in use
            
            \item IAbilitySystemInterface either on \code{PlayerState} or \code{Actor}
        \end{itemize}

    \section{AttributeSet}
        \begin{table}[!htb]
            \begin{tblr}{p{6cm} | p{6cm} | p{6cm}}
                \hline
                    Macro & Function Signature & Description \\
                \hline
                GAMEPLAYATTRIBUTE \_PROPERTY\_GETTER (UMyAttributeSet, Health) &
                static FGameplayAttribute GetHealth() &
                Static function, returns the FGameplayAttribute struct from the engine's reflection system \\
                
                GAMEPLAYATTRIBUTE \_VALUE\_GETTER (Health) &
                float GetHealth() const &
                Returns the current value of the "Health" Gameplay Attribute \\

                GAMEPLAYATTRIBUTE \_VALUE\_SETTER (Health) &
                void SetHealth(float NewVal) &
                Sets the "Health" Gameplay Attribute's value to NewVal \\

                GAMEPLAYATTRIBUTE \_VALUE\_INITTER (Health) &
                void InitHealth(float NewVal) &
                Initializes the "Health" Gameplay Attribute's value to NewVal \\
            \end{tblr}
        \caption{ caption }  
        \end{table}
        

    \section{Make changes to attributes}
        \begin{itemize}
            \item check if ablitySystem and the used  GameplayEffect exist
            \item use:
            \begin{itemize}
                \item \code{FGameplayEffectContextHandle EffectContext = AbilitySystem->MakeEffectContext();} to create a handle
                \item \code{EffectContext.AddSourceObject(this);}
            \end{itemize} 
            \item 
        \end{itemize}
    
    \section{Gameplay Abilities}
        \uline{Gameplay Ability Functions}
        \begin{table}[!htb]
            \begin{tblr}{p{6cm} | p{12cm}}
                \hline
                    Function & Usage \\
                \hline
                    \code{TryActivateAbility} &
                    typical way to execute abilities calls \code{CanActivateAbility} and then \code{CallActivateAbility} \\
                    
                    \code{CanActivateAbility} &
                    lets the caller know if ability is available (eg.: widgets call to indacate status) \\

                    \code{CallActivateAbility} &
                    executes the ability without prior checks \\

                    & \\

                    \code{ActivateAbility} &
                    main code to do things \\

                    \code{CommitAbility} &
                    applies costs to \code{Attribtues} \\

                    \code{CancelAbility} &
                    allows to cancel an ability from anywhere and broadcasts to \code{OnGamepalyAbilityCancelled} (can be prevented with \code{CanBeCanceled}) \\

                    & \\

                    \code{EndAbility} &
                    shuts down the ability. NOT calling this can lead to serious bugs where ability can't be activated anymore or blocks other abilities. \\

                \hline
            \end{tblr}
        \end{table}
    

        \uline{Gameplay Ability Tag Variables}
        \begin{table}[!htb]
            \begin{tblr}{p{6cm} | p{12cm}}
                \hline
                Gameplay Tag Variable(s) & Purpose\\
                \hline
                Cancel Abilities With Tag &
                Cancels any already-executing Ability with Tags matching the list provided while this Ability is executing. \\
                
                Block Abilities With Tag&
                Prevents execution of any other Ability with a matching Tag while this Ability is executing. \\
                
                Activation Owned Tags &
                While this Ability is executing, the owner of the Ability will be granted this set of Tags. \\
                
                Activation Required Tags &
                The Ability can only be activated if the activating Actor or Component has all of these Tags. \\
                
                Activation Blocked Tags &
                The Ability can only be activated if the activating Actor or Component does not have any of these Tags. \\
                
                Target Required Tags &
                The Ability can only be activated if the targeted Actor or Component has all of these Tags. \\
                
                Target Blocked Tags &
                The Ability can only be activated if the targeted Actor or Component does not have any of these Tags. \\
            \end{tblr}
        \end{table}
    
    \section{GameplayCues}
        \begin{itemize}
            \item 
        \end{itemize}
        
        \subsection{GameplayCueParameters}
            \begin{itemize}
                \item has:
                \begin{itemize}
                    \item \code{Instigator} := who owns the ability-system
                    \item \code{SourceObject} := the object that created the effect
                    \item \code{EffectCauser} := physical object that did the damage (auto set to avatarActor)
                \end{itemize}
            \end{itemize}
    
    \section{Code}
        \uline{define player attributes:}
            \begin{lstlisting}
        UPROPERTY(BlueprintReadOnly, Category="Health", ReplicatedUsing = OnRep_Health)
        FGameplayAttributeData Health;
        ATTRIBUTE_ACCESSORS(UWH_PawnAttributes, Health)
            \end{lstlisting}
            \uline{Give ability to players}
            \begin{lstlisting}
        AbilitySystemComponent->GiveAbility(FGameplayAbilitySpec(AbilityClass, AbilityLevel, InputEnum));
            \end{lstlisting}

            For player controlled characters where the ASC lives on the Pawn, I typically initialize on
            the server in the Pawn's PossessedBy() function and initialize on the client in the
            PlayerController's AcknowledgePossession() function.
            \begin{lstlisting}
void APACharacterBase::PossessedBy(AController * NewController)
{
    Super::PossessedBy(NewController);

    if (AbilitySystemComponent)
    {
        AbilitySystemComponent->InitAbilityActorInfo(this, this);
    }

    // ASC MixedMode replication requires that the ASC Owner's Owner be the Controller.
    SetOwner(NewController);
}
        \end{lstlisting}
        
        \begin{lstlisting}
void APAPlayerControllerBase::AcknowledgePossession(APawn* P)
{
    Super::AcknowledgePossession(P);

    APACharacterBase* CharacterBase = Cast<APACharacterBase>(P);
    if (CharacterBase)
    {
        CharacterBase->GetAbilitySystemComponent()->InitAbilityActorInfo(CharacterBase, CharacterBase);
    }

    //...
}
        \end{lstlisting}

        For player controlled characters where the ASC lives on the PlayerState,
        I typically initialize the server in the Pawn's \code{PossessedBy()} (which gets
        called even before \code{BeginPlay()}) function and
        initialize on the client in the Pawn's \code{OnRep_PlayerState()}-function.
        This ensures that the PlayerState exists on the client.
        \begin{lstlisting}
// Server only
void AGDHeroCharacter::PossessedBy(AController * NewController)
{
    Super::PossessedBy(NewController);

    AGDPlayerState* PS = GetPlayerState<AGDPlayerState>();
    if (PS)
    {
        // Set the ASC on the Server. Clients do this in OnRep_PlayerState()
        AbilitySystemComponent = Cast<UGDAbilitySystemComponent>(PS->GetAbilitySystemComponent());

        // AI won't have PlayerControllers so we can init again here just to be sure. No harm in initing twice for heroes that have PlayerControllers.
        PS->GetAbilitySystemComponent()->InitAbilityActorInfo(PS, this);
    }
    
    //...
}        
        \end{lstlisting}

        \begin{lstlisting}
// Client only
void AGDHeroCharacter::OnRep_PlayerState()
{
    Super::OnRep_PlayerState();

    AGDPlayerState* PS = GetPlayerState<AGDPlayerState>();
    if (PS)
    {
        // Set the ASC for clients. Server does this in PossessedBy.
        AbilitySystemComponent = Cast<UGDAbilitySystemComponent>(PS->GetAbilitySystemComponent());

        // Init ASC Actor Info for clients. Server will init its ASC when it possesses a new Actor.
        AbilitySystemComponent->InitAbilityActorInfo(PS, this);
    }

    // ...
}
        \end{lstlisting}


        \uline{Listen for Tag Changes}
        \begin{lstlisting}
    AbilitySystemComponent->RegisterGameplayTagEvent(
    FGameplayTag::RequestGameplayTag(FName("State.Debuff.Stun")),
    EGameplayTagEventType::NewOrRemoved).AddUObject(this, &AGDPlayerState::StunTagChanged);
        \end{lstlisting}
        \begin{lstlisting}
    virtual void StunTagChanged(const FGameplayTag CallbackTag, int32 NewCount);
        \end{lstlisting}

        \uline{Delegate for Attribute changes}
        \begin{lstlisting}
    AbilitySystemComponent->GetGameplayAttributeValueChangeDelegate(
    AttributeSetBase->GetHealthAttribute()).AddUObject(this, &AGDPlayerState::HealthChanged);
        \end{lstlisting}

        \begin{lstlisting}
    virtual void HealthChanged(const FOnAttributeChangeData& Data);
        \end{lstlisting}
