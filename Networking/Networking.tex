\chapter{Networking}
    Networking in UE4 is called 'Replication'

    \section{Basics}
        \begin{itemize}
            \item is Actor-centric
            \item how to create 3 instances and connect them to the server (local debugging) :NEXT LINE:
            \item Editor Preferences $\rightarrow$ type 'Multi' $\rightarrow$ tick 'Enable Network Emulation'
            \item 
            \item replication goes only SERVER $\rightarrow$ CLIENT but CLIENT $\nrightarrow$ SERVER
            \item 
            \item set \colorbox{mygray}{\lstinline{NetUpdateFrequency}} to an appropriate value ex.: 10 = $\frac{1}{5}$ $\rightarrow$ update every 5th of a second (0.2s)
            \item \colorbox{mygray}{\lstinline{net.UseAdaptiveNetUpdateFrequency}} combined with \colorbox{mygray}{\lstinline{MinNetUpdateFrequency}}(minimum update attempts) and \colorbox{mygray}{\lstinline{NetUpdateFrequency}}(maximum update attempts) can improve perfermonce
            \item 
            \item clients must reconnect to the server for some actions, this is called Non-seamless-travel and occurs when:
            \begin{itemize}
                \item loading a map for the first time
                \item connecting to a server for the first time as a client
                \item you want to end a multiplayer game and start a new one
            \end{itemize}
            \item Ping-Time := is Round-Trip-Time (Client $\rightarrow$ Server $\rightarrow$ Client)
            \item delay-based-netcode := is simpler but has less features
            \item rollback-netcode := has more features than delay-based netcode but is more complex
        \end{itemize}

        \includegraphics[width=\textwidth]{DomainExample.png}
        \uline{}
        \begin{itemize}
            \item Server Only: Exist on the server (ex.: AGameMode)
            \item Server \& Client: Exist on the Server and on all Clients (ex.: AGameState, APlayerState, APawn )
            \item Server \& Owning Client: Exist on the Server and the owning Client (ex.: APlayerController)
            \item Owning Client Only: Exist only on the Client (ex.: AHUD, UMG Widgets)
        \end{itemize}
    \begin{itemize}
        \item \uline{Network Modes:} is a property of the world and can have the following values:
        \begin{itemize}
            \item NM\_Standalone := for singleplayer and local-multiplayer
            \item NM\_Client := runs the game without any server-logic
            \item NM\_ListenServer := runs the game and accepts connections
            \item NM\_DedicatedServer := hosts a network multiplayer session without graphics
        \end{itemize}
        \item \uline{Replication-Features:}
        \begin{itemize}
            \item Creation \& Destruction := when an authoritative version of a repicated Actor is spawned on a server, it automatically generates remote proxis
            \item Movement-Replication := bReplicateMovement = true and the Actor will replicate \textit{Location,Rotation,Velocity}
            \item Variable-Replication := any variable that is 
            \item Component-Replication := 
            \item Remote-Procedure-Calls := 
        \end{itemize}
        \item \uline{Network Roles:}
        \begin{itemize}
            \item None := the actor has no role in a network game and does not replicate
            \item Authority := the actor is authoritativ and replicates its information to remote proxies of it on other machines
            \item Simulated Proxy (this are most actors) := the actor is a remote proxy that is controlled entirely by an authoritative actor on another machine
            \item Autonomous Proxy (ex. player character) := the actor is a remote proxy that is capable of performing some functions locally, but receivs corrections from an authoritative actor
        \end{itemize}
    \end{itemize}
\includegraphics[width=\textwidth]{NetworkLayout.jpg}

        \uline{GameState}
        \begin{itemize}
            \item AGameStateBase \& AGameState
            \item replicated to all Clients
            \item most important class for shared information between Server \& Clients
            \item keeps track of the Game State ex.: GameMode contains total KillCount GameState the Kills for every player
            \item 
        \end{itemize}

        \uline{PlayerState}
        \begin{itemize}
            \item most important class for a specific Player
            \item each Player has a PlayerState
            \item access through PlayerArray inside of GameState
            \item 
            \item good for ex.:
            \begin{itemize}
                \item PlayerName
                \item Score
                \item Ping
                \item GuildID
            \end{itemize}
        \end{itemize}

        \uline{Player Controller:}
        \begin{itemize}
            \item exists on the Client \& Server
            \item Clients don't know of each other PlayerControllers
            \item GerPlayerController(0):
            \begin{itemize}
                \item Listen-Server: gets the Listen-Servers PlayerController
                \item Client: gets the Clients PlayerController
                \item Dedicated-Server: gets the first Clients \
            \end{itemize}
        \end{itemize}

        \uline{Userful Classes:}
        \begin{itemize}
            \item AInfo := is an actor used to store information (is an actor in order to use replication)
            \item 
        \end{itemize}

        \uline{RPCs invoked from Server}
        \begin{table}[!htb]
            \begin{tabular}{|p{3cm}|p{3cm}|p{3cm}|p{3cm}|p{3cm}|}
                \hline
                    Actor Ownership & Not replicated & NetMulticast & Server & Client \\
                \hline
                    Client-owned Actor & Runs on Server & Runs on Server and all Clients & Runs on Server & Runs on Actors owning Client \\
                    Server-owned Actor & Runs on Server & Runs on Server and all Clients & Runs on Server & Runs on Server \\
                    Unowned Actor & Runs on Server & Runs on Server and all Clients & Runs on Server & Runs on Server \\
                \hline
            \end{tabular}
        \caption{ caption }  
        \end{table}
        
        \uline{RPCs invoked from Server}
        \begin{table}[!htb]
            \begin{tabular}{|p{3cm}|p{3cm}|p{3cm}|p{3cm}|p{3cm}|}
                \hline
                    Actor Ownership & Not replicated & NetMulticast & Server & Client \\
                \hline
                    Owned by invoking Client & Runs on invoking Client & Runs on invoked Client & Runs on Server & Runs on invoking Client \\
                    Owned by a different Client & Runs on invoking Client & Runs on invoked Client & Dropped & Runs on invoking Client \\
                    Server owned actor & Runs on invoking Client & Runs on invoked Client & Dropped & Runs on invoking Client \\
                    Unowned Actor & Runs on invoking Client & Runs on invoked Client & Dropped & Runs on invoking Client \\
                \hline
            \end{tabular}
        \caption{ caption }  
        \end{table}
        \begin{lstlisting}
    UFUNCTION(Client, unreliable)
    void ClientRPCFunction();

    UFUNCTION(Client, reliable)
    void ReliableClientRPCFunction();

    UFUNCTION(NetMulticast, unreliable)
    void MulticastRPCFunction();

        \end{lstlisting}


        
        \subsection{Network prediction and reconciliation}
        \begin{itemize}
            \item prediction: player's next moves are tried to be predicted
            \item reconciliation: corrects errors that occured during prediction
        \end{itemize}
        
        \subsection{Bandwidth Optimization}
        \begin{itemize}
            \item network relevancy: 
            \item replication conditions: 
        \end{itemize}
        
        
        \subsection{Workflow}
            \begin{itemize}
                \item 
            \end{itemize}


    \section{IRIS}
        \textbf{\uline{Is the new networking plugin}}
        \begin{itemize}
            \item a new API to improve networking with paralization and decoupling gameplay code and network code
        \end{itemize}

        \textbf{\uline{Main parts}}
        \begin{itemize}
            \item \textbf{\uline{Replication Bridge}}
            \begin{itemize}
                \item begins/end replication for actor or Objects
                \item builds descriptors and protocols for replicated data
            \end{itemize}
            \item \textbf{\uline{Net Object}}
            \begin{itemize}
                \item internal representation of an actor/object consisting of
                \item \uline{Replication protocol}
                \item \uline{replication instance protocol}
                \item \uline{buffer to store quantized data}
            \end{itemize}
            \item \textbf{\uline{Replication protocol}}
            \begin{itemize}
                \item per object type and shared between all instances of the same type
                \item contains all replication state descriptors $\rightarrow$ net object can be restored with
            \end{itemize}
            \textbf{\uline{replication fragment}}
            \begin{itemize}
                \item component responsible for carrying replication states back-and-forth between gameplay code and replication system
            \end{itemize}
            \item \textbf{\uline{Net handle}}
            \begin{itemize}
                \item object that API-functions operate on
                \item \uline{handle of} an actor
                \item \uline{handle for} replication system
                \item \uline{generated} by the replication system
                \item \uline{returned} when \code{BeginReplication} is called on an actor
            \end{itemize}
        \end{itemize}


        \textbf{\uline{replication state + replication state descriptor}}
        \begin{itemize}
            \item \textbf{\uline{replication state}}
            \begin{itemize}
                \item bridge between replication system and gameplay code
                \item basic form is a struct containing data to be replicated
            \end{itemize}
            \item 
            \item \textbf{replication state descriptor}
            \begin{itemize}
                \item memory layout
                \item conditionals
                \item filtering
                \item prioritization
                \item serialization
            \end{itemize}
        \end{itemize}


        \subsection{workflow and dataflow}
            \begin{itemize}
                \item \textbf{Registration:}
                \begin{itemize}
                    \item \uline{manually}
                    \item register object with replication system
                    \item declare replication state and replication state descriptor (header properties / explicitly %TODO how do yuo do that?)
                    \item 
                    \item \uline{automatically}
                    \item constructs a replication protocol and replication instance protocol using all replication states defined by the object and its components
                    \item creates a net object corresponding to the newly registered object using the previously constructed replication protocol and replication instance protocol
                    \item assigns the replicated object a unique net handle
                \end{itemize}
            \end{itemize}
            
            \begin{itemize}
                \item \textbf{\uline{Sender}}
                \begin{itemize}
                    \item pre-send-update: polls all replicated objects for state changes if running in legacy mode
                    \item quantizes all dirty state data with the apropriate net serializer
                    \item updates the filtering status and prioritization of all net objects
                    \item 
                    \item send: create and fill packets by ticking all iris data streams
                \end{itemize}
                \item \textbf{\uline{Net-token-data-stream}}: serializes any new tokens with the apropriate net serializer
                \item \textbf{\uline{Replication-data-stream}}: for data read from the replication data stream
                \begin{itemize}
                    \item deserializes and reads the received state data
                    \item instantiates new objects immediately since they are required to build replication protocols
                \end{itemize}
                \item 
            \end{itemize}
            


        \subsection{General notes}
            \begin{itemize}
                \item handle unexpected network behavior with \code{UEngine::OnNetworkFailure}
                
            \end{itemize}


    \section{CPP}
        \subsection{Replicate variables}
            \begin{itemize}
                \item any variable you want to replicate needs \colorbox{mygray}{\lstinline{UPROPERTY(replicated)}} 
                \item and they need to implement the 'GetLifetimeReplicatedProps'-Function
            \end{itemize}
            \begin{lstlisting}
        void ATestPlayerCharacter::GetLifetimeReplicatedProps(TArray<FLifetimeProperty>& OutLifetimeProps) const {
            Super::GetLifetimeReplicatedProps(OutLifetimeProps);

            // Here we list the variables we want to replicate + a condition if wanted
            DOREPLIFETIME(ATestPlayerCharacter, Health, COND_[...]);
        }
            \end{lstlisting}

            \uline{Possible Conditions:}
            \begin{itemize}
                \item COND\_InitialOnly := only attempt to send on the initial bunch
                \item COND\_OwnerOnly := only send to the owner
                \item COND\_SkipOwner := send to every client except the owner
                \item COND\_SimulatedOnly := only send to simulated actors
                \item COND\_AutonomousOnly := only send to autonomous actors
                \item COND\_SimulatedPhysics := send to simulated || bRepPhysics actors
                \item COND\_InitialOrOwner := send to initial packet || actors owner
                \item COND\_Custom := 
            \end{itemize}

            \uline{Replication-Methods}
            \begin{itemize}
                \item replicated := 
                \item ReplicatedUsing=[FunctionName] := will call a function after a value is replicated
            \end{itemize}
            
    \section{The Replicataion System}

        !!! \textbf{not replicated features of Actors, Pawns, Characters} !!!
        \begin{itemize}
            \item Skeletal Mesh and Static Mesh component
            \item Materials
            \item Animation Blueprints
            \item Particle Systems
            \item Sound Emitters
            \item Physics Objects
            \item 
            \item $\rightarrow$ so replicating the variables that drive them $\rightarrow$ every client creates a similar effect
        \end{itemize}



    \section{Blueprints}
        \begin{itemize}
            \item Any BP-class that should replicate $\rightarrow$ set 'Replicates'-Property to true
            \item 
            \item 'IsStandalone'-Node := to check if it's an offline session use 
            \item 
            \item 'HasAuthority'-Node := tells who has authority over the actor \& if it's replicated or not \& mode of replication \url{https://docs.unrealengine.com/4.27/en-US/InteractiveExperiences/Networking/Actors/Roles/}{Official Docs}
            \item $\rightarrow$ ROLE\_SimulatedProxy will update 'AActor' based on it's 'NetUpdateFrequency' and predict ActorMovement etc. between those intervals based on it's velocity
            \item $\rightarrow$ writing custom code for replication can be beneficial
            \item 
        \end{itemize}


    \section{C++}
        \begin{itemize}
            \item you can add specifiers to 'UPROPERTY' in order to enable replication
            \begin{itemize}
                \item 'Replicate' := enables replication
                \item 'ReplicateUsing = FunctionName' := will replicate and execute the specified function on changes
                \item \colorbox{mygray}{\lstinline{GetLifetimeReplicatedProps(TArray <FLifetimeProperty> & OutLifetimeProps) const}}:
                \begin{itemize}
                    \item is the function that is responsible to replicate the properties
                    \item 'Super::GetLifetimeReplicatedProps(OutLifetimeProps);' must be called
                \end{itemize}
            \end{itemize}
            \item 
        \end{itemize}

        \uline{Needed Headers:}
        \begin{itemize}
            \item \colorbox{mygray}{\lstinline{#include "Net/UnrealNetwork.h"}}
            \item 
        \end{itemize}


    \section{Crossplay}
        \begin{itemize}
            \item Game Services: 
            \begin{itemize}
                \item works with any account system
                \item matchmaking
                \item lobbies
                \item peer-to-peer
                \item Voice Chat
                \item 
                \item account linking
                \item player data storage
                \item stats
                \item achievements
                \item leaderboards
                \item inventory
                \item 
                \item Anti-Cheat
                \item Player Reports
                \item Player Sanctions
            \end{itemize}
            \item Account Services: 
        \end{itemize}

    \section{Rollback}
        \begin{itemize}
            \item rollback := is a technique to correct the game state after a network packet arrives
            \item 
            \item 
        \end{itemize}


    \section{Example}
        \begin{itemize}
            \item 90 ping \& 16.6ms (60 FPS) is about 6 frames delay
            \item delay based net-code
            
        \end{itemize}
    

