\chapter{Lights}
    \section{Unsorted}
        \begin{itemize}
            \item Sky Light: emit no photon $\rightarrow$ 1 bounce
            \item Directional, Point and Spot Light emit photons $\rightarrow$ multiple bounces
            \item while building the level turn off 'Auto Exposure' $\rightarrow$ Project Settings $\rightarrow$ search Exposure $\rightarrow$ set to false
        \end{itemize}
    \section{Basics}
        \subsection{Mobility}
            \begin{itemize}
                \item \code{Static}: not  moving or updating in any way; built light; use indirect lighting (indirect ligting sample or volumetric lightmaps)
                \item \code{Stationary}: not moving; No built light (lightmass); uses Cached Shadow Map for Movable Light
                \item \code{Movable}: moved and updated; fully dynamic shadow; LIGHT ACTORS only cast dynamic shadows;
            \end{itemize}
        
        \subsection{IES Light}
            Is a file format describing the light distribution \\

        \subsection{Direct/Indirect}

    \section{Sky Atmosphere}
        \subsection{Setup}
            use extra tool for easy setup \code{Window -> Env. Lightmixer} or the following\\
            \begin{itemize}
                \item Add a directional light
                \item Place Directional Light and set Atmosphere/Fog Sun Light in its properties to true
                \item Add 'SkyAthmosphere'
                \item Add Sky Light
            \end{itemize}

            \underline{SkyAtmosphere Properties:}
            \begin{itemize}
                \item Planet:
                \begin{itemize}
                    \item Ground Radius: Size of the planet (if looked at from high altitude/space)
                \end{itemize}                
                \item \code{Rayleigh Exponential Distribution}: to define the altitude (in kilometers) at which Rayleigh scattering effect is reduced to 40\% due to reduced density
                \item \code{Mie Exponential Distribution}: to define the altitude (in kilometers) at which Mie scattering effect is reduced to 40\% due to reduced density
                \item \code{Atmospheric Height}: to define the height of the atmosphere above which we stop evaluating light interactions with the atmosphere.        
            \end{itemize}
            \includegraphics[width=\textwidth]{PlanetaryView.png}

            \underline{Additional notes}
                \begin{itemize}
                    \item Sun at zenith 120000 Lux
                    \item 150000 Lux on a white surface (measure with AO=off and the luminance meter)
                    \item Multiscattering in the Sky atmosphere = 1
                \end{itemize}

        \subsection{Directional Light}
            Has a 'Cascaded Shadow Map Property' which will display near shadows dynamically and use baked lighting for 
            lights at farther distance.
            \begin{table}[H]
                \begin{tabular}{|p{7cm}|p{12cm}|}
                    \hline
                        Property & Description \\
                    \hline
                        Dynamic Shadow Distance & Distance within which you will see cascading shadow maps \\
                        Num Dynamic Shadow Cascades & More levels $\rightarrow$ better shadow resulation at distance with greater performance cost \\
                        Cascade Distribution Exponent &  \\ % TODO:
                    \hline
                \end{tabular}
            \end{table}  

    \section{Lightmass Global Illumination}
        Lightmass creates Lightmaps for 'Stationary' and 'Static' lights.\\
        You have to surround the static light with the 'Lightmass Volume' \\
        don't forget to set ojects mobility static \\

    \section{Screen Space Global Illumination (SSGI)}
        Improves the shadows and directional light. Quality Level 3 for the best balance between performance and quality. \\
        \underline{In order to enable:} \\
        'Project Settings' $\rightarrow$ 'Engine' $\rightarrow$ 'Rendering' $\rightarrow$ under the Lighting category.  \\
        \begin{lstlisting}
    r.SSGI.Quality
        \end{lstlisting}
        \underline{render in half resolution}
        \begin{lstlisting}
    r.SSGI.HalfRes
        \end{lstlisting}

    \section{Lumen}
        \begin{itemize}
            \item has color bleeding
            \item soft shadows
        \end{itemize}

        \begin{itemize}
            \item directional light
            \item sky light
            \item sky atmosphere
        \end{itemize}