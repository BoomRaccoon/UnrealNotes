\chapter{Animations}

    \section{Unsorted}
        \begin{itemize}
            \item ANimation Curve := 
        \end{itemize}


    \section{Basic Setup}
        \begin{itemize}
            \item Create 'Animation Blueprint' for the character in question with the appropriate skeleton
            \item Create a blendspace with the character skeleton mesh
            \item create an animation graph
        \end{itemize}


    \section{Basic Assets}
        \subsection{Skeleton Asset}
            \begin{itemize}
                \item equivalent to the blender armature
                \item you can attach skeletal meshes to them
                \item skeletons can be used by multiple meshes
                \item rules:
                \begin{itemize}
                    \item keep order and names
                    \item you can append any number of bones
                    \item you can add any number of bones
                \end{itemize}
            \end{itemize}
            \href{https://docs.unrealengine.com/4.27/en-US/AnimatingObjects/SkeletalMeshAnimation/Skeleton/}{Official Doc}

        \subsection{Animations Sequence}
            \begin{figure}
                \includegraphics[width=\textwidth]{Animation_Sequence.jpg}
                \caption{How the Editor-Window looks like for Animation-Sequences}
            \end{figure}
            \begin{itemize}
                \item are the basic animations that are exported from blender and imported in UE5
                \item Each Animation Sequence asset targets a specific Skeleton and can only be played on that Skeleton.
                \item 'Notify' := will fire events at specific points during the 'Animation Sequence'
                \item there are different types of Notifies:
                \begin{itemize}
                    \item Skeletal Notify := is a generic notify and can be accessed inside of Animation Blueprints and their State Machines E.g.: foot steps, particle effects
                    \item Cloth Simulation Notifies := can change the cloth simulation property to play, pause, resume and reset 
                    \item Play Particle Effect := will play a particle effect and it's not available in Animation Blueprints but the details panel gives some options
                    \item Play Sound := will play a sound and it's not available in Animatin Blueprints but the details panel gives some options
                    \item Reset Dynamics := will reset any AnimDynamics
                \end{itemize}
                \item 'Notify State' := 
                \item 'Sync Marker' :=
                \item 
                \item Can be additive := so they can influence other animations inside your 'Animation Blueprint'
                \item in the 'Animation Sequence' under the section 'Additive Settings' :
                \begin{itemize}
                    \item No additive := 
                    \item Local Space := 
                    \item Mesh Space := 
                \end{itemize}
            \end{itemize}
            \href{https://docs.unrealengine.com/4.27/en-US/AnimatingObjects/SkeletalMeshAnimation/Sequences/}{Offizial Doc}

        \subsection{Animation Curves}
            \begin{itemize}
                \item 
            \end{itemize}

        \subsection{Animation Blueprints}
            \begin{figure}[ht]
                \includegraphics[]{Animation_Blueprint.jpg}
                \caption{Structure of a Animation Blueprint}
            \end{figure}
            \begin{itemize}
                \item Controls animation of ONE skeletal Mesh
                \item Has an 'Anim Graph' \& a 'Event Graph'
                \item Anim Graph := with state machine which is used for constant animations like idle, walking, running ...
                \item Event Graph := 
                \begin{itemize}
                    \item Used to update variables
                    \item 
                    \item 'Event Blueprint Initialize Animation' used for the setup
                    \item 'Event Blueprint Update Animation' is the equivalent to the 'Tick' in normal BPs
                \end{itemize}
                \item You will be able to define the exact behaviour inside the 'Animation Graph' with the help of a state machine
                \item In order for the state machine to transition between states you need to feed it with the needed information
                \item This information gathering will be setup inside of the 'Event Graph' of the 'Animation Blueprint'
            \end{itemize}

            \begin{itemize}
                \item B(lend)S(paces) in the state machine will have input nodes
                \item 'Layered blend per bone'
                \item Base Pose: Animation 
                \item Blend Poses: 
            \end{itemize}
    
        \subsection{Animation Montage}
            \begin{itemize}
                \item 
                \item 
                \item enables you to combine animation sequences and add logic
                \item exposing animation controls to blueprints or code
                \item 
                \item add logic to animations (ex. reloading)
            \end{itemize}

        \subsection{Blendspace}
        \begin{figure}
            \includegraphics[width=\textwidth]{Blend_Space.jpg}
            \caption{How the Editor-Window looks like for 'Blend Spaces'}
        \end{figure}
        \begin{itemize}
            \item Blendspace := takes in input and blends between different animations depending on the input values
            \item the input values are inputs on nodes you add to the Animation Blueprint
            \item 1D-Blendspaces := take in only 1 input
            \item 2D-Blendspaces := take in 2 inputs
        \end{itemize}


    \section{Retargeting}
        \begin{minipage}{0.9\textwidth}
            \begingroup \parfillskip=0pt
                \begin{minipage}[t]{0.49\textwidth}
                    \includegraphics[width=\textwidth]{ShowAllBones.png}

                    \label{}
                \end{minipage}
                \begin{minipage}[t]{0.49\textwidth}
                    \includegraphics[width=\textwidth]{NonRetargetedAnimation.png}

                    \label{}
                \end{minipage}
            \par\endgroup
        \end{minipage}
        
        \begin{itemize}
            \item the skeleton saves the bones and their translation(location) data
            \item in order to fix that $\rightarrow$ retarget the animation/animation-blueprint to another skeleton
        \end{itemize}
        \begin{figure}
            \includegraphics[width=\textwidth]{PreRetargeting.jpg}
            \caption{}
        \end{figure}
        \begin{figure}
            \includegraphics[width=\textwidth]{PostRetargeting.jpg}
            \caption{}
        \end{figure}
        \uline{Workflow:}
        \begin{itemize}
            \item Select animation you want to retarget
            \item RC $\rightarrow$ 'Retarget Anim Asset'
            \item Source Skeleton will be selected automatically
            \item Search for the Skeleton you want to retarget to ('Target')
            \item the location of the bones will be adjusted and a new 'Anim Asset' will be created
            \item the rotation does NOT get adjusted $\rightarrow$ the 'Base Pose' has to be adjusted in the 'Retarget Manager'
            \item if the skeletons have differences you might have to change the bone mappings e.g.: greystone
        \end{itemize}
        \uline{some settings to set before retargeting}
        \begin{itemize}
            \item in the skeleton you can set the type of \glqq TranslationRetargeting\grqq
            \begin{itemize}
                \item Animation := bone translation comes from the animation data, unchanged
                \item Skeleton := bone translation comes from the target skeleton's bind pose
                \item Animation scaled := comes from the animation data but is scaled by the skeleton proportions
                \item Animation relative
            \end{itemize}
        \end{itemize}

    
    \uline{copy pose from mesh}  
    \begin{itemize}
        \item feeds in skeletal information from one skeleton to another
        \item takes in source mesh component
        \item 'use attached parent' to get the parent in the ABP
        \item skeletons don't have to match
        \item name and hierarchy match
    \end{itemize}
	
    \uline{retarget pose from mesh}
    \begin{itemize}
        \item uses an IK-Retargetter
        \item -> check compatability by opening the retargetter and set own character as preview
        \item -> check animations
        \item 'use attached parent'
    \end{itemize}
    
    \subsection{IK-Retargetter}
        \begin{itemize}
            \item made with IK-Rigs for easier retargeting
            \item 
        \end{itemize}
        \subsubsection{Notes from experience (solutions to problems) }
            \begin{itemize}
                \item finger retargeting is bad $\rightarrow$ disable \code{FK} retargeting on the metacarpals
                \item 
            \end{itemize}


    \section{Skeleton LODs}
        

    \section{Addative Animation}
        \begin{itemize}
            \item animation is added on top of another
            \item examples
            \begin{itemize}
                \item breathing
                \item compression after jump
                \item leaning forwards while running
            \end{itemize}
        \end{itemize}
        
        \subsection{Nodes}

        
        \href{https://www.youtube.com/watch?v=flHL3qJB3_I}{YT explenation}

    \section{ALS}
        \subsection{Custom character}
            \begin{itemize}
                \item create an IK-Rig for the character
                \item create a retargetter depending for the custom character
                \item create ABP for your character with \code{Retarget Pose from Mesh} and set
                \begin{itemize}
                    \item \code{Use attached Parent} to \code{true}
                    \item \code{IKRetargeter Asset} to the one you created
                \end{itemize}
                \item add a \glqq Sekeletal Mesh component\grqq for your custom character
                \item set
                \begin{itemize}
                    \item \code{Hidden in Game} to \code{true}
                    \item under Optimization set \code{Visibility based anim tick option} to \code{Always Tick Pose and Refresh Bones}
                \end{itemize}
                
            \end{itemize}

    \section{Blend Poses depending on a Boolean value}
        \href{https://www.youtube.com/watch?v=TiQjjcvM7Eo}{YT explenation}
    

    \section{AnimDynamics}
        \begin{itemize}
            \item is a node inside the AnimGraph
            \item can be used to add physically based secondary animations for pouches, necklaces, backpacks ...
            \item detailed video under \href{https://www.youtube.com/watch?v=5h5CvZEBBWo}{https://www.youtube.com/watch?v=5h5CvZEBBWo}
        \end{itemize}

    \section{Animation Modifier}
        Enable users to define a sequence of actions for a given animation sequence or skeleton which enables Animation Sync Markers

    \section{State Machine vs Animation Montage}
        Something that happens upon a button press (attack) \\
        Also interruptable animations are placed here. You will specify a specific frame on which the animation can
        branch out playing different animations depending on some condition \\
        Slots 

        Nodes for the CharacterBP:
        \begin{itemize}
            \item Play Anim Montage (simple)
            \item Play Montage (way more complex)
            \item Layered blend per bone
            \item blend poses by bool
        \end{itemize}
    \smallskip


    \section{Blending Animations with Blendspace \& Animation-Graph}
        \uline{Blendspaces}

    \smallskip
        \uline{Animationsgraph}
        \begin{itemize}
            \item AnimationsGraphs have Animation-States between which you can switch depending on conditions
            \item 
        \end{itemize}

    \section{Add muscle progress or Head/Body Rotation}
        \begin{itemize}
            \item the 'Transform Bone'-Node in the 'AnimGraph' can manipulate specific bones
            \item e.g.: rotate the head depending on cursor location || scale a body part depending on muscle mass ...
        \end{itemize}

    \section{Example LeftRightDab}
        \begin{figure}
            \includegraphics[width=\textwidth]{LeftRightDab.jpg}
            \caption{Example setup for leaning where bool values are set in the character BP and used in the AnimBP}
        \end{figure}

    \section{Clothing}
        \href{https://docs.unrealengine.com/4.27/en-US/InteractiveExperiences/Physics/Cloth/Overview/}{Official Doc}

    \section{Veretex-Animation}
        \begin{itemize}
            \item create 'Animation Sequences'
            \item combine all animations into a 'Animation Composite'
            \item use maya with 'vertex animation tools' to create vertex animation textures %TODO: Blender?
            \item 
        \end{itemize}

    \section{Skeleton LODs}
         \includegraphics[width=\textwidth]{Skeleton_Editor.jpg}
        how to add LODs for the skeleton:
        \begin{itemize}
            \item go into the open the 'Skeleton Editor' of the skeleton
            \item go into the 'Mesh' section
            \item specify the bones you want to remove for each LOD in 'Asset Details'
        \end{itemize}
        \includegraphics[width=\textwidth]{Asset_Details.jpg}
        \href{https://www.youtube.com/watch?v=ti8NopRIgFs}{YT ref}



    \section{Motion Warping}
            \begin{itemize}
            \item Motion Warping := character animations will be adjusted depending on prerequisites (move to certain location before playing the next section of an animation)
            \item \code{RootMotion} must be enabled on the Animation
            \item 
            \item Setup:
            \begin{itemize}
                \item Create Notifies 'Motion Warping' == Motion Windows (1 for rotation and 1 for translation)
                \item before you play the animation montage, use 'Add or Update Sync Point'-Nodes for every 'Motion Warping'-Notify
                \item use 'Make Motion Warping Sync Point'-Node to specify where it should warp to
                \item 'Warp Point Anim Provider' will be in the next UE version
            \end{itemize}
        \end{itemize}

    \section{ControlRig}
        \begin{figure}
            \includegraphics[width=\textwidth]{EditorLayout.png}
            \caption{Recommended layout (used by a lot of epic employees)}
            \label{}
        \end{figure}

        

        \textbf{\uline{basic events in BP}}        
        \begin{itemize}
            \item Construction: where we can create the constrols and extra bones
            \item Forwards-Solve: where the logic for Ctrl $\rightarrow$ Bone is setup
            \item Backwards-Solve: where 
            \item Interaction-Event: any time the controls als moved
        \end{itemize}
        \textbf{\uline{some concepts}}
        \begin{itemize}
            \item Indirect-Controls / Proxy-Control: drives multiple things
            \item dynamic hierarchy: where controls can
            \item 
        \end{itemize}
        
        \begin{figure}
            \includegraphics[width=\textwidth]{Animation_UsefulOptions.png}
            \caption{Some useful options for the viewport when in AnimationMode}
            \label{}
        \end{figure}

        \textbf{\uline{Useful shortcuts}}
        \begin{itemize}
            \item Ctrl + G: reset transform on selected
            \item Ctrl + Shift + G: Reset all control transform
            \item 1 Create Get Transform Node
            \item 2 Create Set Transform Node
        \end{itemize}

        \subsection{Full-Body IK}
            \begin{itemize}
                \item create a ControlRig IK
                \item add the 'Full Body IK'-Node and set the 'Root' to the first skinned bone (most cases pelvis)
                \item Add effectors = $\rightarrow$ RC the bones $\rightarrow$ 'New' $\rightarrow$ 'New Control'
                \item Unparent the effectors
                \item Drag effectors into the 'Rig Graph' $\rightarrow$ connect the 'Transform' to 'Full Body IK'-Node
                \item set bones
                \item add the constraints under 'Bone Settings'
                \item 
                \item Bottom 'Full Body IK'-Settings:
                \begin{itemize}
                    \item 'Iterations': rel. cheap (100 not that bad)
                    \item 'Mass Multiplier': bones rotate depending on their length $\rightarrow$ higher Mass-value = less rotation
                    \item 
                \end{itemize}
            \end{itemize}

    \section{Anim Link Layers}
        \begin{itemize}
            \item is a way to splilt huge anim\_BPs into smaller ones and link them together
            \item 
            \item are instances of animation blueprints
            \item hold functions for every pose
        \end{itemize}

        \uline{SETUP}
        \begin{itemize}
            \item Create a seperate ABP
            \item copy\&paste the state machines you want to split from the main graph
            \item add \code{Linked Anim Graph}-Node to your main graph
            \item select the ABP you created
            \item assign variables that you update in the main ABP to the ones used in the linked graph
            \item 
            \item if the additional ABP needs the output of another ABP use \code{Input Pose}-Node in the using graph
            \item $\hookrightarrow$ adds an input pin when it's used in a \code{Linked Anim Graph}-Node
        \end{itemize}
            \includegraphics[width=\textwidth]{Bilder/LinkedGraphDetails.png}

        \uline{ANIMATION LAYER INTERFACE}
        \begin{itemize}
            \item used for dynamic linked graphs
        \end{itemize}

        \uline{SETUP}
        \begin{itemize}
            \item create an \code{Animtion Layer Interface}
            \item add \code{Animation Layers}
            \item 
            \item create a default implementation in the main graph that will use the linked layer
            \item eg.: use default logic as input\&result
            \item 
            \item create ABPs that implement that interface
            \item implement the actual Layer (double click the animation layer in the \code{My Blueprint}-tab)
            \item 
            \item Add the specific \code{Linked Anim Layer}-Node where you want to use it
            \item 
            \item Set the linked graph from your characters Mesh-Component
            \item 
            \item unlink the anim
        \end{itemize}
            \includegraphics[width=\textwidth]{Bilder/LinkedLayerInterface.png}


    \section{Usefull commands}
        \code{showflag.bones 1}


    \section{Motion Matching}

        \subsection{Basic Concepts}
            \begin{enumerate}
                \item \code{Pose History} := stores the last X poses of the character
                \item \code{Pose-Search-Database} := contains a list of animation assets and a schema
                \item \code{Pose-Search-Schema} := describes what features/properties you care about (eg.: trajectory, position, velocity)
                \item \code{Chooser-Table} := specifies what motion matching can currently use; uses many variables to reduce number of possible animations
                \item \code{Interrupt Mode} := can be specified to either interrupt and change Pose-Search-Database or keep the currently running animation a little longer
                \item 
            \end{enumerate}


        \subsection{Pose Search Database}
            \begin{enumerate}
                \item every channel has a weight, the higher the weight the more important the channel is
            \end{enumerate}
        


        \subsection{}
            \begin{enumerate}
                \item each movement should have it's own pose search database $\rightarrow$ gives more precise pose selection
                \item otherwise the chooser might stay longer in the wrong pose
            \end{enumerate}


        \subsection{Nodes}
            \begin{table}[!htb]
                \begin{tblr}{p{6cm} | p{12cm}}
                    \hline
                        Node & Description \\
                    \hline
                        Pose History & \makecell[l]{collects information for the specified bones, trajectories ...} \\
                        Motion Matching & \makecell[l]{takes input (pose search database) \\ finds the best matching pose from the database and returns it} \\
                        Evaluate Chooser & \makecell[l]{takes the output of the Motion Matching \\ and decides which animation to play} \\
                    \hline
                \end{tblr}
            \caption{ caption }  
            \end{table}

        \subsection{Setup}

        \begin{itemize}
            \item first add a \code{Motion Matching}-Node to the AnimGraph
            \item hook it up to the \code{Pose Histroy}-Node
            \item 
            \item create a \code{Motion Matching Data Asset}
            \item add the animations you want to use
            \item 
            \item create a code{Space-Search-Schema}-Asset := describes what features/properties you care about
            \item first select the skeleton it should be used on
            \item add \code{Channels}
            \item the \code{Samples}
            \begin{itemize}
                \item \code{Offset} := character-location on the trajectory at X seconds (negative := X seconds ago; positive := X seconds in the future)
                \item \code{Weight} := how important is the sample
            \end{itemize}
        \end{itemize}

        pose-search-feature-channel
        \begin{itemize}
            \item those have functions you can override
            \item 
        \end{itemize}

        \subsection{Debugging}
            tools:
            \begin{itemize}
                \item \code{Game Animation}
                \item \code{Rewind Debugger}
            \end{itemize}

            \begin{itemize}
                \item \code{ShowFlag.MotionMatching 1}
                \item \code{ShowFlag.MotionMatching 2}
            \end{itemize}

            \subsubsection{Rewind Debugger}
                \begin{itemize}
                    \item \code{Cost} := 
                    \item 
                \end{itemize}