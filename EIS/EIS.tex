\chapter{Enhanced Input}

\section{Genneral Input Stack}
    \includegraphics[width=\textwidth]{Bilder/InputStack.jpg}
    \begin{itemize}
        \item ViewportWidget
        \item GameViewportClient
    \end{itemize}

    \begin{itemize}
        \item InputRouter := used to specify a specific order for input
        \item PlayerController::BuildInputStack := 
    \end{itemize}

    \subsection{InputEvent}
     \includegraphics[width=\textwidth]{Bilder/InputEvents.jpg}
      \includegraphics[width=\textwidth]{Bilder/NavigationSource.jpg}

\section{How to enable}
    \begin{itemize}
        \item Enable the 'Enhanced Input'-Plugin
        \item 'Project Settings' $\rightarrow$ 'Input' $\rightarrow$ 'Default Classes' $\rightarrow$ Change to 'EnhancedPlayerInput' \& 'EnhancedInputComponent'
        \item $\uparrow$ can also be done manually by adding following to \code{Config/DefaultInput.ini}
        \begin{lstlisting}
[/Script/Engine.InputSettings]
DefaultPlayerInputClass=/Script/EnhancedInput.EnhancedPlayerInput
DefaultInputComponentClass=/Script/EnhancedInput.EnhancedInputComponent                
        \end{lstlisting}
    \end{itemize}

\section{Main Parts}
    \begin{itemize}
        \item \code{UInputActions}:
        \begin{itemize}
            \item the main communication link between the Enhanced Input system and the project code
            \item can be anything that an interactive character might do like jump, open a door or a state like holding a button
            \item doesn't care where the raw input comes from but \code{has a state}
            \item eg.: pick up item action is either on or off but walking needs direction and speed
            \item property-types are:
            \begin{itemize}
                \item 
            \end{itemize}
        \end{itemize}
        \item \code{UInputMappingContext}
        \begin{itemize}
            \item creates the link between \code{Input} and \code{Action}
            \item 
            \item specify conditions to change the key behaviour for different situations
        \end{itemize}
        \item Modifiers (inside Actions and MappingContexts)
        \begin{itemize}
            \item can adjust the raw input coming from the users device
        \end{itemize}
        \item Triggers (inside Actions and MappingContexts)
        \begin{itemize}
            \item applied after the raw-input was modified by the Modifiers
            \item can use output of other Input-Actions
            \item eg.: One Input-Action must be active to trigger another
        \end{itemize}
    \end{itemize}

\section{Using it}
    \begin{itemize}
        \item 1. create \code{Input-Actions}
        \item 2. create \code{Input-Mapping-Context}
        \item 3. Assign \code{Inputs} to \code{Actions} inside the Input-Mapping-Context
        \item 4. add Input-Mapping-Context to \code{LocalPlayer}'s \code{Enhanced-Input-Local-Player-Subsystem}
    \end{itemize}

    \begin{lstlisting}
// There are ways to bind a UInputAction* to a handler function and multiple types of ETriggerEvent that may be of interest.

// This calls the handler function on the tick when MyInputAction starts, such as when pressing an action button.
if (MyInputAction)
{
PlayerEnhancedInputComponent->BindAction(MyInputAction, ETriggerEvent::Started, this, &AMyPawn::MyInputHandlerFunction);
}

// This calls the handler function (a UFUNCTION) by name on every tick while the input conditions are met, such as when holding a movement key down.
if (MyOtherInputAction)
{
PlayerEnhancedInputComponent->BindAction(MyOtherInputAction, ETriggerEvent::Triggered, this, TEXT("MyOtherInputHandlerFunction"));
}
    \end{lstlisting}

    \uline{Signatures for input handlers}
    \begin{itemize}
        \item \code{void ... ()}
        \item \code{void ... (const FInputActionValue& ActionValue)} := simplest that only tells that input was made
        \item \code{void ... (const FInputActinoInstance& Action Instance)} := provides access to the current value of the input
        \item \code{void ... (FInputActionValue ActionValue, float ElapsedTime, float TriggeredTime)} := when dynamically binding to a UFunction by its name
    \end{itemize}

    \begin{itemize}
        \item unreal creates a UPlayer for each player:
        \begin{itemize}
            \item ULocalPlayer := 
            \item UNetConnection := 
        \end{itemize}
        \item the connection between the UPlayer and the pawns is provided with the playercontroller
    \end{itemize}

    \uline{needed Headers inside PlayerController:}
    \begin{itemize}
        \item \code{#include "InputMappingContext.h"}
        \item \code{#include "InputAction.h"}
        \item \code{#include "InputModifiers.h"}
    \end{itemize}
        
    \uline{needed Headers inside Pawn/Character:}
    \begin{itemize}
        \item \code{#include "MyPlayerController.h"}
        \item \code{#include "EnhancedInputComponent.h"}
        \item \code{#include "EnhancedInputSubsystems.h"}
    \end{itemize}

    \uline{actual coding needed inside PlayerController}
    \begin{itemize}
        \item Ceate Mapping Context(s)
        \item Create Input Actions
        \item Assign ValueTypes to the Actions
        \item Optional: Add modifiers (eg.: invert, clamp, multiply)
        \item Now the Actions can be mapped to a Key of a Mapping Context
    \end{itemize}


    \subsection{Autorun}
        \begin{itemize}
            \item 
        \end{itemize}

        \begin{figure}[H]
            \includegraphics[width=\textwidth]{AutoRunEIS.png}
            \caption{Implementing AutoRun}
            \label{}
        \end{figure}


\section{Saving}

    \subsection{Overview}
        \begin{itemize}
            \item \code{EnhancedInputUserSettings} := is the class that holds the settings for the Enhanced Input system
            \item 
            \item PlayerMappableKeySettings :=
            \begin{itemize}
                \item Holds setting information of an Action Input or a Action Key Mapping for setting screen and save purposes
                \item can be set for Input Actions OR overriden in the IMC (Setting Behavior)
            \end{itemize} 
        \end{itemize}
    
        



    \begin{itemize}
        \item 
    \end{itemize}

\newpage
