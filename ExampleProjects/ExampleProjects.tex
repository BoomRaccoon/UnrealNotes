\chapter{Notes on EPIC's example projects}
    \section{Lyra}
        \subsection{Main Parts}
            \begin{itemize}
                \item \code{Experience}:
                \begin{itemize}
                    \item in essence it's a replacement for GameModes
                    \item you define DataAssets
                    \item this DataAssets specify the specifics for your GameMode
                    \item EXAMPLES in Lyra:
                    \begin{itemize}
                        \item Front End Menu
                        \item Shooter Game: Elimination
                        \item Shooter Game: Control
                        \item Top Down Arena : Splode
                    \end{itemize}
                \end{itemize}
                \item \code{Actions}:
                \begin{itemize}
                    \item contained in Game Features and Experiences
                    \item EXAMPLES:
                    \begin{itemize}
                        \item Add Abilities
                        \item Add Components
                        \item Add Widgets
                    \end{itemize}
                \end{itemize}
                \item \code{Pawn Data}:
                \begin{itemize}
                    \item describes what is used for Pawns controlled by Players/Bots
                    \item in detail:
                    \begin{itemize}
                        \item Player Pawn Class : which pawn is used (what do we see?)
                        \item Default Camera Mode : What camera mode is used
                        \item Abilities : what initial abilities should be granted
                        \item Ability relationship mapping : 
                        \item Input configuration : 
                    \end{itemize}
                    \item 
                    \item Default Camera Mode
                \end{itemize}
                \item \code{Ability Set}:
                \begin{itemize}
                    \item \code{Data Asset}-container for \code{Abilities} \code{Gameplay Effects} and \code{Attribtues}
                \end{itemize}
            \end{itemize}

        \subsection{Experience}
            \begin{itemize}
                \item Holds:
                \begin{itemize}
                    \item Pawn Data
                    \item Actions
                \end{itemize}
            \end{itemize}

        \subsection{Actions}
            \begin{itemize}
                \item 
            \end{itemize}

        \subsection{Input}
            \begin{figure}
                \includegraphics[width=\textwidth]{LyraInputOverview.png}
                \caption{These are the parts used for the input configuration}
            \end{figure}
            \begin{itemize}
                \item \code{Enhanced Input System} is used
                \item is driven with \code{Hero Data} and \code{Actions}
                \item description of the parts (see image):
                \begin{itemize}
                    \item \code{Input Actions} : represents the logical actions the player can input
                    \item \code{Mapping Context} : holds the keybinds for the Input Actions
                    \item \code{Player Mappable Input Configs} : describes how the Mapping Context will show up in the options menu (description)
                    \item \code{Input Config} : Binds an \code{Action} to an \code{Input Tag}
                    \item \code{Input Tag} : is the binding piece to that links the input to the an Ability Set and thus activates Abilities
                    \item \code{}
                \end{itemize}
            \end{itemize}
            
            \subsubsection{Workflow to add new Actions}
                \begin{itemize}
                    \item Create new \code{Action}
                    \item Bind the Action inside the \code{Mapping Context}
                    \item define what \code{Tag} will be triggered by the Action inside the \code{Input Config}
                    \item associate the Tag with an \code{Ability Set} and a specific \code{Ability}
                    \item 
                \end{itemize}


        \subsection{GameFeatures}
            \begin{itemize}
                \item the manually registered \code{HandlePawnExtension()}-function is registered as a manual extension handler
                \item responds to different extensino events
                \item \code{UGameFeatureAction_AddInputBinding}
                \item \code{NAME_ExtensionRemoved} \& \code{NAME_ExtensionAdded} are called when the extension handler is first added or removed for all actors
                \item responds to game-specific \code{NAME_BindInputsNow} event that is emitted by the \code{LyraHeroComponent}
                \item 
                \item \code{GetFeatureName} returns all feature names on the object
                \item 
                \item all Actors implement an \code{InitState}
                \item used for later operations/querying
                \item \code{GameFrameworkStateInterface} := simple framework for a simple C++ state machine
                \item 
            \end{itemize}

    \section{Game Animation Sample (Motion Matching)}

        \begin{itemize}
            \item character blueprint
            \begin{itemize}
                \item controls how the capsule moves around the level
                \item has the traversal logic
            \end{itemize}
            
            \item ABP
            \begin{itemize}
                \item 
            \end{itemize}
        \end{itemize}


        \begin{itemize}
            \item Future Velocity %TODO: what is this?
            \item 
        \end{itemize}

        \begin{itemize}
            \item have gait
            \item character speed is mapped with a curve map
            \item turn in place in ABP
            \item movere will be in the future as the movement component
        \end{itemize}


        \begin{itemize}
            \item multiple, nested chooser-tables that are used to select the correct animation
            \item last chooser-table is the one that selects the animation based on the return value of functions defined in the ABP
            \item \code{Continuing-Pose} := is a negative cost bias that we apply to the currently playing animation; can stop the chooser to go into the next state $\rightarrow$ InterruptMode will allow to actually change Databases
        \end{itemize}


        \subsection{Try Traversal (CBP)}
            \begin{itemize}
                \item depends on spline points placed on the obstacles $\rightarrow$ isn't fully dynmamic
                \item seperate chooser- and motion-matching-node to select between different databases for hurdle, vault, mantle
                \item depends on object-height, object-depth, distance, speed
                \item PoseSearchBlueprintResult := gives the animation with arguments for play rate, looping ...
            \end{itemize}
            