\chapter{Niagara}
    \section{VFX-Theory}
        \subsection{Effect Types}
            \begin{itemize}
                \item Paricle System: using only particles 
                \item Mesh: uses meshes as particles for the spawning system
                \item Flipbook/SpriteSheets/Atlases: multiple images in one texture (so you can iterate over them)
                \item Shader: ...
                \item Hybrids: use multiple of the above
            \end{itemize}
        \subsection{VFX Principles}
            \underline{How to convey information}
            \begin{itemize}
                \item Gameplay: effect purpose
                \item Timing:
                \begin{itemize}
                    \item Anticipation: build-up (smooth transition) ex. high damage longer build-up
                    \item Climax:
                    \item Dissipation/Fade-Out:
                \end{itemize}
                \item Shape:
                \item Contrast: to create a focal point (ex. brightest most important)
                \item Color:
            \end{itemize}
            \includegraphics[width=\textwidth]{VFX_Principles.png}
    \section{Niagara Basics}
            \uline{Niagara Components:}
            \begin{itemize}
                \item Niagara-System: contains multiple Niagara-Emitters to create a complex effect
                \item Niagara-Emitter: is an effect that spawns particles
                \item Niagara-Node: one part of an emitter (displayed in the graph)
                \item Niagara-Module: 
                \item Niagara-Module-Script: are the parts that make up the module (functions)
            \end{itemize}
            \uline{Niagara-Modules:}
            \begin{itemize}
                \item add some logic
                \item 
                \item you can create Custom-Modules
            \end{itemize}
            \uline{How to Create a Niagara Emitter:}
            \begin{enumerate}
                \item Create a Niagara System
                \item Add Emitters to the System
                \item Add Modules - that specify the behaviour of the emitter - to the Emitter
                \item Bounds: can be calculated automatically by going to the 'Bounds' dropdown and selecting 'Set fixed bounds' (Emitters)
            \end{enumerate}

        \subsection{GPU vs CPU}
                \begin{itemize}
                    \item GPU can't calculate bounds (Niagara-System might get culled even though it's visible but bounds might be too small)
                    \item $\hookrightarrow$ resize the bounds := above viewport \code{Bounds} $\rightarrow$ \code{Dropdown-menu} $\rightarrow$ \code{Set fixed bounds}
                \end{itemize}
            
            Modules $\rightarrow$ Emitter $\rightarrow$ System \\
            There is a SystemEmitter from which other Emitters can inherit values \\
            'Effect Types' stores advanced settings \\
            'Max time without render' tells how long to look away until the emitter should be removed \\

        \subsection{User Parameters}
            \underline{Parameter-Space in Emitters}
            \begin{itemize}
                \item Emitter: exists in the 'Emitter' namespace
                \item Transient: exists in 'Niagara System' namespace
            \end{itemize}
\bigskip
\bigskip
            \underline{Parameter-Space in Niagara-Systems}
            \begin{itemize}
                \item User: initialized per system;  
                \item 
            \end{itemize}


            In order to use the Value you have to add a 'Set Parameter' Module in the 'Emitter Spawn'



            \begin{itemize}
                \item Every value can be edited through the 'Details Panel' if you 'Make' it a 'Read from new User Parameter'
                \item the DEFAULT value can be set through the 'Niagra Overview Node' of the emitter system
            \end{itemize}

            \underline{Inside of an Emitter}
            \begin{itemize}
                \item Every value can be edited through the 'Details Panel' if you 'Make' it a 
            \end{itemize}
            

\smallskip
            \underline{Parameters and Parameter Types:}
            \begin{itemize}
                \item Primitive: numeric data with varying precision and channel widths
                \item Enum: define fixed named values
                \item Struct: combine Primitives and Enums
                \item Data Interface: defines functinos that provide data from external sources like other UE4 parts or outside applications
            \end{itemize}
\smallskip
            \begin{itemize}
                \item Niagara operates as a stack with modules, executed from top to bottom
                \item Every module is assign to a group to determine when to be executed
                \item 'System Spawn' 'System Update' are executed first handling the shared behaviour between modules ( TODO: add examples of shared behaviour)
                \item Then modules and items from the emitter groups 'Emitter Spawn' and 'Emitter Update'
                \item Parameters in the particle groups execute for each unique particle
                \item finally renderer group items which describe how to render each particle of a emitter
            \end{itemize}                      
\smallskip
            \includegraphics[width=\textwidth]{NiagaraGroups.png} \\
            \begin{itemize}
                \item Emitter Settings
                \item Emitter Spawn: modules that effect emitter on spawn
                \item Emitter Update: modules that effect emitter over time
                \item Particle Spawn: modules that effect the particle spawn
                \item Particle Update: modules that effect the particle over time
                \item Event Handlers: events that define certain data for one or more emitters to trigger a listener and his behaviour
            \end{itemize}

        \section{Data Interface}
            \begin{itemize}
                \item 
            \end{itemize}

        \section{Events}
            \begin{itemize}
                \item only on CPU
                \item needs \code{Requires Persistant IDs}
                \item Can be placed in any module to share data
            \end{itemize}
\bigskip
            \underline{List of Events:}
            \begin{itemize}
                \item Location Events
                \item Death Events
                \item Collision Events
            \end{itemize}

            \subsection{Event Handlers}
                \includegraphics[width=\textwidth]{EventHandlerProperties.png}
                \underline{Event Handler 'Properties':}
                \begin{itemize}
                    \item defines the basics of the Handler
                    \item Source: with a dropdown of all Gernerated Event modules
                    \item Execution Mode: 
                \end{itemize}
\bigskip
                \underline{Event Handler 'Receive Event':}
                \begin{itemize}
                    \item 
                \end{itemize}

            \subsection{Collision example}
                \begin{itemize}
                    \item use \code{Generate Collision event} on one emitter
                    \item add \code{Event Handler} with \code{Source: Collision Event} on another
                    \item $\hookrightarrow$ add \code{Receive: Collision Event} to get actual collision-data
                \end{itemize}

            \subsection{Useful Nodes}
                \begin{itemize}
                    \item Add Velocity
                    \item Add Acceleration
                    \item Drag
                    \item PRESS SPACE BAR TO RESET EMITTER
                \end{itemize}
                
            \subsection{Side Notes}
            \begin{itemize}
                \item Events are only supported by CPU simulations (19.5)
                \item Simulations can be performed once and the result can be shared with other modules
                \item Drag = Widerstand
                \item Masked material gives hard edges that's why translucent is better when doing fx
            \end{itemize}


    \section{Crowds}
        \begin{itemize}
            \item AI with Behavior-Trees is the standard approach
            \item 
            \item Create an empty 'Niagara System
            \item 
            \item add a new empty emitter to it
            \item 
            \item delete 'Initialize Particle'
            \item replace 'Sprite Renderer' with 'Mesh Renderer'
            \item 
            \item under 'Emitter Update' add 'Spawn Burst Instantaneous'
            \item change the spawn Count to 1
            \item 
            \item under 'Emitter State' change 'Life Cycle Mode' = 'Self' $\rightarrow$ 'Loop Behavior' = 'Once' + 'Loop Duration Mode' = 'Infinite'
            \item 
            \item under 'Particle Update' $\rightarrow$ 'Paricle State' $\rightarrow$ disbale 'Kill Particles When Liftime Has Elapsed'
            \item add the 'New Scratch Pad Module' (:= visual scripting for HLSL)
            \item add 'NFS Play Animation'
            \item 
        \end{itemize}


    \section{Some effects}
        \subsection{Ground Crack}
            \begin{itemize}
                \item Create 2048x2048 image in PS
                \item draw with grey circle
                \item add 'cracks' on the edge pointing outwards
                \item highlight with small bright brush
                \item lessen opacity outwards with erase tool
                \item 
                \item copy paste on second layer
                \item rotate and erase some parts
                \item 
                \item export as png
            \end{itemize}


        \subsection{Flipbook}
            \begin{itemize}
                \item create emitter with following modules and set the properties
                \item \code{Emitter-State} := \code{Loop-Behavior} = Once; \code{Loop-duration} = fixed;
                \item \code{Initialize-Particle} := \code{Lifetime} = shorter $\rightarrow$ plays Flipbook faster
                \item \code{Spawn Burst Instantaneous} := \code{Spawn} count = 1
                \item \code{Sprite Renderer}
                    \begin{itemize}
                        \item \code{Material} := your Flipbook
                        \item \code{Sub Image Size} := number of images in the Flipbook
                    \end{itemize}
                \item \code{SubUVAnimation}
                    \begin{itemize}
                        \item \code{Number of frames} := number of images in Flipbook
                    \end{itemize}
            \end{itemize}


    \section{Debugging}
        \begin{itemize}
            \item Niagara Debugger allows to debug the system by:
            \begin{itemize}
                \item showing playing emitter 
                \item number of particles
                \item showing the bounds
            \end{itemize}
        \end{itemize}
 
