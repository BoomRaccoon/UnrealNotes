\chapter{Audio}
    \section{Basics}
        \begin{itemize}
            \item Sound Cue:
            \item Sound Wave:
            \begin{itemize}
                \item sample sound data
                \item codec can be specified on each sound wave
                \item can be edited in the \code{Sound Wave Editor}
            \end{itemize}
            \item Sound Class:
            \begin{itemize}
                \item group together sound and apply effects for all
                \item can have child classes
                \item Sounds within a Sound Class can be sent to a Submix, which allows multiple sound types to be processed with submix effects at once
            \end{itemize}
            \item Meta Sound:
            \begin{itemize}
                \item 
                \item synthesized sound
            \end{itemize}
            
            \item Submix: 
            \item Mixing: Group active so8und sources and apply effects to them
            \item 
            \item \code{Take Recorder}: you can use the Take Recorder to record audio from a microphone and synch it to the sequence asset
            \item 
            \item \code{Crossfade}: is used to transition smoothly between two audio sources
            \item \code{Envelopes}: time based modulation
            \item \code{Low Frequency Modulator (LFO)}: generate periodic signals
            \item \code{Filters}: can be used to remove unwanted frequencies
        \end{itemize}

    \section{Sound classes}
        \begin{itemize}
            \item can have passive sound mix modifiers
            \item if Audio Modulation plugin is installed:
            \begin{itemize}
                \item possible to add an array of Modulations Settings to all sounds within a Sound Class
            \end{itemize} 
        \end{itemize}

    \section{Submixes}
        \begin{itemize}
            \item 
            \item To mix audio generated from individual sources into a single output buffer
            \item To optimize the application of digital signal processing (DSP) effects to multiple sound sources simultaneously
        \end{itemize}

    \section{Sound-classes vs submixes}
        \begin{itemize}
            \item submixes act on already mixed audio $\rightarrow$ cheaper to apply changes
            \item sound-classes act on each source $\rightarrow$ expensive
        \end{itemize}


    \section{Meta Sounds}
        \begin{itemize}
            \item Output is by default Mono; can be changed in the Asset Properties
            \item you can define extra inputs/outputs
            \item set input: use nodes like \code{Set Float Parameter}
            \item 
            \item can be used to create procedural audio
            \item \code{MetaSoundPatch}: is like a function library in order to use logic accross multiple MetaSoundAssets
            \item coordinate playback of multiple sound assets
            \item synthesize sound using sine, square, triangle, saw oscilator, noise, etc.
            \item 
            \item 
            \item 
            \item performance is better than sound waves
            \item 
            \item have a variable type \code{Trigger} == like a callback
            \item can be fired while simulating
            \item 
            \item own effect nodes can be implemented in C++
        \end{itemize}

    \section{Debug Commands}
        \begin{table}[!htb]
            \begin{tblr}{p{6cm} | p{12cm}}
                \hline
                    Command & Description \\
                \hline
                ShowSoundClassHierarchy & \makecell[l]{...} \\
                ListSoundClassVolumes & \makecell[l]{...} \\
                ListAudioComponents & \makecell[l]{...} \\
                ListSoundDurations & \makecell[l]{...} \\
                SoundTemplateInfo & \makecell[l]{...} \\
                SetBaseSoundMix & \makecell[l]{...} \\
                IsolateDryAudio & \makecell[l]{...} \\
                IsolateReverb & \makecell[l]{...} \\
                TestLPF & \makecell[l]{...} \\
                TestHPF & \makecell[l]{...} \\
                TestLFEBleed & \makecell[l]{...} \\
                DisableLPF & \makecell[l]{...} \\
                DisableHPF & \makecell[l]{...} \\
                DisableRadio & \makecell[l]{...} \\
                EnableRadio & \makecell[l]{...} \\
                ResetSoundState & \makecell[l]{...} \\
                ToggleSpatializationExtension & \makecell[l]{...} \\
                EnableHRTFForAll & \makecell[l]{...} \\
                Solo & \makecell[l]{...} \\
                ClearSolo & \makecell[l]{...} \\
                PlayAllPIEAudio & \makecell[l]{...} \\
                Audio3dVisualize & \makecell[l]{...} \\
                AudioMem & \makecell[l]{...} \\
                AudioSoloSou & \makecell[l]{...} \\
                AudioSoloSo & \makecell[l]{...} \\
                AudioSoloS & \makecell[l]{...} \\
                AudioMixerDeb & \makecell[l]{...} \\
                SoundCla & \makecell[l]{...} \\
                AudioDeb & \makecell[l]{...} \\
                ResetAllDynamicSoundVolumes & \makecell[l]{...} \\
                ResetDynamicSoundVolume & \makecell[l]{...} \\
                GetDynamicSoundVolume & \makecell[l]{...} \\
                SetDynamicSound & \makecell[l]{...} \\
                GetAudioDeviceList & \makecell[l]{...} \\
                

                \hline
            \end{tblr}
        \end{table}
        

    \section{CPP}
        \begin{itemize}
            \item \code{FAudioDeviceManager} := class that manages the audio devices
            \item \code{FAudioDeviceHandle} := strong handle to an audio device
            \item 
        \end{itemize}


