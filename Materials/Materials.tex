\chapter{Materials}
\begin{itemize}
    \item Component Mask node
    \item Saturate instead of Clamp
    \item 
\end{itemize}

    \section{Basics}
        \includegraphics[width=\textwidth]{MaterialMaster.png} \\
        \subsection{Math basics}
            \begin{itemize}
                \item Power := the black values grow (small values reach 0 much faster)
                \item Multiply := 
                \item VertexNormal := the direction the vertex is facing (gives a smooth look (Phong-shading))
                \item FaceNormal := the direction the face is facing (gives a facetted look (Flat-shading))
            \end{itemize}


    \section{Material Expressions}
        \begin{itemize}
            \item 
        \end{itemize}

    \section{Virtual Texturing}
        \begin{itemize}
            \item divides textures into tiles of fixed size (typically 128x128)
            \item analyse the pixels and only stream what's needed
            \item loads only visible parts of the texture
            \item closer $\rightarrow$ high-res
            \item example usage = landscape (you only see a small part of the whole landscape/texture at a time)
        \end{itemize}

        \subsection{Runtime-Virtual-Texturing}
            \begin{itemize}
                \item fills in with GPU instead of texture from disk
                \item render to cache and stream from it
                \item size of virtual-texture
            \end{itemize}
            \textbf{\underline{Worflow}}
            \begin{itemize}
                \item enable 'Virtual Texture Streaming' on the texture
                \item create rvt
                \item create rvt-volume
                \item material with 'Rutime-Virtual-Texture-Output' and 'Runtime-Virtual-Texture-Sample'
                \item 
                \item Far-Blend-Distance
            \end{itemize}

        \subsection{Runtime Virtual Texture}
            \uline{Workflow:}
            \begin{itemize}
                \item create Virtual Textures for the heightmap and colormap of the landscape
                \item depending of the usecase, change the 'Virtual Texture Content'
                \item create Runtime 'Virtual Texture Volume'
                \item drag the RVT into the 'Virtual Texture'-slot in the 'Virtual Texture Volume' details
                \item add the 'Runtime Virtual Texture Output'-Node to the landscape material so you can write the info to VTs
                \item add 2 VT slots on the landscape material and drag in the VTs
                \item add the setup to blend with the VT to the desired Materials
            \end{itemize}

            \begin{figure}
                \includegraphics[width=\textwidth]{RVT_Setup.jpg}
                \caption{Node Setup for the RVT inside the Landscape material}
            \end{figure}

            \begin{figure}
                \includegraphics[width=\textwidth]{VT-Material.jpg}
                \caption{Blend Setup}
            \end{figure}

    \section{Material Domain}
        Define the overall usage of a material and the attributes/nodes it has available (UI will also change preview in the material editor):
        \begin{itemize}
            \item Surface
            \item UI
            \item Post Process
        \end{itemize}

    \section{Transparency/Opcaity/Alpha Map}
        Differentiated in vertex-alpha and pixel-alpha. Where vertex alpha is more coarse. \\
        \underline{Methods:}
        \begin{itemize}
            \item Alpha-Blend: Has a wide range of transparency values $\rightarrow$ semi-transparency possible
            \begin{itemize}
                \item Overlapping alpha-blended meshes (paricles/grass) increase fill rate because same screen pixels are drawn again and again
                \item Sorting problems
            \end{itemize}
            \item Alpha-Test: totally-hard-transparency with a threshhold to make mid-values transparent or solid
            \begin{itemize}
                \item no sorting problem
                \item much faster than alpha-blend
                \item no AA $\rightarrow$ pixelly
            \end{itemize}
        \end{itemize}

    \section{Constant Material Instance/Dynamic Material Instance}
        \begin{itemize}
            \item Constant Material Instance: is a material-instance with non-changing parameters
            \item Dynamic Material Instance: is a material-instance with changing parameters. Achieved by BP-Node 'Create Dynamic Material Instance' or code
        \end{itemize}

    \section{World Displacement}
        The object bounds will not be changed. \\


    \section{Vertex Shader}
        \begin{itemize}
            \item in order to execute code on the Vertex-Shader use the \code{VertexInterpolator}-Node 
            \item 
        \end{itemize}

    \section{World Position Offset}


    \section{Render Targets}
        \begin{itemize}
            \item storing buffers for deferred renderer
            \item display complex effects (ripples on water)
        \end{itemize}
        \subsection{Create Render Targets}
            \begin{itemize}
                \item are a asset type
                \item $\curvearrowright$  $\rightarrow$ Render Target
                \item Blueprint:
                \begin{itemize}
                    \item ok
                \end{itemize}
            \end{itemize}

        
    \section{Tessellation}
        \subsection{Flat-Triangles}

        \subsection{PN-Triangles}

    \section{Parallax Occlusion Mapping}
        Is like a Bump-Map but a lot better \\
        Has a dedicated node in the material editor \\


    \section{Vertex-Color-Material}
        \begin{itemize}
            \item Add a 'Vertex Color'-Node to change something (Simplest is to multiply it with the 'Base-Color'-Values)
            \item Then in 'Paint-Mode' make sure Mode = Colors (Not Blend Weights)
            \item Select the channel you want to paint on !!!
        \end{itemize}

    
    \section{Material Layers}
        \uline{Solves what problem?:} 
        \begin{itemize}
            \item per-pixel control over where Materials are placed, then use layered Material functions or Material Layers.
            \item Simplify large and complex setups into custom nodes to make these operations reusable and graphs using them esier to read
        \end{itemize}
        \uline{When Shouldn't you use them?}
        \begin{itemize}
            \item 
            \item can be heavy in terms of performance, if the Materials used in layer functions are complex themselves
            \item all your layers are rendered simultaneously, and then blended $\rightarrow$
            \begin{itemize}
                \item you have four layers in a Material
                \item engine tests for each pixel of object to see which is blended
                \item then reject any not in use
            \end{itemize} 
            \item Layered Material is generally too heavy for this to be used on mobile platforms
            \item more Material Elements increase draw calls, but are generally much more efficient $\rightarrow$ multiple Materials instead of using a Layered Material
        \end{itemize}

        \begin{itemize}
            \item create a 'Material Layer asset' and add 'MakeMaterialAttributes' node == where you define parameters and where they go
            \item create 'Material Layer Blend Asset' == where the actual blending is defined
            \item Create 'Material'
            \item add 'Material Attribute Layers' Node 
            \item create 'Material Instance' of the Base Material and enable the 'Use Material Attributes' property
            \item use 'Layer Parameters'
        \end{itemize}

        \begin{itemize}
            \item Create a new Material Function and edit the node graph to perfection. This function will act as a layer when you call it in your base Material.
            \item Connect your node network to a new Make Material Attributes node, and connect it to the Function output.
            \item Save the Material Function.
            \item Repeat this process for any other Material Function layers you wish to create.
            \item Create a new Material and open it in the Material Editor.
            \item Drag your Material Functions from the Content Browser into the new Material to use as layers.
            \item Blend multiple Material Functions together using the Material Layer Blend functions. 
        \end{itemize}

        \begin{figure}
            \includegraphics[width=\textwidth]{MaterialLayerAssetGraph.png}
            \caption{Node Setup for a 'Material Layer'-Asset}
            %\label{fig:MaterialLayer}
        \end{figure}

        \begin{figure}
            \includegraphics[width=\textwidth]{MaterialLayerBlendAssetGraph.png}
            \caption{Basic 'Material Blend' setup}
            %\label{fig:MaterialBlend}
        \end{figure}


    \section{Post Process Material}
        \begin{itemize}
            \item are used to apply effects to the screen after the scene has been rendered
            \item NOT for effects like color correction or adjustments, bloom, depth of field, and various other effects, you should use the settings inherent to the Post Process Volume
            \item \code{Material Domain} : has to be set to \code{Post Process}
            \item \code{Blendable Location}:
            \begin{itemize}
                \item \code{Before Tonemapping}:  	
                        all lighting is provided in HDR with scene color when Scene Texture expression's Post Process Input 0
                        is used. It fixes issues with temporal anti-aliasing (TAA) and GBuffer lookups. For example,
                        issues that can happen when using depth and normals.
                \item \code{After Tonemapping}:
                    This option indicates that post processing will take place after tonemapping
                    and color grading has been completed. It is the preferred location \textbf{for performance}
                    since the color is LDR and requires less precision and bandwidth. When this option is selected,
                    the SceneTexture expression's Post Process Inputs 2 and 3 are used to control where
                    Scene Color is in the pipeline.
                    Input 2 applies scene color before tonemapping.
                    Input 3 applies scene color after tonemapping.
                \item \code{Before Translucency}: This is even earlier in the pipeline than 'Before Tonemapping' before translucency was combined with the scene color. Note that SeparateTranslucency is composited later than normal translucency.
                \item \code{Replacing the Tonemapper}: PostProcessInput0 provides the HDR scene color, PostProcessInput1 has the SeparateTranslucency (Alpha is mask), PostprocessInput2 has the low resolution bloom input.
            \end{itemize}
            \item \code{Custom Depth} := adds extra draw calls; can be actiuvated on the mesh
        \end{itemize}

        \subsection{Useful nodes}
            \begin{itemize}
                \item \code{SceneTexture}: to access the scene color, depth, normals, etc.
                \item \code{SceneColor}: to access the scene color
                \item \code{SceneDepth}: to access the scene depth
                \item \code{ViewSize}: gives the size of the viewport
            \end{itemize}


        \subsection{Creating a outline}
            \begin{itemize}
                \item create \code{PostProcess Volume}
                \item set \code{Extends Infinite}
                \item 
                \item \code{Actors} need \code{Render Custom Depth Pass} to be enabled
                \item 
                \item Create \code{Material} for the postprocess
                \item use \code{Scene-Texture}-Node
                \item set \code{Scene-Texture-Id} to \code{CustomDepth}
                \item Divide the CustomDepth by a big number
                \item clamp it between 0-1
                \item 
            \end{itemize}

        

    \section{Decal}
        Project Settings $\rightarrow$ Engine $\rightarrow$ Rendering $\rightarrow$ DBuffer Decals \\
        CameraDirectionVector * Offset into 'World Position Offset' \\
        Blend Modes \\
        Receive Decals Decal \\
        Sort Order \\
        Animated Decal Material \\

    \section{Landscape Material}
        \subsection{Basic Setup}
            \begin{itemize}
                \item create material
                \item create material
            \end{itemize}
        %\label{material_landscape}
        \subsection{Landscape specific nodes}
            \begin{itemize}
                \item LandscapeLayerBlend: enables to blend together multiple textures or material networks to be used as landscape layers.
                \item LandscapeGrassOutput
                \item LandscapeLayerSample
                \item LandscapeLayerSwitch: to exclude material operations if a layer is not present (weight=0)
                \item LandscapeLayerWeight: allows material network blending, based on the weight on the landscape material (layer weight as blending alpha)
                \item LandscapeVisibilityMask: to add holes in a landscape (connected to opacity mask material input)
            \end{itemize}

        \subsection{Blend Modes (LandscapeLayerBlend)}
            \underline{LB Weight Blend:}
            \begin{itemize}
                \item if layer coming from an external program (ex. world machine) 
                \item painting layers independently from another
                \item without layer order
            \end{itemize}
            \underline{LB Alpha Blend:}
            \begin{itemize}
                \item painting in detail
                \item defined layer order
                \item ex. snow over rock and grass
            \end{itemize}
            \underline{LB Height Blend:}
            \begin{itemize}
                \item like LB Weight Blend
                \item adds detail to the transition between layers based on height map
                \item ex. dirt in the gaps of rocks
            \end{itemize}

        \subsection{Grass}
        \subsection{Auto Landscape}
            \begin{itemize}
                \item Select 3 materials from the 'Auto\_Landscape' 'Materials'-Folder
                \item migrate them into the desired project
                \item create a material instance of the 'M\_Auto\_Landscape'
                \item add textures to the materials 'Material\_A' (slopes) 'Material\_B' (Grass) ...
                \item adjust Material properties like tiling, Cell Bombing, Specular, Triplanar Projection (Slopes), Use Normal Map Blend, Z\_Slope\_Bias
                \item Add Foliage:
                \begin{itemize}
                    \item Use imported 'Landscape\_Grass\_Type'-Asset or create a new one
                \end{itemize}
            \end{itemize}
             \includegraphics[width=\textwidth]{Auto_Landscape_1.jpg}

    \section{Physics Material}
        \begin{itemize}
            \item describes the properties of a material
            \item is set on in the details of the material output under 'Physical Material' 
        \end{itemize}

    \section{Example setups}
        \subsection{Moving Distortion (Water effect)}
            The basic concept is that the uv-coordinates move around
            in order to make it look like the surface is moving. \\
            \includegraphics[width=\textwidth]{MovingDistortionShader.png} \\


        \subsection{Swaying Grass}
            \begin{itemize}
                \item Use 'SimpleGrassWind' node
                \begin{itemize}
                    \item WindWeight: grayscale map to specifying influence weight
                    \item AdditionalWPO: takes in additional world position offset network or function
                \end{itemize} 
            \end{itemize}


        \subsection{Glass}
            \uline{High-Quality}
            \begin{itemize}
                \item \code{Blend Mode} = \code{Translucent}
                \item 
                \item 
            \end{itemize}

            \uline{Low-Quality}
            \begin{itemize}
                \item \code{Blend Mode} = \code{Additive} or \code{Modulate}
            \end{itemize}
    \section{Hotkeys}
        \begin{itemize}
            \item T: TextureSample
            \item U: TextureCoordinates
            \item 1-4: Vector
            \item A: Add
            \item M: Multiply
            \item D: Divide
            \item S: Skalar Parameter
            \item CTRL: click on output to drag all connectors
            \item O: 1-x
        \end{itemize}

    \section{Node Reference}
        \begin{table}[H]
            \begin{tabular}{|c|c|}
                \hline
                    Node & Description \\
                \hline
                    Mask & Seperate values (for example rgb or a V3) \\
                    CheapContrast & negative: moves to white, positive: moves to black; works only on one channel \\
                    ChecpContrastRGB & works on all channels \\
                    ComponentMask & specify which channels you want to use \\
                    Clamp & Min-Max \\
                    LinearInterpolate & Output value between A and B depending on the Alpha \\
                    If & ... \\
                \hline
            \end{tabular}
        \end{table}


    \section{Optimization for older hardware and mobile}
        \begin{itemize}
            \item In order to adjust a material to older hardware use the 'FEATURE LEVEL SWITCH'-node
            \item there you can input different calculations for different feature levels
        \end{itemize}


    \section{Substance Designer}
    \begin{itemize}
        \item Create Material in substance designer
        \item expose parameters you want to be editable
        \item export sbsar
        \item import into unreal 
    \end{itemize}


    \section{Curve Atlas (LUT)}
        \begin{itemize}
            \item used to put different gradients into one texture
            \item no runtime support
            \item should use size with power of 2
        \end{itemize}


    \section{Substrate}
        \begin{itemize}
            \item has to be enabled under \code{Plugins}
            \item is a new way to render materials
            \item makes working with translucency more straightforward 
        \end{itemize}

        \subsection{Basics}
            \begin{itemize}
                \item has a root node like the legacy materials where you can set material domain, blend mode, shading model, etc.
                \item a \code{Substrate Shading Models}-Node (Substrate-Slab-BSDF) is the connected to the \code{Front Material}-Input on the root-node
                \item  
                \item a \code{Slab} is the basic building block of a substrate material
                \item it consists of 2 parts:
                \begin{itemize}
                    \item \code{Interface}: is the boundry that interacts with light and is defined by defined by the \code{Roughness}, \code{Normal}, \code{Diffuse} \code{Albedo}, \code{F0} and \code{F90} 
                    \item \code{Medium}: is beneath the interface where light is absorbed and scattered and is defined by Mean-Free-Path
                \end{itemize}
                \item Node-Types:
                \begin{itemize}
                    \item \code{BSDFs}: These nodes represent most types of surfaces, from simple materials to more complex ones like hair, eyes, and water.
                    \item \code{Operators}: These nodes mix and layer multiple Substrate Slab BSDFs to create complex and varied surfaces.
                    \item \code{Building-Blocks}: These nodes translate common material types for use with Substrate, like creating a coated layer or the default legacy material shading model of Unreal Engine.
                    \item \code{Extras}: These nodes define a Material Domain for a Substrate Material, and are directly analogous to their legacy Material Domain namesakes.
                    \item \code{Helpers}: These nodes are used to do some conversion within the material, such as mapping transmittance to Mean Free Path for a Substrate Slab.
                \end{itemize}
            \end{itemize}
            \begin{figure}
                \includegraphics[width=\textwidth]{substrate-slab-composition.png}
                \caption{1 = Interface, 2 = Medium}
                \label{}
            \end{figure}
            
            \subsubsection{BSDFs}
                \begin{figure}[!htb]
                    \includegraphics[width=\textwidth]{substrate-bsdf-nodes.png}
                    \caption{Available bsdf nodes}
                    \label{}
                \end{figure}
                \begin{table}[!htb]
                    \begin{tblr}{p{4cm} | p{12cm}}
                        \hline
                            Substrate-Slab-Input& Description \\
                        \hline
                            Diffuse Albedo      & \makecell[l]{ percentage of light reflected as diffuse from a surface. \\
                                                                similar to the local base color of the medium \\
                                                                default value = 0.18.} \\
                            F0                  & \makecell[l]{ color and brightness of the specular highlight where the surface is \\
                                                                perpendicular to the camera. dielectric Materials (non-metals) = 0 - 0.08 \\
                                                                metallic Materials up to 1. Gemstones are in a range up to around 0.16.} \\
                            F90                 & \makecell[l]{ color of the specular highlight where the surface normal is 90 degrees from the camera \\
                                                                Only hue and saturation are perceived, as brightness is fixed at 1.0 \\
                                                                This fades to black as F0 drops below 0.02.} \\
                            Roughness           & \makecell[l]{ Defines the roughness of the surface. The default value is 0.5.} \\
                            Anisotropy          & \makecell[l]{ Controls the anisotropy direction of the Material \\
                                                                (-1: highlight aligned to the bi-tangent, 1: highlight is aligned with the tangent).} \\
                            Normal              & \makecell[l]{ Defines the normal map of the surface.} \\
                            Tangent             & \makecell[l]{ Take a surface tangent as input. The normal is considered tangent or world space \\
                                                                according to the space properties on the material root node. \\
                                                                This input defines the shading tangent per-pixel.} \\
                            SSS MFP             & \makecell[l]{ Only used when there is no SS-Profile-Asset assigned to the root node \\
                                                                Defines the distance light travels through the medium before being absorbed or scattered. \\
                                                                The default value is 1.} \\
                            SSS MFP Scale       & \makecell[l]{ density of the material $\rightarrow$ influences the absorption and scattering of light by the Material. \\
                                                                defines the average distance at which a photon interacts with a particle of matter \\
                                                                controlled per color channel. } \\
                            SSS Phase Anisotropy& \makecell[l]{ Positive values elongate the phase function along the light direction $\rightarrow$ forward scattering.
                                                                Negative values elongate the function backward along the light direction $\rightarrow$ back scattering.} \\
                            Emissive Color      & \makecell[l]{ Defines the color of the light emitted by the Material.} \\
                            Second Roughness    & \makecell[l]{ Controls the roughness of a secondary specular lobe \\
                                                                does not influence diffuse roughness } \\
                            Second Roughness Weight & \makecell[l]{ The mix factor between the primary and secondary specular lobe \\
                                                                    0 renders the primary lobe only 1.0 renders the secondary lobe only.} \\
                            Fuzz Roughness      & \makecell[l]{ Controls the roughness of the fuzz layer \\
                                                                default value = Roughness-Input} \\
                            Fuzz Amount         & \makecell[l]{ Adds a fuzz-like layer at the interface, causing color retroreflectivity \\
                                                                amount of fuzz applied on top of a surface layer. Usually used to create fabric Materials.} \\
                            Fuzz Color          & \makecell[l]{ Defines the color of the fuzz layer.} \\
                            Glint Density       & \makecell[l]{ The logarithm representation of micro facet density on the surface of a material \\
                                                                Requires r.Substrate.Glints=1 to be set in the ConsoleVariables.ini configuration file } \\
                            Glint UVs           & \makecell[l]{ The UVs used to sample the glint texture \\
                                                                Requires r.Substrate.Glints=1 to be set in the ConsoleVariables.ini configuration file } \\
                        \hline
                    \end{tblr}
                \caption{ *MFP = Mean-Free-Path}  
                \end{table}
            
            \subsubsection{Operators}
                \begin{figure}
                    \includegraphics[width=\textwidth]{substrate-operator-nodes.png}
                    \caption{only for Substrate-Slab-BSDF and Substrate-Simple-Clear-Coat}
                    \label{}
                \end{figure}
                
                \begin{table}[!htb]
                    \begin{tblr}{p{6cm} | p{12cm}}
                        \hline
                            Operator & Description \\
                        \hline
                            Substrate Coverage Weight   & \makecell[l]{ input from a Slab and controls the amount of coverage \\
                                                                        Reducing the weight reduces the coverage of matter of the slab, \\
                                                                        meaning you will see through to the matter underneath \\
                                                                        should be used in conjunction with the Substrate Vertical Layer operator \\
                                                                        to have opaque matter on top of another, like dust and dirt layers\\
                                                                        where you want to control how much they cover the surface below.} \\
                            Substrate Vertical Layer    & \makecell[l]{ input from two Slabs: a Top and Bottom layer. \\
                                                                        The bottom Slab is coated by the top Slab with the bottom layer's appearance \\
                                                                        influenced by the properties of the top layer. \\
                                                                        Use the Top Thickness input to control how thick the top layer is over the bottom. \\
                                                                        ideal for creating car paints, wood varnishes, and wetness on a surface.} \\
                            Substrate Horizontal Blend  & \makecell[l]{input from two Slabs: a Background and Foreground. The Mix input controls how much these two Slabs mix together using a linear interpolation.} \\
                            Substrate Add               & \makecell[l]{input from two Slabs and adds them together. The material created is not physically plausible because it creates more outgoing energy from the surface than incoming energy} \\
                        \hline
                    \end{tblr}
                \caption{ caption }  
                \end{table}

                \begin{itemize}
                    \item \code{Use Parameter Blending}:    Operator nodes include an option to blend their background and foreground into
                                                            a single material when toggling on Use Parameter Blending. Because Substrate Operators
                                                            can create complex material appearances by mixing and layering slabs together, their expense
                                                            at runtime (primarily due to ligthing evaluation) can be costly to performance.
                                                            Parameter blending is an optimization that trades expensive lighting evaluation
                                                            for runtime performance and less costly lighting evaluation.
                \end{itemize}
                
                
            
    \section{HLSL}

        \begin{itemize}
            \item High-Level Shader Language := C-like language
            \item 
        \end{itemize}