\chapter{Plugins}

    \section{Unsorted Notes}
        \begin{itemize}
            \item a game feature plugin "knows" about the game but the game doesn't know about game features
            \item a game knows about plugins but plugins don't know about the game
            \item a game feature knows about a plugin since it can depend on
        \end{itemize}

    \section{Enable/Disable Default plugin}
        \begin{itemize}
            \item simply edit the \code{.uplugin-file}
            \item add/remove \code{\"EnabledByDefault\": true,}
        \end{itemize}

    \section{Plugin Structure}
        \uline{\textbf{In File-Explorer:}}
        \begin{itemize}
            \item Binaries: compiled binaries (dll and debug databases)
            \item Content:
            \item Intermediat: contains temporary files
            \item Source: contains the source code
            \item Resources: files that the plugin needs (for example the icon \colorbox{mygray}{\lstinline{Icon128.png}} with a size of 128x128)
            \item 
            \item $[PluginName].uplugin$: JSON-file containing the information of the plugin and used by the engine to register the plugin
        \end{itemize}

        \uline{\textbf{Plugin-Locations:}}
        \begin{itemize}
            \item Engine: $[UE4 Root]/Engine/Plugins/[Plugin Name]/$
            \item Game: $[Project Root]/Plugins/[PluginName]/$
        \end{itemize}

        \uline{\textbf{Plugin-Configuration-Location:}}
        \begin{itemize}
            \item Engine plugins: $[PluginName/]/Config/Base/[PluginName].ini$
            \item Game plugins: $[PluginName]/Config/Default/[PluginName].ini$
        \end{itemize}

        \uline{\textbf{Possible Plugin Dependencies}} \\
        \includegraphics[width=\textwidth]{PluginDependencies.jpg}

        \section{Create Plugins}
            How to create new plugins: \\
            \includegraphics[width=\textwidth]{CreateNewPlugin.png} \\
            Everything will be setup if created this way \\
            \underline{Filestructure:}
            \begin{itemize}
                \item Will be placed in the Plugin folder of the project
                \item has the basic structure
                \item code itself is placed in 'Source'
                \item 'Source' $\rightarrow$ 'Public' + 'Private'
                \item every module needs it's own build file
                \item \textbf{Content-Plugin:} "CanContainContent" setting within the Plugin's descriptor must be set to "true"
            \end{itemize}
            
            \uline{\textbf{Types:}}
            \begin{itemize}
                \item Runtime: will always be loaded (even in shipped games)
                \item RuntimeNoCommandlet:
                \item Developer: development runtime or editor builds but never in shipped games
                \item Editor: only when editor is starting up
                \item EditorNoCommandlet:
                \item Program:
            \end{itemize}

    \section{Plugin-Descriptor-File}
        \begin{itemize}
            \item has the \colorbox{mygray}{\lstinline{.uplugin}}-extension
            \item JSON-format
            \item 
        \end{itemize}
        \uline{\textbf{Example of a plugin descriptor-file}}
        \begin{lstlisting}
    {
        "FileVersion" : 3,
        "Version" : 1,
        "VersionName" : "1.0",
        "FriendlyName" : "UObject Example Plugin",
        "Description" : "An example of a plugin which declares its own UObject type.  This can be used as a starting point when creating your own plugin.",
        "Category" : "Examples",
        "CreatedBy" : "Epic Games, Inc.",
        "CreatedByURL" : "http://epicgames.com",
        "DocsURL" : "",
        "MarketplaceURL" : "",
        "SupportURL" : "",
        "EnabledByDefault" : true,
        "CanContainContent" : false,
        "IsBetaVersion" : false,
        "Installed" : false,
        "Modules" :
        [
            {
                "Name" : "UObjectPlugin",
                "Type" : "Developer",
                "LoadingPhase" : "Default"
            }
        ]
    }
\end{lstlisting}

        \subsection{Module-Descriptors}
            \href{https://docs.unrealengine.com/4.27/en-US/API/Runtime/Projects/FModuleDescriptor/}{off. API}
            \begin{itemize}
                \item valid type options are:
                \begin{itemize}
                    \item Runtime := even in shipped games
                    \item RuntimeNoCommandlet
                    \item Developer := development build or editor build
                    \item Editor
                    \item EditorNoCommandlet
                    \item Program
                \end{itemize}
            \end{itemize}
        \uline{\textbf{Example of a Module-Descriptor}}
        \begin{lstlisting}
    {
        "Name" : "UObjectPlugin",
        "Type" : "Developer"
        "LoadingPhase" : "Default"
    }
        \end{lstlisting}

    \section{Bare Minimum}
        \begin{itemize}
            \item EditorModule has to derive from IModuleInterface
            \item implement 'StartupModule()' \& 'ShutdownModule()'
            \item 
        \end{itemize}

    \section{Useful Modules}
        \begin{itemize}
            \item UnrealEd:
            \item SlateCore
            \item Slate
            \item ContentBrowser
            \item PropertyEditor
            \item LevelEditor
            \item DetailCustomizations
        \end{itemize}

    \section{Create a Asset Type}
        \begin{itemize}
            \item First declare a assset types C++ class
            \item add factories to create instances of the asset
            \item customize asset appearence
            \item add asset specific content browser action
            \item add custom asset editor UI
        \end{itemize}

    \section{Add Factory}
        \begin{itemize}
            \item UFactory
            \item different types: Content Browser Context Menu, Content Browser Drag \& Drop
        \end{itemize}

    \section{Publish a Plugin}
        \begin{itemize}
            \item files/folders you will have to send to epic:
            \begin{itemize}
                \item \code{YourPlugin.uplugin}-file
                 \item \code{Resources}-folder
                 \item \code{Source}-folder
                 \item no build files
            \end{itemize}
        \end{itemize}