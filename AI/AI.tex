\chapter{AI}%TODO:
    \section{Basics}
        \begin{itemize}
            \item A pawn can be controlled by a player or AI
            \item AI needs: a component that specifies what it should do (Behavior tree) \& AIController
            \item AIController: observes the world and makes decisions based on it
            \item 
            \item \code{PawnSensing} := older
            \item \code{Perception-System} := newer
            \item
        \end{itemize}

        \uline{AI-Classification}
            AIs can be differentiated in order to optimize:
            \begin{itemize}
                \item Low AIs:
                \begin{itemize}
                    \item super low poly
                    \item gives the illusion of being fully featured
                \end{itemize}
                \item Medium AIs:
                \begin{itemize}
                    \item clustered
                    \item running simpler animations
                    \item low poly
                \end{itemize}
                \item High AIs:
                \begin{itemize}
                    \item Smart
                    \item complex animations
                \end{itemize}
                \item Culled AIs:
                \begin{itemize}
                    \item are not visible by player
                    \item can stop any dynamic behavior
                \end{itemize}

            \end{itemize}

            \subsection{Stop AI/BT}
                 \begin{itemize}
                    \item best solution: in order to be able to resume the AI you use \code{LockResources}
                    \begin{itemize}
                        \item \code{static void LockResources(UBrainComponent* Brain) { Brain->LockResource(EAIRequestPriority::Logic);}}
                        \item \code{static void LockPathFollowing(AAIController* Controller) { Controller->GetPathFollowingComponent()->LockResource(EAIRequestPriority::Logic); }}
                    \end{itemize}
                    \item can be done by unpossessing the pawn
                    \item OR you use the  \code{Stop Logic}-node from AIController $\rightarrow$ BrainComponent
                 \end{itemize}
                  \includegraphics[width=\textwidth]{StopBT.png}


    \section{NavMesh}
        \begin{itemize}
            \item stop meshes from influencing $\rightarrow$ totally ignore in navigation (invisible for AI)
            \item and also navigation section of the static mesh
            \item put the aiperception on the controller (so it is shared between the pawns?)
        \end{itemize}


    \section{Behavior Trees}
        Stores information about the states you can be in and actions to perform depending on those states.
        There are 2 types of Nodes
        \begin{itemize}
            \item Root: stores properties of the Behavior-Tree(BT); currently only the \code{Blackboard} (BB)
            \item Composite: Define the root of a branch and the base rules for how that branch is executed:
            \begin{itemize}
                \item Selector: executes from left-to-right until an execution is successfull and goes back to parent composite afterwards to continue flow
                \item Sequence: executes branches from left-to-right until it fails
                \item Simple Parallel: has 2 connections; 1. can only be a Task 2. is the "background branch"
            \end{itemize}
            \item Decorator: is added to other nodes and adds conditions under which the node should be executed
            \item Service: runs in the background of Tasks\&Composites at a frequency until the subtree finishes; used for checks and to update the BB
            \item Task: the actual behavior that should be executed (Move, Attack, ...)
            \begin{itemize}
                \item a dedicated graph to add functionality via BPs
                \item "Event Receive Execute AI" := called when this Task is activated
                \item "Event Receive Abort" := called when this task aborts
                \item "Event Receive Tick" := %TODO:
            \end{itemize} 
        \end{itemize}

    \smallskip
        The position of the nodes matters LEFT MOST IMPORTANT RIGHT LEAST IMPORTANT
    \smallskip
        You have to create tasks for some functionality you might want to have in your game:
        \begin{itemize}
            \item Move to a location
            \item 
        \end{itemize}
    \smallskip
        Useful Nodes in Event Graphes for BTTs
        \begin{itemize}
            \item Event Receive Execute AI
            \item GetRandomReachablePointInRadius
        \end{itemize}

    \section{Blackboard}
        In the BB you can specify variables that will influence the behaviour of the AIController

    \section{EQS}%TODO:
        \begin{itemize}
            \item allows to query the level
            \item components:
            \begin{itemize}
                \item Generators := used to produce the locations or actors that will be tested and weighted
                \item Contexts := used as a frame of reference for any Tests or Generators
            \end{itemize}
        \end{itemize}

    \section{Perception}
        \subsection{Basics}
            \begin{itemize}
                \item main parts:
                \begin{itemize}
                    \item \code{UPawnSensingComponent}
                    \item \code{UPawnNoiseEmitterComponent}
                    \item \code{}
                \end{itemize}
            \end{itemize}
        \subsection{Perception Tree}%TODO:
        \subsection{AI Controllers}%TODO:
            \begin{itemize}
                \item is used with a Pawn
                \item calls the actual BT
                \item contains a 'AIPerception'-Component
                \item 
                \item 
            \end{itemize}
        
        \subsection{AIPerception-Component}
            \begin{itemize}
                \item meant for 'AIController'
                \item 
                \item listens for 'Stimuli'
                \item saves 'Stimuli Sources'
                \item Events:
                \begin{itemize}
                    \item On Perception Updated := when stimuli source is registered
                    \item On Target Perception Updated := for specific stimuli
                    \item On Component Activated := ... 
                    \item On Component Deactivated := ...
                \end{itemize}
                \item 
                \item useable Senses:
                \begin{itemize}
                    \item AIDamage := how to react on 'Event Any Damage', 'Event Point Damage', 'Event Radial Damage'
                    \item AIHearing := how to react to detected sounds by 'Report Noise Event'
                    \item AIPrediction := asks Perception System to supply Requestor with PredictedActor's predicted location in PredictionTime seconds
                    \item AISight := ...
                    \item AITeam := tells if someone on the same team is close by
                    \item AITouch := detect when the AI bumps into something or something bumps into it
                \end{itemize}
                \item specifies the Senses the AI should have
                \item currently affiliation can only be added through C++
            \end{itemize}

        \subsection{AIPerceptionStimuliSource-Component}
            \begin{itemize}
                \item meant for 'Actors' that should influence AI
            \end{itemize}

    \section{Debugging}
        \href{https://docs.unrealengine.com/4.27/en-US/InteractiveExperiences/ArtificialIntelligence/AIDebugging/#perception}{Official DOCs}
        \begin{itemize}
            \item apostrophe to enable AI-debugging in the viewport
            \item 
        \end{itemize}

        \subsection{Visual-logger} \label{sec:visuallogger}
            \begin{itemize}
                \item timeline of events
                \item window $\rightarrow$ developer tools $\rightarrow$ visual logger
                \begin{itemize}
                    \item timeline
                    \item info panel (behaviour tree/what state actors are in)
                    \item debug logger for the recorded timeframe
                \end{itemize}
                    \item 
            \end{itemize}
            \underline{Code:} \\
            \includegraphics[width=\textwidth]{VisualLogger.jpg} \\

            \subsubsection{workflow}
                \begin{itemize}
                    \item use \code{DEFINE_LOG_CATEGORY_STATIC( LogSampleVisualLog, Log, All )}
                    \item use a define guard for the actual logging code \code{#if ENABLE_VISUAL_LOG ... #endif}
                    \item check if it's recording \code{FVisualLogger& Vlog = FVisualLogger.Get();}
                    \item check if it's recording \code{if( Vlog.IsRecording() )}
                    \item actually log using:
                    \begin{itemize}
                        \item \code{UE_VLOG_LOCATION}
                        \item \code{UE_VLOG_SEGMENT}
                        \item \code{UE_VLOG_UELOG}
                        \item more in \code{VisualLogger.h}
                    \end{itemize}
                \end{itemize}

        \subsection{Cheat commands}
            \begin{itemize}
                \item can be defined to define commands
            \end{itemize}


    \section{Smart-Objects}
        \begin{itemize}
            \item \code{GameplayBehavior} : defines what should be done ()
            \item \code{GameplayBehaviorConfig} : used to get a CDO of an assigned Gameplay-Behavior and used inside the \code{SmartObjectDefinition}
            \item \code{SmartObjectDefinition} : defines the behavior for each slot in the SO
            \item \code{SmartObject}(Component) : the component that is placed on an \code{Actor}; holds a reference a \code{SmartObjectDefinition}
            \item \code{}
            \item 
            \item has to call \code{EndBehavior} $\rightarrow$ will release the SO (set occupied to false)
            \item 
        \end{itemize}

        \subsection{Gameplay-Behaviour}
            \uline{Overridable Functions}
            \begin{itemize}
                \item \code{OnFinished}
                \item \code{OnFinishedCharacter}
                \item \code{OnFinishedPawn}
                \item \code{OnTriggered}
                \item \code{OnTriggeredCharacter}
                \item \code{OnTriggeredPawn}
            \end{itemize}


        \subsection{SmartObjectSlotDefinitionData}
            \begin{itemize}
                \item is an extension to the SmartObjectDefinition in the way that using it you can add special data to use
                \item can be accessed through \textbf{FStateTreeExecutionContext}
            \end{itemize}

            \subsubsection{SmartObjectSlotAnnotation}
                \begin{itemize}
                    \item adds functionallity to visualize additional data in the viewport
                    \item examples are:
                    \begin{itemize}
                        \item Entry/Exit-points for the SmartObjectSlot
                    \end{itemize}
                \end{itemize}




        \subsection{General workflow}
            \begin{itemize}
                \item create a \code{GameplayBehavior}
                \item create a \code{GameplayBehaviorConfig}
                \item assign the \code{GameplayBehavior} in the \code{GameplayBehaviorConfig}
                \item define the behavior in the \code{GameplayBehavior} inside one of the Overridable-functions and call \code{EndBehavior}
                \item 
                \item create a \code{DataAsset} of type \code{SmartObjectDefinition}
                \item create a slot in the \code{SmartObjectDefinition} and assign a \code{Default Behavior Definition}
                \item 
                \item create an \code{Actor} and add a \code{SmartObject}-Component
                \item assign the \code{GameplayBehavior} inside of the SO-Component
                \item 
                \item create a \code{BT} and \code{Blackboard}
                \item add a variable of type \code{SO Claim Handle} to the Blackboard
                \item create a \code{BT-Task} to 'Find the SO' and 'Use the SO'
                \item in the 'Find the SO' BT-Task use \code{Event Receive Execute AI} and the \code{Smart-Object-Subsystem} to find SmartObjects in a volume around the AI
                \item 
                \item 
            \end{itemize}


    \section{StateTree}
        \begin{itemize}
            \item Is
            \begin{itemize}
                \item \code{StateTreeSchema}: defines which inputs, evaluators and tasks can be used
                \item \code{FGameplayInteractionContext}: holds the data to perform the action
                \item 
            \end{itemize}
            \item Has
            \begin{itemize}
                \item States
                \item Transitions
                \item Tasks
                \item Evaluators
            \end{itemize}
        \end{itemize}

        \textbf{Example schema}
        \begin{lstlisting}
    UGameplayInteractionStateTreeSchema::UGameplayInteractionStateTreeSchema()
    : ContextActorClass(AActor::StaticClass())
    , SmartObjectActorClass(AActor::StaticClass())
    ,ContextDataDescs({
        {UE::GameplayInteraction::Names::ContextActor, AActor::StaticClass(), FGuid(0xDFB93B9E, 0xEDBE4906, 0x851C66B2, 0x7585FA21)},
        {UE::GameplayInteraction::Names::SmartObjectActor, AActor::StaticClass(), FGuid(0x870E433F, 0x99314B95, 0x982B78B0, 0x1B63BBD1)},
        {UE::GameplayInteraction::Names::SmartObjectClaimedHandle, FSmartObjectClaimHandle::StaticStruct(), FGuid(0x13BAB427, 0x26DB4A4A, 0xBD5F937E, 0xDB39F841)},
        {UE::GameplayInteraction::Names::SlotEntranceHandle, FSmartObjectSlotEntranceHandle::StaticStruct(), FGuid(0x283CBA09, 0x95CD42CF, 0xA11F510E, 0x17CB3530)},
        {UE::GameplayInteraction::Names::AbortContext, FGameplayInteractionAbortContext::StaticStruct(), FGuid(0xEED35411, 0x85E844A0, 0x95BE6DB5, 0xB63F51BC)},
    })
{
}
        \end{lstlisting}

        \subsection{Tasks}
            \begin{itemize}
                \item have a:
                \begin{itemize}
                    \item \code{Enter State}
                    \item \code{Tick}
                    \item \code{Exit State}
                \end{itemize}
                \item be carefull with \code{Finish Task}
            \end{itemize}


        \subsection{Random notes}
            \begin{itemize}
                \item in order to stop some logic use \code{LockResource} and \code{ClearResourceLock}
            \end{itemize}

    \section{Contextual Anim Scene}
        problems (with potential fixes):
        \begin{itemize}
            \item motion warping doesn't work 
            \begin{itemize}
                \item make sure you set the \code{role} and \code{Target Name} when providing the location and rotation in the task
                \item the motion warping window is too short (0.7s is the minimum)
            \end{itemize}
            \item contextual anim scene doesn't play the lead in animation when it's a little short $\rightarrow$ having a longer animation fixes it but looks weird
            \item 
            \item you can define \code{IKTargets} in the \code{Sections}
        \end{itemize}

        \begin{figure}
            \includegraphics[width=\textwidth]{ContextualAnimScene_IKTargets.png}
            \caption{ContextualAnimScene IKTargets can be added in sections}
            \label{}
        \end{figure}

    \section{Crowd}
        Use either DetourCrowd AIController or RVO \\
        RVO (reciprocal velocity obstacle) avoidance uses forces to push characters while they are moving to avoid each other. \\
        Detour works by updating the path to do the avoidance. \\