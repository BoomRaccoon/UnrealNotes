\chapter{World Creation}
    \section{Landscape Settings}
        \begin{itemize}
            \item \code{Components}: Are the large squares; base unit of rendering, visibility calculation and collision; verts of shared component edges are duplicated
            \item \code{Sections}: the smaller little squares inside the components can made out of of either 1 section or 4 sections (2x2); base unit for LOD calculations
            \item \code{Section Size}: is the number of verts for each section (8, 16, 32, 64, 128, 256)
            \item \code{Quads}:
        \end{itemize}

    \section{Landscape Foliage}
        \underline{Types}
        \begin{itemize}
            \item Painted foliage
            \item Landscape grass output (using material to paint on the layers)
            \item procedural foliage volume
        \end{itemize}
        
    \section{Notes}
        \begin{itemize}
            \item Procedural Foliage has no collision
        \end{itemize}
        \begin{itemize}
            \item Landscape
        \end{itemize}
        \underline{Importing heightmap}
        \begin{itemize}
            \item Z-Scale: 1 $\rightarrow$ (1x256) 512cm height difference; 100 $\rightarrow$ (2x256) 512m height difference
            \item White = high; black low
        \end{itemize}
        \begin{itemize}
            \item 
        \end{itemize}
        \underline{Color coding landscape creation}
        \begin{itemize}
            \item Yellow: Landscape Edge
            \item Light Green: Landscape Component Edge
            \item Medium Green: Landscape Section Edge
            \item Dark Green: Landscape Individual Quad
        \end{itemize}
        Each component = 1 draw call $\rightarrow$ Less Components $\rightarrow$ less draw calls \\
\bigskip
        \underline{CREATING THE LANDSCAPE}
        \begin{itemize}
            \item First create a landscape
            \item form it with a height-map in the creation process or
            \item sculp it
            %\item assign a landscape material \hyperref[material_landscape]{creating a landscape material}
        \end{itemize}



        FOLIAGE: DISABLE ALIGN TO NORMAL \\
    \section{Foliage}
        \underline{Basics}
        \begin{itemize}
            \item Overdraw costs more than tries $\rightarrow$ try to make the leafes fill out the whole area
            \item set dynamic shadow cascades to 1
        \end{itemize}

        \underline{Ways to place foliage}
        \begin{itemize}
            \item Foliage Mode Drag\&Drop static meshes you want to spawn
            \item Enable Procedural Spawner in Editor Settings $\rightarrow$ Procedural Foliage Spawner
            \item Landscape Material with Grass Output Type (assign meshes to array) 'Layer Sample Node' with layer name you want Grass-Type to spawn on
        \end{itemize}
        \begin{itemize}
            \item SHIFT+4
            \item Add Foliage
            \item STATIC MESH (instanced mesh in order to make only 1 draw call)
            \item Magnifying Glass in bottom right to edit Static Mesh
            \item Brush Mode
            \item Filter: where is it placable
        \end{itemize}


    \section{Grass on the landscape}
        \underline{Used Nodes}
        \begin{itemize}
            \item Landscape Layer Blend
            \item Landscape Sample (Must have the same Parameter as Layer you want to spawn the Grass on)
            \item Landscape Grass (is also used for rocks)
            \item Landscape Foliage $\rightarrow$ Landscape Grass Type (Culling settings)
        \end{itemize}
        We can restrict foliage spawning to specific Landscape Layers under 'Placement' $\rightarrow$ Advanced $\rightarrow$ 'Landscape Layers' \\
\bigskip            
        You can change the distance the foliage will stop spawning at with the
        console command(smaller number will spawn further):
\begin{lstlisting}
foliage.MinimumScreenSize 0.00001            
\end{lstlisting}

      
